% Created 2018-03-23 vie 16:03
\documentclass[11pt]{article}
\usepackage[utf8]{inputenc}
\usepackage[T1]{fontenc}
\usepackage{fixltx2e}
\usepackage{graphicx}
\usepackage{longtable}
\usepackage{float}
\usepackage{wrapfig}
\usepackage{rotating}
\usepackage[normalem]{ulem}
\usepackage{amsmath}
\usepackage{textcomp}
\usepackage{marvosym}
\usepackage{wasysym}
\usepackage{amssymb}
\usepackage{hyperref}
\tolerance=1000
\author{Mario Román}
\date{\today}
\title{Math Notes}
\hypersetup{
  pdfkeywords={},
  pdfsubject={},
  pdfcreator={Emacs 25.3.2 (Org mode 8.2.10)}}
\begin{document}

\maketitle
\tableofcontents


\section*{Papers \& articles}
\label{sec-1}
\subsection*{{\bfseries\sffamily TODO} Dependent types at work - Ana Bove, Peter Dybjer}
\label{sec-1-1}
\subsubsection*{1. What are dependent types?}
\label{sec-1-1-1}
\subsubsection*{2. Simply Typed Functional Programming in Agda}
\label{sec-1-1-2}
\begin{itemize}
\item 2.1. Truth Values
\label{sec-1-1-2-1}
\item 2.2. Natural numbers
\label{sec-1-1-2-2}
\begin{itemize}
\item Notion of Inductive type
\label{sec-1-1-2-2-1}
\emph{Recursive types} in Haskell are \textbf{inductive types} in constructive type
theory.
\item Notion of Canonical form
\label{sec-1-1-2-2-2}
Elements on canonical form are built up by constructors only. They do not
contain defined functions. Martin-Löf considers \emph{lazy canonical forms}, where
it suffices to begin with a constructor:

\begin{verbatim}
Zero * Zero        -- Not a canonical form
Succ (Zero + Zero) -- Lazy canonical form
Succ (Succ Zero)   -- Canonical form
\end{verbatim}
\end{itemize}

\item 2.3. Lambda Notation and Polymorphism
\label{sec-1-1-2-3}
In Agda we have no type variables, we have families of functions:

\begin{verbatim}
id : (A : Set) -> A -> A
id = \(A : Set) -> \(x : A) -> x
\end{verbatim}

\item 2.4. Implicit Arguments
\label{sec-1-1-2-4}
Implicit arguments are declared by enclosing their typings within curly 
braces.

\item 2.5. Gödel System T
\label{sec-1-1-2-5}
Gödel System T is a system of primitive recursive functionals. All typable
programs in Godel System T terminate. We can only use $\beta$-reduction and the
definitions of:

\begin{verbatim}
true
false
zero
succ
if_then_else
natrec
\end{verbatim}

We can define all primitive recursive functions, but also others such as the
Ackermann fuction.

\item 2.6. Parametrised Types
\label{sec-1-1-2-6}
\item 2.7. Termination-checking
\label{sec-1-1-2-7}
In M-L Type Theory, all recursion is \textbf{primitive recursion}; a structural
recursion on the well-founded data types.

As the Agda's termination-checker has not yet been documented, if Agda will
be used as a system for formalising mathematics rigorously, it is advisable to
stay within a well-specified subset such as Martin-Löf type theory.

In fact, the termination checker will not recognize calls to non-constructors
as smaller arguments. \texttt{(m-n)} will not be recognized as smaller than \texttt{m},
for example.
\end{itemize}

\subsubsection*{3. Dependent Types}
\label{sec-1-1-3}
\begin{itemize}
\item 3.1. Vectors of a given length
\label{sec-1-1-3-1}
We have to alternatives to define vectors of a given length:

\begin{itemize}
\item \textbf{As a Recursive Family}:

\begin{verbatim}
Vec : Set -> Nat -> Set
Vec A zero = Unit
Vec A (succ n) = A X Vec A n
\end{verbatim}

Functions must be written by induction on the length of the vector.

\item \textbf{As an Inductive Family}:

\begin{verbatim}
data Vec (A : Set) : Nat -> Set where
  [] : Vec A zero
  _::_ : {n : Nat} -> A -> Vec A n -> Vec A (succ n)
\end{verbatim}
\end{itemize}

We can use type-checking to define functions that work only over non-empty
vectors, such as \texttt{tail} or \texttt{head}.

\item 3.2. Finite Sets
\label{sec-1-1-3-2}
This data type is useful when we want to access the element at a certain
position in a vector.

\item 3.3. More Inductive Families
\label{sec-1-1-3-3}
\end{itemize}
\subsubsection*{{\bfseries\sffamily TODO} 4. Propositions as Types}
\label{sec-1-1-4}
\subsection*{{\bfseries\sffamily TODO} Monads for functional programming - Philip Wadler}
\label{sec-1-2}
\subsubsection*{1. Introduction}
\label{sec-1-2-1}
\subsubsection*{2. Evaluating monads}
\label{sec-1-2-2}
\begin{itemize}
\item 2.1. Variation zero: The basic evaluator
\label{sec-1-2-2-1}
\item 2.2. Variation one: Exceptions
\label{sec-1-2-2-2}
\item 2.3. Variation two: State
\label{sec-1-2-2-3}
\item 2.4. Variation three: Output
\label{sec-1-2-2-4}
\item 2.5. A monadic evaluator
\label{sec-1-2-2-5}
\item 2.6. Variation zero, revisited: The basic evaluator
\label{sec-1-2-2-6}
\item 2.7. Variation one, revisited: Exceptions
\label{sec-1-2-2-7}
\item 2.8. Variation two, revisited: State
\label{sec-1-2-2-8}
\item 2.9. Variation three, revisited: Output
\label{sec-1-2-2-9}
\end{itemize}
\subsubsection*{3. Monad Laws}
\label{sec-1-2-3}
Son equivalentes \texttt{return,join} y \texttt{return,bind}. Y además, desde cualesquiera
de ellos, se define \texttt{map}.
\subsubsection*{4. State}
\label{sec-1-2-4}
\begin{itemize}
\item 4.1. Arrays
\label{sec-1-2-4-1}
\item 4.2. Array transformers
\label{sec-1-2-4-2}
\item 4.3. Array readers
\label{sec-1-2-4-3}
Conmutative monads.
\item 4.4. Conclusion
\label{sec-1-2-4-4}
\end{itemize}
\subsubsection*{{\bfseries\sffamily TODO} 5. Parsers}
\label{sec-1-2-5}
\subsection*{{\bfseries\sffamily TODO} P?=NP - Scott Aaronson}
\label{sec-1-3}
\subsection*{Classification of Surfaces - Chen Hui George Tao}
\label{sec-1-4}
\subsubsection*{1. Introducción}
\label{sec-1-4-1}
Vamos a demostrar que todas las superficies compactas son homeomorfas
a la esfera, la suma conexa de toros o la suma conexa de planos proyectivos.

\subsubsection*{2. Superficies}
\label{sec-1-4-2}
\begin{itemize}
\item Superficies
\label{sec-1-4-2-1}
Una \textbf{superficie} es una 2-variedad. Un espacio Hausdorff contable
localmente homeomorfo a $\mathbb{R}^2$.

\item Idea del artículo
\label{sec-1-4-2-2}
Dado un polígono, si identificamos las aristas en pares, tendremos una
superficie. Veremos que toda superficie se construye a partir de un
polígono con las aristas identificadas.
\end{itemize}

\subsubsection*{3.1. Triangulaciones. Complejos simpliciales}
\label{sec-1-4-3}
\begin{itemize}
\item Simplex
\label{sec-1-4-3-1}
Dados $v_0,\dots,v_k$ en posición general, el \textbf{simplex} que generan es el
conjunto de combinaciones convexas bajo la topología inducida.

\item Complejo simplicial euclídeo
\label{sec-1-4-3-2}
Un \textbf{complejo simplicial} es una colección $K$ de símplices cumpliendo:

\begin{enumerate}
\item Si $\sigma \in K$, cada cara suya está en $K$.
\item Si $\sigma,\tau \in K$, $\sigma \cap \tau$ es vacía o una cara de ambas.
\item Cada punto tiene un entorno que interseca a sólo finitos símplices.
\end{enumerate}

\item Poliedro
\label{sec-1-4-3-3}
La unión de todos los símplices de $K$ es un espacio simplicial llamado
su \textbf{poliedro}, $|K|$.

\item Homomorfismo simplicial
\label{sec-1-4-3-4}
Función continua entre dos poliedros cuya restricción a cada simplex
es afín. Es \textbf{isomorfismo simplicial} cuando es homeomorfismo.
\end{itemize}

\subsubsection*{3.2. Triangulaciones}
\label{sec-1-4-4}
\begin{itemize}
\item Triangulación
\label{sec-1-4-4-1}
Una triangulación es un homeomorfismo entre un espacio topológico
y un espacio simplicial euclídeo.

\item Teorema de Radó
\label{sec-1-4-4-2}
Toda superficie es un poliedro de un complejo simplicial 2-dimensional.
Donde además, cada 1-símplex es cara de dos 2-símplex.

\begin{itemize}
\item Demostración
\label{sec-1-4-4-2-1}
La demostración es larga. La idea es recubrir toda la superficie con
discos regulares y usar el Teorema de Schonflies.
\end{itemize}
\end{itemize}

\subsubsection*{4.1. Presentación poligonal. Polígonos}
\label{sec-1-4-5}
\begin{itemize}
\item Región poligonal
\label{sec-1-4-5-1}
Compacto $P$ del plano cuya frontera es un 1-símplex cumpliendo:

\begin{enumerate}
\item Cada $q$ que no es vértice tiene un entorno $U$ tal que $P \cap U$ es
intersección de $U$ con un plano.
\item Cada $q$ que es vértice tiene un entorno $U$ tal que $P \cap U$ es
intersección de $U$ con dos planos con fronteras intersecando en $q$.
\end{enumerate}

\item Una región poligonal relacionada a pares es una superficie compacta
\label{sec-1-4-5-2}
Sea $P$ región poligonal. Dada una relación que identifique cada 
arista con exactamente otra por isomorfismo simplicial, el
espacio cociente resultante es una superficie compacta.

\begin{itemize}
\item Demostración
\label{sec-1-4-5-2-1}
Sea $M = P/\sim$, con proyección $\pi:P \longrightarrow M$. Por compacidad, $\pi(P) = M$
es compacto. Podemos dividir los puntos de $M$ en:

\begin{itemize}
\item Puntos en una cara
\label{sec-1-4-5-2-1-1}
Como la proyección es homeomorfismo local en el interior del polígono,
tenemos que son localmente euclídeos.

\item Puntos en una arista
\label{sec-1-4-5-2-1-2}
Claramente, existe un entorno sin vértices. Por definición de la
relación, el punto está identificado con exactamente otro y podemos
usar los entornos $V_1,V_2$ que son discos de intersecciones con planos.

Ahora creamos aplicaciones afines $\alpha_1,\alpha_2$ que peguen las dos partes del 
disco en $\mathbb{R}^2$ y las usamos para construir una proyección de $V_1\cup V_2$ a
$\mathbb{R}^2$. Por tener la misma relación de equivalencia que $\pi$, los espacios
cocientes son homeomorfos, y podemos ver que el punto tiene un
entorno euclídeo en este espacio.

\item Vértices
\label{sec-1-4-5-2-1-3}
Repetimos exactamente lo mismo que hemos hecho con la arista pero
sabiendo que cada identificación del vértice nos da un ángulo que
debemos pegar después en $\mathbb{R}^2$.
\end{itemize}
\end{itemize}
\end{itemize}

\subsubsection*{4.2. Presentación poligonal. Suma conexa de superficies}
\label{sec-1-4-6}
\begin{itemize}
\item Suma conexa
\label{sec-1-4-6-1}
Dadas superficies $M_1,M_2$, bolas regulares $B_1,B_2$, y un homeomorfismo
$f : dM_2' \longrightarrow dM_1'$. El espacio que identifica cada punto con su imagen
es la \textbf{suma conexa}.

\item Suma conexa de superficies conexas
\label{sec-1-4-6-2}
La suma conexa de superficies conexas es una superficie conexa.

\begin{itemize}
\item Demostración
\label{sec-1-4-6-2-1}
Debemos ver que es localmente euclídea y Hausdorff. Tomamos como
proyección:

\[
\pi : M_1' \sqcup M_2' \longrightarrow M_1\# M_2
\]

Y tenemos dos tipos de puntos.

\begin{itemize}
\item Puntos en el interior
\label{sec-1-4-6-2-1-1}
Los puntos que no tocan al disco de unión tienen a la proyección
localmente homeomorfa en ellos y por eso son localmente euclídeos.

\item Puntos en el borde
\label{sec-1-4-6-2-1-2}
Tomamos un entorno de ambos puntos tal que contengan los mismos
puntos identificados del borde. Los proyectamos a $\mathbb{R}^2$ pegando
ambos bordes y nos damos cuenta de que es la misma relación de
equivalencia que daría $\pi$, luego son espacios homeomorfos y el
punto en ellos, llevado al $0$, es localmente euclídeo.
\end{itemize}
\end{itemize}
\end{itemize}

\subsubsection*{4.3. Presentación poligonal}
\label{sec-1-4-7}
\begin{itemize}
\item Presentación poligonal
\label{sec-1-4-7-1}
Una \textbf{presentación poligonal} es un conjunto finito con finitas palabras
$W_1,\dots,W_k$, cada una de longitud 3 o mayor.

\item Realización geométrica de una presentación poligonal
\label{sec-1-4-7-2}
La \textbf{realización geométrica} de una presentación poligonal se construye:

\begin{enumerate}
\item Cada palabra $W_i$ da $P_i$, región poligonal de $k$ lados construída del
polígono regular modelo.
\item Damos una biyección de cada símbolo con los lados de $P_i$ en orden.
\item Unimos disjuntamente los $P_i$ e identificamos aristas con el mismo
nombre y homeomorfismos afines.
\end{enumerate}

\item Presentación de superficie
\label{sec-1-4-7-3}
Presentación poligonal donde cada símbolo ocurre exactamente dos veces.

\begin{itemize}
\item La realización de una presentación de superficie es superficie compacta
\label{sec-1-4-7-3-1}
Hemos probado antes que en este caso, obteníamos una \hyperref[sec-1-4-5-2]{superficie compacta}
en la realización.
\end{itemize}

\item Presentaciones topológicamente equivalentes
\label{sec-1-4-7-4}
Dos presentaciones son equivalentes si tienen la misma realización 
geométrica.

\item Toda superficie compacta tiene una presentación de superficie
\label{sec-1-4-7-5}
Toda superficie compacta tiene una presentación de superficie.

\begin{itemize}
\item Demostración
\label{sec-1-4-7-5-1}
Dada una superficie $M$, por triangulación es homeomorfa a un complejo
simplicial donde cada arista es cara de dos símplices. Dado un complejo
simplicial podemos construir una presentación donde:

\begin{itemize}
\item Cada 2-símplex es una palabra de longitud 3.
\item Dos aristas se llaman igual si vienen del mismo símplex.
\end{itemize}

La presentación entonces nos da dos proyecciones desde los polígonos
hasta la realización de la presentación y al símplex.

\begin{itemize}
\item $\pi_K : P_1\sqcup\dots\sqcup P_n \longrightarrow |K|$
\item $\pi_{\cal P} : P_1\sqcup\dots\sqcup P_n \longrightarrow |{\cal P}|$
\end{itemize}

\begin{itemize}
\item Ambas proyecciones identifican los mismos puntos
\label{sec-1-4-7-5-1-1}
Es claro que identifican las mismas aristas por construcción.
Debemos comprobar que identifican los mismos vértices. Sea $v$
un vértice, que debe estar en una arista que debe estar en dos 
2-símplex $\sigma,\sigma'$. Definimos una relación entre 2-símplices si
comparten una arista. Para comprobar que los vértices se mantienen
por una proyección entre aristas, comprobaremos que hay una sola
clase de equivalencia.

Si hubiera dos clases de equivalencia $\{\sigma_i\},\{\tau_i\}$, podemos tomar una
bola suficientemente pequeña (por la condición de finitud de los
complejos simpliciales) para que interseque sólo a símplices 
conteniendo a $v$. Esto nos da una bola homeomorfa a $\mathbb{R}^2$, luego
$U \setminus \{v\}$ es conexo. Podríamos quitar el $v$ en los complejos simpliciales
de ambas clases de equivalencia y serían disconexas.
\end{itemize}
\end{itemize}

\item Extensión de isomorfismo de bordes
\label{sec-1-4-7-6}
Sean $P_1,P_2$ polígonos convexos con $f : bP_1 \longrightarrow bP_2$ isomorfismo simplicial,
entonces se extiende a un homeomorfismo $F : P_1 \longrightarrow P_2$.

\begin{itemize}
\item Demostración
\label{sec-1-4-7-6-1}
Cualquier punto en el interior forma uniéndose con los vértices un
complejo simplicial. Los poliedros de ambos son homeomorfos porque
los complejos simpliciales lo son.
\end{itemize}

\item Las transformaciones elementales dan realizaciones equivalentes
\label{sec-1-4-7-7}
Las transformaciones elementales de las presentaciones dan lugar a
superficies topológicamente equivalentes

\begin{itemize}
\item Reflexión
\label{sec-1-4-7-7-1}
Claramente una aplicación afín de reflexión nos da lo buscado.

\item Rotación
\label{sec-1-4-7-7-2}
La rotación es una aplicación afín que nos da lo buscado.

\item Cortar
\label{sec-1-4-7-7-3}
Tomamos las dos proyecciones de presentación antes y después de
cortar y comprobamos que identifican los mismos puntos.

\item Doblar
\label{sec-1-4-7-7-4}
Tomamos las dos proyecciones y añadimos las aristas que faltan para
comprobar que identifican los mismos puntos.
\end{itemize}

\item Presentación de la suma conexa
\label{sec-1-4-7-8}
La presentación de la suma conexa es la unión de las palabras.

\begin{itemize}
\item Demostración
\label{sec-1-4-7-8-1}
Dadas $W_1,W_2$, cortamos un disco como $W_1c^{-1}b^{-1}a^{-1}$ y $abcW_2$ e 
identificamos las aristas dadas.
\end{itemize}
\end{itemize}

\subsubsection*{5. Teorema de clasificación}
\label{sec-1-4-8}
\begin{itemize}
\item Lema: Botella de Klein
\label{sec-1-4-8-1}
\item Lema: Suma de toro y plano proyectivo
\label{sec-1-4-8-2}
\item Teorema de clasificación
\label{sec-1-4-8-3}
Toda superficie compacta conexa es homeomorfa a una de las siguientes:

\begin{itemize}
\item $\mathbb{S}^2$
\item $\mathbb{T}^{\#n}$
\item $\mathbb{RP}^{2\#n}$
\end{itemize}

\begin{itemize}
\item Demostración
\label{sec-1-4-8-3-1}
Tomamos transformación desde la presentación hasta llegar a la
presentación de un modelo.

\begin{itemize}
\item Paso 1: Una sola cara
\label{sec-1-4-8-3-1-1}
\item Paso 2: Sin pares complementarios adyacentes
\label{sec-1-4-8-3-1-2}
\item Paso 3: Todos los pares retorcidos adyacentes
\label{sec-1-4-8-3-1-3}
\item Paso 4: Identificamos todos los vértices en un punto
\label{sec-1-4-8-3-1-4}
\item Paso 5: Comprobamos que los complementarios están entrelazados
\label{sec-1-4-8-3-1-5}
\item Paso 6: Llevamos los complementarios juntos
\label{sec-1-4-8-3-1-6}
\item Paso 7: Comprobamos que es una presentación modelo
\label{sec-1-4-8-3-1-7}
\end{itemize}
\end{itemize}
\end{itemize}
\subsection*{Koszul Pairs and applications - Pascual Jara, Dragoş Ştefan}
\label{sec-1-5}
\subsubsection*{Introduction}
\label{sec-1-5-1}
\begin{itemize}
\item Koszul ring
\label{sec-1-5-1-1}
\textbf{Koszul ring}. A graded ring $A$ is \textbf{Koszul} if $A^0$ is a semisimple ring 
and it has a resolution $P_\ast$ by projective graded left A-modules such 
that each $P_n$ is generated by homogeneous elements of degree $n$.

\item Graded ring
\label{sec-1-5-1-2}
\textbf{Graded ring}. A ring that is a direct sum of abelian groups:

\[ A = \bigoplus_{n \in \mathbb{N}} A_n\]

such that $A_iA_j \subset A_{i+j}$.

\begin{itemize}
\item Homogeneous Elements
\label{sec-1-5-1-2-1}
A \textbf{homogeneous element} is an element of any factor $A_i$ of the 
decomposition.

\textbf{Example:} A polynomial ring $A = \mathbb{K}[x_1,x_2, \dots]$ is graded with $A_i$ 
being the abelian group of polynomials with only monomials of 
degree $i$.
\end{itemize}

\item Semisimple group
\label{sec-1-5-1-3}
\textbf{Semisimple group}. A group is semisimple if it has no non-trivial 
normal abelian subgroups.

Different uses of this term can be found \href{http://planetmath.org/semisimplegroup}{here}.

\item Semisimple module
\label{sec-1-5-1-4}
\textbf{Semisimple module}. It is a direct sum of simple modules, that is, 
they have no non-zero proper submodules.

\item Semisimple algebra
\label{sec-1-5-1-5}
An associative finite dimensional algebra $A$ is \textbf{semisimple} if
$A$ is a direct product of simple algebras or equivalently, if $A$ has
trivial Jacobson radical.
\end{itemize}

\subsubsection*{1. Almost-koszul pairs}
\label{sec-1-5-2}
\begin{itemize}
\item 1.1. R-rings
\label{sec-1-5-2-1}
\begin{itemize}
\item R-Ring
\label{sec-1-5-2-1-1}
\textbf{R-ring}. Associative and unital algebra. It is an associative and 
unital ring $A$ together with a morphism $u : R \longrightarrow A$.

\item Graded and connected R-rings
\label{sec-1-5-2-1-2}
\textbf{Graded and connected R-rings}. A R-ring is graded if it is equipped 
with a decomposition:

\[A = \bigoplus_{n \in \mathbb{N}} A^n \]

such that multiplicaton $m^{p,q}$ maps $A^p \otimes A^q$ into $A^{p+q}$. It is \textbf{connected} 
when $A_0 = R$. It is \textbf{strongly graded} when $m^{1,p}$ is surjective. We 
call $\pi^n_A$ to the projection of $A$ onto $A^n$.
\end{itemize}

\item 1.2. R-corings
\label{sec-1-5-2-2}
\begin{itemize}
\item Definition of coalgebra
\label{sec-1-5-2-2-1}
A \href{https://en.wikipedia.org/wiki/Coalgebra#Formal_definition}{coalgebra} over a field $K$ is a \textbf{vector space} $V$ together with linear
maps $\Delta : V \longrightarrow V \otimes V$ and $\varepsilon : V \longrightarrow K$ such that:

\begin{enumerate}
\item $(id \otimes \Delta) \circ \Delta = (\Delta \otimes id) \circ \Delta$
\item $(id \otimes \varepsilon) \circ \Delta = id 
    = (\varepsilon \otimes id) \circ \Delta$
\end{enumerate}

Sometimes, the coalgebras use \href{https://en.wikipedia.org/wiki/Coalgebra#Sweedler_notation}{Sweedler notation}.

\item Examples of coalgebras
\label{sec-1-5-2-2-2}
\begin{itemize}
\item The divided power coalgebra
\label{sec-1-5-2-2-2-1}
Consider $K[X]$, the polynomial ring, where we define by linearity:

\[\Delta(X^n) = \sum^n_{k=0} {n \choose k} X^k \otimes X^{n-k}\]

\[ \epsilon(X^n) = \twopartdef{1}{n=0}{0}{n>0}\]

When the structures of algebra and coalgebra are compatible, they
are called \href{https://en.wikipedia.org/wiki/Bialgebra}{bialgebras}.
\end{itemize}

\item R-coring
\label{sec-1-5-2-2-3}
\textbf{R-coring}. Coassociative and counital coalgebra. It is an R-bimodule 
with a \emph{comultiplication} $\Delta : C \longrightarrow C \otimes C$ and 
a \emph{counit} $\epsilon : C \longrightarrow R$.

\item Graded corings
\label{sec-1-5-2-2-4}
\textbf{Graded corings}. Decomposition $C = \bigoplus_{n \in \mathbb{N}} C_n$, 
such that:

\[\Delta(C_n) \subset \bigoplus_{p=0}^n C_p \otimes C_{n-p}\]
\end{itemize}

\item 1.3. Almost-Koszul pair
\label{sec-1-5-2-3}
\textbf{Almost-Koszul pair}. Connected R-ring and R-coring $(A,C)$ with an 
isomorphism $\theta_{C,A} : C_1 \longrightarrow A^1$, that satisfies the relation:

\[ m^{1,1} \circ (\theta_{C,A} \otimes \theta_{C,A}) \circ \Delta_{1,1}
= 0\]

Or, using Sweedler notation, for any $c \in C_2$:

\[ \sum \theta_{C,A}(c_{(1,1)}) \theta_{C,A}(c_{(2,1)}) = 0\]

\item 1.4. Opposite Koszul pair
\label{sec-1-5-2-4}
If $(A,C)$ is a Koszul pair, then $(A^{op},C^{op})$ are Koszul pairs with
respect to:

\[\theta_{C^{op},A^{op}} = \theta_{C,A}\]

\item 1.5. The normalized bar resolution of R
\label{sec-1-5-2-5}
For every strongly graded R-ring A, there is a graded coring C such that
$(A,C)$ is an almost-Koszul pair.

\begin{itemize}
\item The normalized right bar resolution
\label{sec-1-5-2-5-1}
The exact sequence $\beta_\ast^r(A)$:

\[ 0 \longleftarrow 
R \overset{\delta_0}\longleftarrow 
A \overset{\delta_1}\longleftarrow
\overline{A} \otimes A \overset{\delta_2}\longleftarrow
\overline{A} \otimes \overline{A} \otimes A \overset{\delta_3}\longleftarrow
\overline{A} \otimes \overline{A} \otimes \overline{A} \otimes A \longleftarrow
\dots
\]

is called the \textbf{normalized right bar resolution}. Where
the $\delta$ are defined as:

\begin{itemize}
\item $\delta_0 = \pi^0_A$
\item \[ \delta_n(a_1 \otimes \dots \otimes a_n \otimes a_{n+1}) 
      = \sum_{i=1}^n (-1)^i  a_1 \otimes \dots \otimes a_ia_{i+1} \otimes \dots \otimes a_{n+1}\]
\end{itemize}

\item {\bfseries\sffamily TODO} Normalized bar complex
\label{sec-1-5-2-5-2}
\end{itemize}
\end{itemize}

\subsubsection*{2. Koszul Pairs}
\label{sec-1-5-3}

\subsubsection*{3. Hochschild (co)homology of Koszul rings}
\label{sec-1-5-4}
\begin{itemize}
\item 3.1. The cyclic tensor product
\label{sec-1-5-4-1}
\begin{itemize}
\item Enveloping algebra of R
\label{sec-1-5-4-1-1}
The tensor product algebra $R^e = R \otimes_\mathbb{K} R^{op}$ is called the 
\textbf{enveloping algebra} of $R$.
\end{itemize}
\end{itemize}

\subsubsection*{4. Almost-Koszul pairs associated to twisted tensor products}
\label{sec-1-5-5}

\subsubsection*{5. The Hochschlid cohomology of a twisted tensor product}
\label{sec-1-5-6}

\subsection*{The derivative of a regular type is its type of one-hole contexts - Connor McBride}
\label{sec-1-6}
Presented by \href{https://www.youtube.com/watch?v=K7tQsKxC2I8}{Erik Hinton}.

\subsubsection*{Types and fixed points}
\label{sec-1-6-1}
Empty type, unit type, product and other basic types.
We use parametric types with type variables.

Fixed points are used to define types. Naturals are
the fixed point of $Z + S x$. We write the fixed point
of a formula $F$ over a variable $x$ as $\mu x.F$.

\[
\mathtt{Nat} = \mu x. 1 + x
\]

\subsubsection*{Zippers and holes}
\label{sec-1-6-2}
One-hole contexts with respect to some interior type. A zipper is a
one-hole context of a type and the value that was removed.

\subsubsection*{Derivatives}
\label{sec-1-6-3}
To find the type of a context of type $T$ with a hole in place of some
$x$, take the partial derivative of $T$ with respect to $x$.

Partial derivatives with respect of a type variable work directly.

Product and sum rules can be proved. Chain rule can be proved.

\subsubsection*{Recursive derivatives}
\label{sec-1-6-4}

\subsubsection*{Questions}
\label{sec-1-6-5}
Negative and fractional types. Algebraic types and the field of rationals.
Computing on the field of rationals.
\subsection*{From sets to types to categories to sets - Steve Awodey}
\label{sec-1-7}
\subsubsection*{1. Sets to Types}
\label{sec-1-7-1}
\subsubsection*{2. Types to Categories}
\label{sec-1-7-2}
\begin{itemize}
\item Syntactic topos
\label{sec-1-7-2-1}
\end{itemize}
\subsubsection*{3. Categories to Sets}
\label{sec-1-7-3}
\begin{itemize}
\item How to extract an elementary set theory from a topos
\label{sec-1-7-3-1}
\item At least BIST
\label{sec-1-7-3-2}
\end{itemize}
\subsubsection*{{\bfseries\sffamily TODO} 4. Composites}
\label{sec-1-7-4}

\section*{The Catsters}
\label{sec-2}
\subsection*{Adjunctions}
\label{sec-2-1}
Serie de \href{https://www.youtube.com/playlist?list=PL54B49729E5102248}{vídeos} sobre funtores adjuntos.

\subsubsection*{Adjuntions 1}
\label{sec-2-1-1}
Tenemos varias nociones de igualdad entre categorías.

\begin{definition}
\textbf{Isomorfismo de categorías}. Ocurre con dos functores:

\[ \begin{tikzcd}
{\cal C} \arrow[bend left]{r}{F} & {\cal D} \arrow[bend left]{l}{G}
\end{tikzcd}
\]

Tales que $1_C = GF$ y $FG = 1_D$.
\end{definition}

\begin{definition}
\textbf{Equivalencia de categorías}. Ocurre con dos functores:

\[ \begin{tikzcd}
{\cal C} \arrow[bend left]{r}{F} & {\cal D} \arrow[bend left]{l}{G}
\end{tikzcd}
\]

Tales que $1_C \cong GF$ y $FG \cong 1_D$. Entendiendo la isomorfía en la 
categoría de funtores, es decir, una \href{https://ncatlab.org/nlab/show/natural+isomorphism}{isomorfía natural}.
\end{definition}

\begin{definition}
\textbf{Adjunción}. Ocurre con dos functores:

\[ \begin{tikzcd}
{\cal C} \arrow[bend left]{r}{F} & {\cal D} \arrow[bend left]{l}{G}
\end{tikzcd}
\]

Tales que tenemos transformaciones naturales $1_C \overset{\eta}\Longrightarrow GF$ y 
$FG \overset{\epsilon}\Longrightarrow 1_D$ que cumplen las dos identidades triangulares siguientes:

\[ \begin{tikzcd}
F \arrow{r}{\eta} \arrow{dr}{id} & FGF \arrow{d}{\epsilon} \\
 & F
\end{tikzcd}   
\]     \[ \begin{tikzcd}
G \arrow{r}{\eta} \arrow{dr}{id} & GFG \arrow{d}{\epsilon} \\
 & G
\end{tikzcd}
\]
\end{definition}

En este caso escribimos $F \dashv G$, y $F$ es funtor adjunto de $G$.

\subsubsection*{Adjuntions 2}
\label{sec-2-1-2}
Damos una definición equivalente de funtores adjuntos.

\begin{definition}
\textbf{Adjunción}. Una adjunción es un isomorfismo natural:

\[Hom_D(FX,Y) \cong Hom_C(X,GY)\]

Natural sobre $X$ fijado cualquier $Y$ y natural sobre $Y$ fijado 
cualquier $X$. Entendiendo que usamos los funtores contravariantes $Hom(F-,Y)$,
$Hom(-,GY)$ por un lado y los funtores covariantes $Hom(FX,-)$ y $Hom(X,G-)$;
que nos dan los siguientes cuadrados de naturalidad:

\[ \begin{tikzcd}
Hom_D(FX',Y) \arrow{d}[swap]{Hom_D(Ff,Y)} \arrow{r}{\alpha_{X'}} & Hom_C(X',GY) \arrow{d}{Hom_C(f,GY)}\\
Hom_D(FX, Y) \arrow{r}{\alpha_{X}}& Hom_C(X,GY)
\end{tikzcd}
\] 

\[ \begin{tikzcd}
Hom_D(FX,Y) \arrow{d}[swap]{Hom_D(FX,g)} \arrow{r}{\beta_{Y}} & Hom_C(X,GY) \arrow{d}{Hom_C(X,Gf)}\\
Hom_D(FX,Y') \arrow{r}{\beta_{Y'}}& Hom_C(X,GY')
\end{tikzcd}
\] 
\end{definition}

Esta definición es equivalente intuitivamente a la anterior porque
podemos crear $\eta$ y $\epsilon$ desde las identidades usando las
siguientes transformaciones naturales:

\[Hom_D(FX,FX) \cong Hom_C(X,GFX)\]

\[Hom_D(FGY,Y) \cong Hom_C(GY,GY)\]

\subsubsection*{Adjuntions 3}
\label{sec-2-1-3}
Podemos presentar ejemplos de adjunciones.
Los \textbf{funtores libres y de olvido} suelen ser adjuntos. Entre $Set$ y $Monoid$ tenemos:

\[ \begin{tikzcd}
{Set} \arrow[bend left]{r}{Free} & {Monoid} \arrow[bend left]{l}{Forget}
\end{tikzcd}
\]

Con la adjunción $Free \dashv Forget$. 

\begin{theorem}
\textbf{Mónada de una adjunción}. Cada adjunción da lugar a una mónada.
\end{theorem}

Tenemos un funtor $T = GF : {\cal C}  \longrightarrow {\cal C}$. Podemos definir la unidad de
la mónada como la unidad de la adjunción $\eta : 1_C \Longrightarrow T$ y la
multiplicación podemos definirla usando $id \ast \epsilon \ast id : GFGF \Longrightarrow GF$.

Ahora debemos comprobar que cumple los axiomas de mónada. El primero
se obtiene directamente desde los triángulos de la adjunción:

\[ \begin{tikzcd}
T \arrow{r}{T\eta} \arrow{dr}{id} & T^2 \arrow{d}{\mu} \\
 & T
\end{tikzcd}   
\]   \[ \begin{tikzcd}
GF \arrow{r}{GF\eta} \arrow{dr}{id} & GFGF \arrow{d}{G \epsilon F} \\
 & GF
\end{tikzcd}   
\]

Donde el segundo es resultado de aplicar el funtor $G$ a uno de los triángulos conmutativos
de la adjunción. Comprobamos el segundo axioma:

\[ \begin{tikzcd}
T^2 \arrow{d}{\mu} & T \arrow{dl}{id} \arrow{l}[swap]{\eta T} \\
T
\end{tikzcd}   
\]   \[ \begin{tikzcd}
GFGF \arrow{d}{G \epsilon F} & GF \arrow{dl}{id} \arrow{l}[swap]{\eta GF} \\
GF
\end{tikzcd}   
\]

Donde tenemos el resultado de aplicar $F$ por la derecha al otro triángulo conmutativo.

Y finalmente el axioma de conmutatividad de la mónada se comprueba como:

\[ \begin{tikzcd}
T^3 \arrow{d}{T \mu} \arrow{r}{\mu T} & T^2 \arrow{d}{\mu} \\
T^2 \arrow{r}{\mu} & T
\end{tikzcd} \]  \[ \begin{tikzcd}
GFGFGF \arrow{d}{GFG \epsilon F} \arrow{r}{G \epsilon FGF} & GFGF \arrow{d}{G\epsilon F} \\
GFGF \arrow{r}{G \epsilon F} & GF
\end{tikzcd} \] 

Donde el segundo diagrama se obtiene desde la naturalidad de $\epsilon$ aplicando funtores.

\subsubsection*{Adjuntions 4}
\label{sec-2-1-4}
Vamos a probar la igualdad entre las dos definiciones de adjunción.
Supongamos primero que tenemos el isomorfismo natural entre los dos 
conjuntos de morfismos, es decir, tenemos:

\[ (-) : Hom_D(FX,Y) \cong Hom_C(X,GY) \]

Si tomamos ahora los dos cuadrados naturales que teníamos por este 
isomorfismo y tomamos en ellos los casos particulares $Y = FX$ primero,
y $X = GY$ después:

\[ \begin{tikzcd}
Hom_D(FX,FX) \arrow{d}[swap]{\_ \circ Ff} \arrow{r}{(-)} & Hom_C(X,GFX) \arrow{d}{\_\circ f}\\
Hom_D(FX', FX) \arrow{r}{(-)}& Hom_C(X',GFX)
\end{tikzcd}
\]

Si tomamos la identidad $1_{FX}$ y llamamos $\eta_X = \overline{1_{FX}}$, tenemos que
\(\eta \circ f = \overline{Ff}\). Ahora, si damos la vuelta al isomorfismo $(-)$ en este 
diagrama a la vez que hacemos $X = GY$:

\[ \begin{tikzcd}
Hom_D(FGY,Y) \arrow{d}[swap]{\_ \circ Ff}  & Hom_C(GY,GY) \arrow{l}[swap]{(-)} \arrow{d}{\_\circ f}\\
Hom_D(FGY',Y) & Hom_C(GY',GY) \arrow{l}[swap]{(-)}
\end{tikzcd}
\]

Volviendo a tomar la identidad $1_{GY}$ y llamando $\epsilon_Y = \overline{1_{GY}}$, tenemos
$\epsilon \circ Ff = \overline{f}$.

Ahora tomamos el segundo cuadrado natural, y repetimos el mismo
proceso.

\[ \begin{tikzcd}
Hom_D(FX,FX) \arrow{d}[swap]{g \circ \_} \arrow{r}{(-)} & Hom_C(X,GFX) \arrow{d}{Gg\circ \_}\\
Hom_D(FX,FX') \arrow{r}{(-)}& Hom_C(X,GFX')
\end{tikzcd}
\] 

Obteniendo desde la identidad en $FX$ la ecuación $\overline{g} = Gg \circ \eta$. Y volviendo
a dar la vuelta a los isomorfimos llegamos a:

\[ \begin{tikzcd}
Hom_D(FGY,Y) \arrow{d}[swap]{g \circ \_}  & Hom_C(GY,GY) \arrow{l}[swap]{(-)} \arrow{d}{Gg \circ \_}\\
Hom_D(FGY,Y') & \arrow{l}[swap]{(-)} Hom_C(GY,GY')
\end{tikzcd}
\]

Obteniendo finalmente $\overline{Gg} = g \circ \epsilon$. De este proceso hemos obtenido finalmente
las siguientes ecuaciones:

\[ \begin{aligned}
\eta \circ f &= \overline{Ff} \\
\epsilon \circ Ff &= \overline{f} \\
Gg \circ \eta &= \overline{g} \\
g \circ  \epsilon &= \overline{Gg} 
\end{aligned} \]

Con ellas podemos probar la naturalidad de $\eta$ y la naturalidad de
$\epsilon$:

\[ \begin{tikzcd}
GFX  \arrow{r}{GFf} & GFY \\
X \arrow{u}[swap]{\eta_X} \arrow{r}[swap]{f} & Y \arrow{u}{\eta_Y}
\end{tikzcd}
\]   \[ \begin{tikzcd}
FGX \arrow{d}[swap]{\epsilon_X} \arrow{r}{FGg} & FGY \arrow{d}{\epsilon_Y}\\
X \arrow{r}[swap]{g} & Y
\end{tikzcd}
\]

Ya que $\eta \circ f = \overline{Ff} = GFf \circ \eta$ y $f \circ \epsilon = \overline{Gf} = \epsilon \circ FGf$. Y además podemos probar
los dos triángulos de naturalidad.

\[ \begin{tikzcd}
F \arrow{r}{F \eta_X} \arrow{dr}{id} & FGF \arrow{d}{\epsilon_{FX}} \\
 & F
\end{tikzcd}   
\]     \[ \begin{tikzcd}
G \arrow{r}{\eta_{GX}} \arrow{dr}{id} & GFG \arrow{d}{G\epsilon_X} \\
 & G
\end{tikzcd}
\]

Teniendo finalmente que:

\[ \begin{aligned}
\epsilon \circ F\eta &= \overline{\eta} = 1 \\
G\epsilon \circ \eta &= \overline{\epsilon} = 1
\end{aligned} \]

El otro sentido de la demostración se tiene llegando primero a las
cuatro ecuaciones, y usándolas para definir el isomorfismo
$(-)$. Falta entonces demostrar su naturalidad.

\section*{Harpreet Bedi's channel}
\label{sec-3}
\subsection*{Sheaves and coho}
\label{sec-3-1}
\subsubsection*{Preseaves and sheaves}
\label{sec-3-1-1}
\begin{itemize}
\item Preseaf definition
\label{sec-3-1-1-1}
\begin{definition}
\textbf{Preseaf}. A preseaf ${\cal F}$ of abelian groups on a topological space $X$ consists of:

\begin{itemize}
\item For each open set $U$, an abelian group ${\cal F}(U)$, whose elements are called 
\textbf{sections}.
\item For each inclusion $V \subseteq U$, a \textbf{restriction map}, homomorphism of the form:
\end{itemize}


\[p_{U,V} : {\cal F}(U) \longrightarrow {\cal F}(V)\]

such that $p_{U,W} = p_{V,W} \circ p_{U,V}$.
\end{definition}

We can write the restriction of an element $u \in U$ to a set $V \subseteq U$ as
$u|_V = p_{U,V}(u)$.

\item Sheaf definition
\label{sec-3-1-1-2}
\begin{definition}
\textbf{Gluability axiom}. Given $U = \bigcup U_i$ with sections $s_i \in {\cal F}(U_i)$, if we have:

\[ s_\alpha|_{U_\alpha \cap U_\beta} = s_\beta|_{U_\alpha \cap U_\beta} \]

then there exists $s \in {\cal F}(U)$ such that $s|_U_\alpha = s_\alpha$.
\end{definition}
\begin{definition}
\textbf{Uniqueness axiom}. Given $U = \bigcup U_i$ with sections $s,t \in {\cal F}(U)$ such that:

\[\forall U_\alpha:\ s|_U_\alpha = t|_U_\alpha\]

then $s=t$.
\end{definition}
\begin{definition}
\textbf{Sheaves}. A presheaf satisfiying gluability and uniqueness.
\end{definition}
\end{itemize}
\subsection*{Homological Algebra}
\label{sec-3-2}
\subsubsection*{2. Chain Complex and Homology}
\label{sec-3-2-1}
\subsubsection*{4. Homology Theorem}
\label{sec-3-2-2}
\begin{itemize}
\item Setting
\label{sec-3-2-2-1}
Given a SES of chain complexes \$0 \longrightarrow \{\cal A\}
\longrightarrow{\cal B}
\longrightarrow{\cal C}
\longrightarrow 0\$, we have a long exact
sequence like:

\[ \begin{tikzcd}
 & \dots\rar & H_{n+1}({\cal C}) \arrow[out=355,in=175,swap]{dll}{\delta_{n+1}} \\
H_{n}({\cal A})\rar & H_{n}({\cal B}) \rar & H_{n}({\cal C}) \arrow[out=355,in=175,swap]{dll}{\delta_n}\\
H_{n-1}({\cal A})\rar & \dots & 
\end{tikzcd} \]

\item Naturality
\label{sec-3-2-2-2}
When we have two SES of chain complexes:

\[ \begin{tikzcd}
0 \rar & {\cal A}\rar\dar & {\cal B}\rar\dar & {\cal C}\rar\dar & 0 \\
0 \rar & {\cal A}'\rar & {\cal B}'\rar & {\cal C}'\rar & 0 \\
\end{tikzcd} \]

where it hols for every $n$ that:

\[ \begin{tikzcd}
H_n({\cal C}) \rar\dar & H_{n-1}({\cal A})\dar \\
H_n({\cal C}') \rar & H_{n-1}({\cal A}')
\end{tikzcd} \]
\end{itemize}

\subsubsection*{8. Proj, inj and flat modules}
\label{sec-3-2-3}
\begin{itemize}
\item Definitions
\label{sec-3-2-3-1}
An R-module $D$ is:

\begin{enumerate}
\item \textbf{Projective} if $Hom(D, -)$ is exact.
\item \textbf{Injective} if $Hom(-,D)$ is exact.
\item \textbf{Flat} if $D \otimes -$ is exact.
\end{enumerate}

\item Considerations
\label{sec-3-2-3-2}
We know that $Hom(D,-)$ and $Hom(-,D)$ are left-exact and that
$D\otimes -$ is right-exact; so for them to be exact, we only need:

\begin{itemize}
\item A module $D$ is \textbf{projective} when $B \longrightarrow C$ surjective induces
$Hom(D,B) \longrightarrow Hom(D,C)$ surjective.

\[ \begin{tikzcd}
               & B \dar[two heads] \\
   D \rar\urar[dashed]{\exists} & C
   \end{tikzcd} \]

\item A module $D$ is \textbf{injective} when $A \longrightarrow B$ surjective induces
$Hom(B,D) \longrightarrow Hom(A,D)$ surjective.

\[ \begin{tikzcd}
     & A \dar[two heads]\dlar \\
   D & B \lar[dashed]{\exists}
   \end{tikzcd} \]

\item A module $D$ is \textbf{flat} when $A \longrightarrow B$ injective induces 
$D\otimes A \longrightarrow D \otimes B$ injective.
\end{itemize}
\end{itemize}

\subsubsection*{9. Resolutions: projective, injective and flat}
\label{sec-3-2-4}
\begin{itemize}
\item Definitions
\label{sec-3-2-4-1}
\begin{itemize}
\item Resolutions
\label{sec-3-2-4-1-1}
Resolutions are \textbf{exact sequences}.

\item Projective resolution
\label{sec-3-2-4-1-2}
A resolution, with $d_i$ maps:

\[\dots\longrightarrow P_2\longrightarrow P_1\longrightarrow P_0
\longrightarrow M \longrightarrow 0\]

where $P_i$ is projective.

\item Injective resolution
\label{sec-3-2-4-1-3}
A resolution:

\[0 \longrightarrow M \longrightarrow E_0\longrightarrow E_1
\longrightarrow E_2 \longrightarrow \dots\]

where $E_i$ is injective.

\item Flat resolution
\label{sec-3-2-4-1-4}
A resolution:

\[\dots\longrightarrow F_2\longrightarrow F_1\longrightarrow F_0
\longrightarrow M \longrightarrow 0\]

where $F_i$ is flat.
\end{itemize}

\item How to form a resolution
\label{sec-3-2-4-2}
It is important to notice that, given a module $M$, we can always find
a surjection from a proyective module (if we have \emph{enough
projectives}). So we can construct a projective resolution as follows:

\[ \begin{tikzcd}[column sep=tiny]
&\ker f_2 \drar&&&&\ker \pi\drar &&& \\
\dots&&P_2 \drar[two heads]{f_2}&&P_1 \urar[two heads]{f_1} && P_0 \ar[two heads,rr]{\pi} && M \rar & 0\\
&&&\ker f_1 \urar&&&&
\end{tikzcd} \]

We can also reverse the arrows to obtain an injective resolution.
\end{itemize}

\subsubsection*{{\bfseries\sffamily TODO} 10. Homotopic projective resolutions}
\label{sec-3-2-5}
\begin{itemize}
\item Extending a morphism
\label{sec-3-2-5-1}
Given two projective resolutions of two $R$ modules, $A$ and $A'$, and a morphism
between them, $f$. We can extend it to $f_n \in Hom(P_n,P_n')$.

\[ \begin{tikzcd}
\dots\rar & P_{n+1}\rar & P_n\rar& \dots
 \rar & P_1\rar{d_1} & P_0\rar{d_0}& A \dar{f} \rar& 0 \\
\dots\rar & P_{n+1}'\rar & P_n'\rar&\dots
 \rar & P_1'\rar{d_1¡} & P_0'\rar{d_0'}& A' \rar& 0 \\
\end{tikzcd} \]

\begin{itemize}
\item Extending the morphism, base case
\label{sec-3-2-5-1-1}
We use that $P_0$ is projective to construct:

\[ \begin{tikzcd}
     & P_0 \arrow[ddl,"f_0",dashed,swap] \dar\\
     & A \dar{f} \\
P_0' \rar[two heads] & A'
\end{tikzcd} \]

\item Extending the morphism, inductive case
\label{sec-3-2-5-1-2}
We are going to show that $f_n(\im d_{n+1}) \subset \im d_{n+1}' = \ker d_n'$. That is, 
$d_n' \circ f_n \circ d_{n+1} = 0$. And that follows from diagram chasing. We use
again the projectivity of $P_{n+1}$.

\[ \begin{tikzcd}
     & P_{n_+1} \arrow[ddl,"f_{n+1}",dashed,swap] \dar\\
     & \im d_{n+1} \dar{f_n} \\
P_{n+1}' \rar[two heads] & \im d_{n+1}'
\end{tikzcd} \]
\end{itemize}

\item {\bfseries\sffamily TODO} Homotopic resolutions
\label{sec-3-2-5-2}
\end{itemize}
\subsubsection*{11. Derived functors Ext and Tor}
\label{sec-3-2-6}
\begin{itemize}
\item Right derived functors
\label{sec-3-2-6-1}
Let $F$ be additive, covariant and left-exact. Let 
$0 \longrightarrow M \longrightarrow E^\bullet$ be an injective resolution with $M$ deleted; then $F(E^\bullet)$ 
is a complex, and we define:

\[R^i F(M) = H^i(F(E^\bullet)) = 
\frac{\ker \{F(E_i) \longrightarrow F(E_{i+1})\}}
{\im\{ F(E_{i-1}) \longrightarrow F(E_i)\}}\]

That is, if we take the injective resolution:

\[ 0 \longrightarrow M \longrightarrow E_0 \longrightarrow E_1 
\longrightarrow \dots\]

Delete $M$ and apply $F$ to get a (non neccesarily exact) complex where 
we can compute the homology:

\[ 0 \longrightarrow F(E_0) \longrightarrow F(E_1)
\longrightarrow F(E_2) \longrightarrow \dots\]

\item Left derived functors
\label{sec-3-2-6-2}
Let $F$ be additive, contravariant and left-exact. Let 
$P^\bullet \longrightarrow M \longrightarrow 0$ be a projective resolution with $M$ deleted; 
then $F(P^\bullet)$ is a complex, and we define:

\[R^i F(M) = H^i(F(P^\bullet)) = 
\frac{\ker \{F(P_i) \longrightarrow F(P_{i+1})\}}
{\im\{ F(P_{i-1}) \longrightarrow F(P_i)\}}\]

That is, if we take the injective resolution:

\[\dots \longrightarrow P_2\longrightarrow P_1\longrightarrow P_0
\longrightarrow M \longrightarrow 0\]

Delete $M$ and apply $F$ to get a (non neccesarily exact) complex 
where we can compute the homology:

\[ 0 \longrightarrow F(P_0) \longrightarrow F(P_1)
\longrightarrow F(P_2) \longrightarrow \dots\]
\end{itemize}

\subsubsection*{12. Computations of some standard Ext and Tor examples}
\label{sec-3-2-7}
\subsubsection*{13. Long Exact Sequence for Tor}
\label{sec-3-2-8}
\subsection*{Algebraic Geometry}
\label{sec-3-3}
\subsubsection*{1. Intro to Algebraic Geometry}
\label{sec-3-3-1}
\section*{Algebra: chapter 0}
\label{sec-4}
\subsection*{III. Anillos y módulos}
\label{sec-4-1}
\subsubsection*{7. Complejos y homología}
\label{sec-4-1-1}
\begin{itemize}
\item 7.1. Complejos y secuencias exactas.
\label{sec-4-1-1-1}
\begin{definition}
\textbf{Complejo}. Un complejo es una serie de morfismos $d_i$ entre R-Módulos:

\[\dots \longrightarrow M_{i+1} \longrightarrow M_i \longrightarrow M_{i-1} \longrightarrow \dots\]

tales que $d_i \circ d_{i+1} = 0$.
\end{definition}

Además lo llamamos \textbf{exacto} cuando $im (d_{i+1}) = ker (d_i)$.

\begin{proposition}
\textbf{Exactitud de monomorfismos y epimorfismos}. Dos complejos de la forma:

\[ \dots \longrightarrow 0 \longrightarrow L \overset{\alpha}\longrightarrow M \longrightarrow \dots \]
\[ \dots \longrightarrow M \overset{\beta} \longrightarrow N \longrightarrow 0 \longrightarrow \dots \]

Son exactos en $L$ y $N$ ssi $\alpha$ y $\beta$ son monomorfismo y epimorfismo, 
respectivamente.
\end{proposition}

\begin{definition}
\textbf{Secuencia exacta corta}. Una secuencia exacta corta es un complejo de la forma:

\[ 0 \longrightarrow L \overset{\alpha}\longrightarrow M \overset{\beta}\longrightarrow N \longrightarrow 0 \]
\end{definition}

El primer teorema de isomorfía nos dice que $N \cong \frac{M}{ker(\beta)} = \frac{M}{im(\alpha)}$ lo que nos 
lleva a identificar   $N \cong \frac{M}{L}$. De hecho, cada monomorfismo da lugar a una 
secuencia exacta corta:

\[ 0 \longrightarrow \ker(\phi) \longrightarrow M \longrightarrow im(\phi) \longrightarrow 0 \]

\item 7.2. Secuencias exactas escindidas
\label{sec-4-1-1-2}
\begin{definition}
\textbf{Secuencia escindida}. Una secuencia exacta corta:

\[ 0 \longrightarrow M_1 \longrightarrow N \longrightarrow M_2 \longrightarrow 0 \]

es escindida si es isomorfa a una secuencia de la forma siguiente:

\[ \begin{tikzcd}
 0   \arrow{r}{} & 
 M_1 \arrow{d}{\sim}\arrow{r}{} & 
 N   \arrow{d}{\sim}\arrow{r}{} & 
 M_2 \arrow{d}{\sim}\arrow{r}{} & 
 0 \\
 0   \arrow{r}{} & 
 M_1 \arrow{r}{} & 
 M_1 \oplus M_2   \arrow{r}{} & 
 M_2 \arrow{r}{} & 
 0
 \end{tikzcd} \]

Es decir, hay un isomorfismo entre secuencias.
\end{definition}

\begin{theorem}
\textbf{Relación entre secuencias escindidas e inversas}. Sea $\phi$ un homomorfismo;
entonces tiene inversa izquierda ssi la secuencia siguiente escinde:

\[ 0 \longrightarrow M \overset{\phi}\longrightarrow N \longrightarrow coker(\phi) \longrightarrow 0 \]

Y tiene inversa derecha si la secuencia siguiente escinde:

\[ 0 \longrightarrow ker(\phi) \longrightarrow M \overset{\phi}\longrightarrow N \longrightarrow 0 \]
\end{theorem}

\item 7.3. Homología, y el lema de la serpiente
\label{sec-4-1-1-3}
\begin{definition}
\textbf{Homología}. La i-ésima homología de un complejo,

\[ \dots \longrightarrow M_{i+1} \overset{d_{i+1}}\longrightarrow M_i \overset{d_i}\longrightarrow M_{i-1} \longrightarrow \dots \]

es el R-módulo:

\[H_i(M) = \frac{ker(d_i)}{im(d_{i+1})}\]
\end{definition}

La homología mide lo que se aleja de ser exacto en un punto determinado, y
es $0$ cuando el complejo es exacto. Puede verse como una generalización de
kernel y cokernel; que los realiza en este caso extremo:

\[ 0 \longrightarrow M_1 \overset{\phi}\longrightarrow M_0 \longrightarrow 0 \]

En el que $H_1(M) \cong ker(\phi)$ y $H_0(M) \cong coker(\phi)$.

\begin{theorem}
\textbf{Lema de la serpiente}. Teniendo dos secuencias exactas en el diagrama 
conmutativo siguiente:

\[ \begin{tikzcd}
 0 \rar & L_1 \rar{\alpha_1}\arrow{d}{\lambda} & M_1 \rar{\beta_1}\arrow{d}{\mu} & N_1 \rar\arrow{d}{\eta} & 0 \\
 0 \rar & L_0 \rar{\alpha_0}                   & M_0 \rar{\beta_0}               & N_0 \rar                & 0
 \end{tikzcd} \]

Existe una secuencia exacta de la forma:

\[ 0 \overset{}\longrightarrow 
 ker(\lambda) \overset{}\longrightarrow 
 ker(\mu) \overset{}\longrightarrow 
 ker(\eta) \overset{\delta}\longrightarrow 
 coker(\lambda) \overset{}\longrightarrow 
 coker(\mu) \overset{}\longrightarrow 
 coker(\eta) \overset{}\longrightarrow 
 0\]
\end{theorem}

El diagrama desde el que se deduce todo esto, con columnas exactas, es
el siguiente:

\[ \begin{tikzcd}
	& 0 \dar              & 0 \dar            & 0 \dar           &   \\
 0 \rar & ker(\lambda) \rar \dar  & ker(\mu) \rar \dar    & ker(\eta) \dar \ar[out=355, in=175,looseness=1, overlay, swap]{dddll}{\delta}       &   \\
 0 \rar & L_1 \rar{\alpha_1} \dar{\lambda}  & M_1 \rar{\beta_1} \dar{\mu} & N_1 \rar \dar{\eta}        & 0 \\
 0 \rar & L_0 \rar{\alpha_0} \dar & M_0 \rar{\beta_0} \dar & N_0 \rar \dar        & 0 \\
	& coker(\lambda) \rar \dar & coker(\mu) \rar \dar  & coker(\eta) \rar \dar & 0 \\
	& 0                   & 0                 & 0                &
 \end{tikzcd} \]
\end{itemize}

\subsection*{IV. Álgebra lineal}
\label{sec-4-2}
\subsubsection*{4. Presentaciones y resoluciones}
\label{sec-4-2-1}
\begin{itemize}
\item 4.1. Torsión
\label{sec-4-2-1-1}
\begin{definition}
\textbf{Torsión}. Un elemento $m \in M$ módulo de $R$ es de \textbf{torsión} si $\{m\}$ es linealmente
dependiente. Es decir,

\[ \exists r \in R,\ r \neq 0\ :\ rm = 0 \]

El conjunto de elementos de torsión se llama $Tor(M)$. Un módulo es \textbf{libre de torsión}
si $Tor(M) = 0$ y \textbf{de torsión} si $Tor(M)=M$.
\end{definition}

Un anillo conmutativo es libre de torsión sobre sí mismo si y sólo si es dominio de
integridad. Cuando esto ocurre, $Tor(M)$ es siempre submódulo de $M$. Submódulos o
sumas de módulos libres de tensión serán libres de torsión, y por todo esto, los módulos
libres sobre dominios de integridad serán libres de torsión.

\begin{definition}
\textbf{Cíclico}. Un módulo es \textbf{cíclico} cuando es generado por un elemento. Es decir,
cuando $M \cong R/I$ para algún ideal.
\end{definition}

Cuando en un dominio de integridad todos sus
módulos cíclicos son libres de torsión, es un cuerpo. Otra forma de pensar sobre un módulo
cíclico es como aquel que admite un epimorfismo:

\[ R \longrightarrow M \longrightarrow 0 \]

\item 4.2. Módulos finitamente presentados y resoluciones libres
\label{sec-4-2-1-2}
\begin{definition}
\textbf{Anulador.} El anulador de un módulo $M$ es:

\[Ann_R(M) = \{ r \in R\ |\ \forall m \in M, rm = 0 \}\]
\end{definition}

Es un ideal de $R$. Cuando $M$ es finitamente generado y $R$ es dominio de integridad,
$M$ es de torsión si y sólo si $Ann(M) \neq 0$.

\begin{definition}
\textbf{Módulos finitamente generados y presentados}. Sabemos que todos los módulos admiten un
epimorfismo de la forma:

\[ R^{\oplus A} \longrightarrow M \longrightarrow 0\]

Cuando lo admiten con $A$ finito, se tiene $M$ \textbf{finitamente generado}. Un módulo se dice
\textbf{finitamente presentado} si hay una secuencia exacta de la forma:

\[R^n \overset{\phi}\longrightarrow R^m \longrightarrow M \longrightarrow 0\]

.
\end{definition}

Si $R$ es Noetheriano, todo módulo finitamente generado es finitamente presentado.

\begin{definition}
\textbf{Resolución}. Una resolución de $M$ mediante módulos libres finitamente generados es
un complejo exacto:

\[ \dots \rightarrow R^{m_3} \rightarrow R^{m_2} \rightarrow R^{m_1} \rightarrow R^{m_0} \rightarrow M \rightarrow 0 \]
\end{definition}

Aquí podemos entender que $R^{m_0}$ contiene los generadores, $R^{m_1}$ las relaciones
entre los generadores, $R^{m_2}$ las relaciones entre relaciones, y así sucesivamente.

Un dominio de integridad es \textbf{cuerpo si y sólo si todos sus módulos son finitamente generados},
esto es equivalente a tener:

\[ 0 \longrightarrow R^m \longrightarrow M \longrightarrow 0 \]

para cualquier módulo.

Un dominio de integridad es \textbf{PID si todas las resoluciones como finitamente generado 
extienden a finitamente presentado}, de la forma:

\[0 \longrightarrow R^{m_1} \longrightarrow R^{m_0} \overset{\pi}\longrightarrow M \longrightarrow 0\]

esto equivale a pedir que $\ker(\pi)$ sea libre.

\item 4.3. Leyendo una presentación
\label{sec-4-2-1-3}
Hemos visto que podemos estudiar un módulo finitamente presentado por un
morfismo $\phi: R^n \longrightarrow R^m$, donde $M = coker(\phi)$. Esto quiere decir que 
podemos asignarle una matriz explícita.

\begin{theorem}
\textbf{Producto de módulos en matrices}. Sean $M,N$ módulos con matrices $A,B$.
Tenemos $M \oplus N$ con matriz:

\[\left(\begin{array}{c|c}
 A & 0 \\ \hline 0 & B 
 \end{array}\right)\]
\end{theorem}

Además nótese que las \textbf{matrices equivalentes} representan el mismo 
homeomorfismo, y por tanto el mismo módulo.

\begin{theorem}
\textbf{Transformaciones de matrices de módulos}. Una matriz representa el mismo módulo
tras las transformaciones de:
\begin{itemize}
\item Permutar filas o columnas
\item Añadir filas o columnas linealmente dependientes
\item Multiplicar filas o columnas por una unidad
\item Quitar una fila y columna en la que sólo queda una unidad
\end{itemize}
\end{theorem}

Las primeras son consecuencia de la equivalencia. La última puede colocarse como
una parte de identidad en una matriz de la forma:

\[A = \left(\begin{array}{c|c}
 u & 0 \\ \hline 0 & A' 
 \end{array}\right)\]

Que no afecta al cokernel.
\end{itemize}

\subsection*{VII. Cuerpos}
\label{sec-4-3}
\subsubsection*{1. Extensiones de cuerpos I}
\label{sec-4-3-1}
\begin{itemize}
\item 1.1. Definiciones básicas
\label{sec-4-3-1-1}
\begin{itemize}
\item Categoría de los cuerpos
\label{sec-4-3-1-1-1}
Los cuerpos forman la \textbf{categoría $\mathtt{Fld}$} con los homomorfismos de 
anillos entre ellos. Todo homomorfismo de anillos entre cuerpos
es inyectivo y todo morfismo en esta categoría es monomorfismo.

Así, todo morfismo entre cuerpos en $Hom(k,K)$ es una extensión $K/k$.

\item Característica de un cuerpo
\label{sec-4-3-1-1-2}
La \textbf{característica} de $K$ es el generador de $ker(i)$ para 
$i : \mathbb{Z} \longrightarrow K$. Las extensiones preservan la 
característica, así que podemos particionar la categoría en categorías 
$\mathtt{Fld}_p$.

\item Cuerpos primos
\label{sec-4-3-1-1-3}
El inicial de $\mathtt{Fld}_0$ es $\mathbb{Q}$, y el de $\mathtt{Fld}_p$ es $\mathbb{F}_p = \mathbb{Z}/p\mathbb{Z}$. Todos los
cuerpos son extensiones de uno de estos llamados \textbf{cuerpos primos}.

\item Grado de una extensión
\label{sec-4-3-1-1-4}
El \textbf{grado}, $[F : K]$, de una extensión es su dimensión como espacio
vectorial sobre la base. Es \textbf{finita} o \textbf{infinita} si lo es su grado.
\end{itemize}

\item 1.2. Extensiones simples
\label{sec-4-3-1-2}
\begin{itemize}
\item Extensión simple
\label{sec-4-3-1-2-1}
Una extensión es \textbf{simple} si es de la forma $K(\alpha)$ donde 
$K(\alpha)$ es la intersección de todos los subcuerpos de algún
$F$ conteniendo al cuerpo $K$ y el elemento $\alpha$.

\item Polinomio irreducible mínimo
\label{sec-4-3-1-2-2}
Dada una extensión simple $K(\alpha)$, consideramos la evaluación
$\epsilon : K[X] \longrightarrow K(\alpha)$ por casos:

\begin{itemize}
\item Es \textbf{inyectiva} ssi es una \textbf{extensión infinita}. En este
caso $K(\alpha) \cong K(X)$ es el cuerpo de funciones racionales.
\item No es \textbf{inyectiva}. Existe un único polinomio mónico
irreducible $p$ que genera el núcleo,

\[ K(\alpha) \cong \frac{K[t]}{(p(t))}\]

Se le llama \textbf{polinomio mínimo}.
\end{itemize}

\item {\bfseries\sffamily TODO} Extensión de isomorfismos a extensiones simples
\label{sec-4-3-1-2-3}
Proposition 1.5
\item Automorfismos de una extensión
\label{sec-4-3-1-2-4}
El \textbf{grupo de automorfismos} de una extensión $Aut_K(F)$, es el
grupo de los automorfismos de cuerpos que dejan fijo $K$.
\item Automorfismos y raíces
\label{sec-4-3-1-2-5}
Sea $K(\alpha)$ con $p$ polinomio mínimo. Entonces $p$ tiene $|Aut_K(K(\alpha))|$ raíces
distintas en $K(\alpha)$. En particular,

\[ |Aut_K(K(\alpha))| \leq [K(\alpha):K] \]

y el caso de igualdad se tiene con $p$ factorizando en factores 
lineales sobre $F$.
\end{itemize}
\item 1.3. Extensiones finitas y algebraicas
\label{sec-4-3-1-3}
\begin{itemize}
\item Elementos algebraicos y trascendentes
\label{sec-4-3-1-3-1}
Sea $F/K$ una extensión con $\alpha \in F$, entonces $\alpha$ es \textbf{algebraico}
cuando $K(\alpha)/K$ es finita, y \textbf{trascendente} si no. Una extensión
es \textbf{algebraica} si todos sus elementos lo son.
\end{itemize}
\end{itemize}

\subsubsection*{6. Un poco de teoría de Galois}
\label{sec-4-3-2}
\begin{itemize}
\item 6.1. Correspondencia de Galois y extensiones de Galois
\label{sec-4-3-2-1}
\begin{itemize}
\item Cuerpo fijo
\label{sec-4-3-2-1-1}
Sea $F/k$ extensión y $G \subseteq Aut_k(F)$. Llamamos \textbf{cuerpo fijo} de $G$ a:

\[ F^G = \{ \alpha\in F \mid \forall g \in G, g\alpha=\alpha\}\]

\item Correspondencia de Galois
\label{sec-4-3-2-1-2}
Hay correspondencia entre los cuerpos intermedios de la extensión
y los subgrupos del grupo de automorfismos.

Dado $E$ cuerpo intermedio, lo enviamos a $Aut_E(F)$. Dado $G$ lo enviamos
a $F^G$.

\item Inclusión y correspondencia
\label{sec-4-3-2-1-3}
Para cualesquiera subgrupo $G$ y cuerpo intermedio $E$:

\begin{itemize}
\item $E \subseteq F^{Aut_E(F)}$
\item $G \subseteq Aut_{F^G}(F)$
\end{itemize}

Si llamamos $E_1E_2$ al menor subcuerpo de $F$ conteniendo $E_1,E_2$ y llamamos
$<G_1,G_2>$ al menor subgrupo de los automorfismos conteniendo $G_1,G_2$:

\begin{itemize}
\item $Aut_{E_1E_2}(F) = Aut_{E_1}(F) \cap Aut_{E_2}(F)$
\item $F^{<G_1,G_2>} = F^{G_1} \cap F^{G_2}$
\end{itemize}

\item Extensiones de Galois
\label{sec-4-3-2-1-4}
Sea $F/k$ extensión, equivalen:

\begin{itemize}
\item $F$ es cuerpo de descomposición de algún $f \in k[t]$.
\item $F/k$ es normal y separable.
\item $|Aut_k(F)| = [F : k]$.
\item La correspondencia de Galois es biyección.
\item $F/k$ separable y, si $E/F$ es algebraica con $\sigma \in Aut_k(E)$, $\sigma(F)=F$.
\end{itemize}

Llamamos a esto una \textbf{extensión de Galois}.
\end{itemize}
\end{itemize}
\subsection*{VIII. Vuelta al álgebra lineal}
\label{sec-4-4}
\subsubsection*{1. Preliminares}
\label{sec-4-4-1}
\begin{itemize}
\item 1.1. Funtores
\label{sec-4-4-1-1}
\begin{definition}
\textbf{Funtor}. Un funtor covariante:

\[{\cal F} : C \longrightarrow D\]

Asigna a cada $A \in C$ un ${\cal F}(A) \in D$ y mapea los morfismos entre cada par de objetos:

\[Hom_C(A,B) \rightarrow Hom_D({\cal F}(A),{\cal F}(B))\]

Respetando la identidad y la composición de morfismos. 

Un \textbf{funtor contravariante} es un funtor desde la categoría opuesta:

\[{\cal F} : C^{op} \longrightarrow D\]
\end{definition}

Los funtores preservan los diagramas conmutativos. Llamamos \textbf{prehaz} a un funtor
contravariante $C \longrightarrow \mathtt{Set}$.

\begin{definition}
\textbf{Funtor aditivo}. Llamamos a un funtor 
${\cal F}: R-\mathtt{Mod} \longrightarrow S-\mathtt{Mod}$ \textbf{aditivo} cuando
la función $Hom_{R}(A,B) \rightarrow Hom_{S}({\cal F}(A),{\cal F}(B))$ es homomorfismo de grupos.
\end{definition}

\item 1.3. Equivalencia de categorías
\label{sec-4-4-1-2}
\begin{definition}
\textbf{Funtores plenamente fieles}. Dada la función inducida:
\[Hom_C(A,B) \rightarrow Hom_D({\cal F}(A),{\cal F}(B))\]
Un funtor es \textbf{fiel} si es inyectiva, \textbf{pleno} si es sobreyectiva y \textbf{plenamente fiel}
si es biyectiva.
\end{definition}

\begin{definition}
\textbf{Equivalencia de categorías}. Un funtor es una equivalencia de categorías si 
es plenamente fiel y esencialmente sobreyectivo, es decir, para cada $Y \in D$,
existe un $X \in C$ tal que $F(X) \cong Y$.
\end{definition}

\item 1.4. Límites y colímites
\label{sec-4-4-1-3}

\begin{definition}
\textbf{Límite}. Para un funtor ${\cal F}: {\cal I} \longrightarrow C$, su límite es
un objeto $L \in C$ con morfismos $\lambda_I: L \longrightarrow {\cal F}(I)$ tales que

\begin{itemize}
\item Conmuta el siguiente diagrama para cualquier $\alpha : I \longrightarrow J$:
\end{itemize}

\[ \begin{tikzcd}[column sep=1.5em]
  & L \arrow{dr}{\lambda_J} \arrow{dl}[swap]{\lambda_I} \\
 {\cal F}(I) \arrow{rr}{{\cal F}(\alpha)} && {\cal F}(J)
 \end{tikzcd} \]

\begin{itemize}
\item $L$ es final en este diagrama.
\end{itemize}
\end{definition}

Será esencialmente único y puede notarse por $\varprojlim {\cal F}$.

\begin{theorem}
\textbf{Límites sobre cadenas en R-Mod}. En R-Mod siempre existe un límite llamado \(\varprojlim {\cal A}_i\) sobre una
cadena de la forma:

\[ \begin{tikzcd}
 & & A 
 \arrow{lld}[swap]{\phi_5}
 \arrow{ld}{\phi_4}
 \arrow{d}{\phi_3}
 \arrow{rd}[swap]{\phi_2}
 \arrow{rrd}{\phi_1} 
 & & \\
 \dots \arrow{r}[swap]{\phi_{45}}  &
 A_4 \arrow{r}[swap]{\phi_{34}} &
 A_3 \arrow{r}[swap]{\phi_{23}} &
 A_2 \arrow{r}[swap]{\phi_{12}} &
 A_1
 \end{tikzcd} \]
\end{theorem}

Este límite es el submódulo de las \emph{secuencias coherentes} en $\prod_i A_i$, es decir, de
aquellas tales que $a_i = \phi_{i,i+1}(a_{i+1})$; teniendo como morfismos $\phi_i$ las proyecciones
canónicas


\begin{definition}
\textbf{Colímite}. La noción dual de límite es el \textbf{colímite}, es decir, para
un funtor ${\cal F} : I \longrightarrow C$, su colímite es un objeto $L \in C$ con morfismos $\gamma_i : {\cal F}(I) \longrightarrow L$
tales que

\begin{itemize}
\item Conmuta el siguiente diagrama para cualquier $\alpha : I \longrightarrow J$:
\end{itemize}

\[ \begin{tikzcd}[column sep=1.5em]
  & L  \\
 {\cal F}(I) \arrow{ur}{\gamma_I} \arrow{rr}{{\cal F}(\alpha)} && {\cal F}(J) \arrow{ul}[swap]{\gamma_J}
 \end{tikzcd} \]

\begin{itemize}
\item $L$ es inicial en este diagrama.
\end{itemize}
\end{definition}

\item 1.5. Comparando funtores
\label{sec-4-4-1-4}
\begin{definition}
\textbf{Transformación natural}. Una transformación natural entre dos funtores ${\cal F} \Longrightarrow {\cal G}$ 
consiste en morfismos $\upsilon_X : {\cal F}(X) \longrightarrow {\cal G}(X)$ tales que conmuta el diagrama:

\[ \begin{tikzcd}
 {\cal F}(X) \arrow{r}{{\cal F}(\alpha)} \arrow{d}{\upsilon_X} & {\cal F}(Y) \arrow{d}{\upsilon_Y} \\
 {\cal G}(X) \arrow{r}{{\cal G}(\alpha)} & {\cal G}(Y)
 \end{tikzcd}
 \]

para cualquier morfismo $\alpha$.

Llamamos \textbf{isomorfismo natural} a una transformación natural donde cada $\upsilon$
es un isomorfismo.
\end{definition}

\begin{definition}
\textbf{Funtor adjunto}. Llamamos ${F}$ y ${G}$ adjuntos si tenemos:

\[ Hom_C(X,GY) \cong Hom_D(FX,Y) \]

Isomorfismos naturales.
\end{definition}

Lo que nos da realmente un isormorfismo natural de $Hom_C(F-,-)$ con $Hom_D(-,G-)$,
entendidos como funtores. Llamamos aquí adjunto izquierdo a $F$ y adjunto derecho a $G$.
Tenemos más sobre funtores adjuntos en la lista de reproducción de \href{https://www.youtube.com/playlist?list=PL54B49729E5102248}{The Catsters}.

\begin{theorem}
\textbf{Continuidad de adjuntos}. Los funtores adjuntos derechos son continuos, los adjuntos
izquierdos son cocontinuos. Es decir, para $I : {\cal I}\longrightarrow D$, $J : {\cal J}\longrightarrow C$

\[G(\varprojlim I) = \varprojlim (G \circ I)\]
\[F(\varinjlim J) = \varinjlim (F \circ J)\]
\end{theorem}

Siempre que existan los límites. La demostración de esto se puede hacer aplicando los
funtores en los diagramas conmutativos y usando las propiedades universales de los límites.

\begin{definition}
\textbf{Funtor exacto}. Un funtor exacto respeta la exactitud de las secuencias. Es decir,
siendo la siguiente secuencia exacta:

\[ 0 \longrightarrow A \overset{\phi}\longrightarrow B \overset{\psi}\longrightarrow C \longrightarrow 0\]

La siguiente secuencia será exacta:

\[ 0 \longrightarrow FA \overset{F\phi}\longrightarrow FB \overset{F\psi}\longrightarrow FC \longrightarrow 0\]
\end{definition}

En particular, lo llamamos \emph{exacto a la izquierda} si preserva la exactitud de:

\[ 0 \longrightarrow A \overset{\phi}\longrightarrow B \overset{\psi}\longrightarrow C\]

Y \emph{exacto a la derecha} si preserva la exactitud de:

\[ A \overset{\phi}\longrightarrow B \overset{\psi}\longrightarrow C \longrightarrow 0\]
\end{itemize}

\subsubsection*{2. Producto tensor y el funtor Tor}
\label{sec-4-4-2}
\begin{itemize}
\item 2.1. Aplicaciones bilineales
\label{sec-4-4-2-1}
\begin{definition}
\textbf{Aplicación bilineal}. Una aplicación $\phi:M\times N \longrightarrow P$ es bilineal si
son lineales $\phi(\_,n)$ y $\phi(m,\_)$ para cualesquiera $m,n$.
\end{definition}

\begin{definition}
\textbf{Producto tensor}. $M \otimes_R N$ es el producto tensor de $M$ y $N$ como módulos de $R$
si cualquier aplicación bilineal factoriza de forma única a través de él:

\[ \begin{tikzcd}
 M \times N \arrow{r}{\phi} \arrow{d}{\otimes} & P \\
 M \otimes N \arrow{ru}[swap]{\exists! \overline\phi} &
 \end{tikzcd} \]
\end{definition}

Usando universalidad podemos ver que $R \otimes N \cong N$ y que $M\otimes N \cong N\otimes M$. La construcción
explícita del producto tensor se hace sobre el módulo libre sobre $M \times N$ provocando un
cociente sobre los submódulos generados por:

\[(m,r_1n_1+r_2n_2) - r_1(m,n_1) - r_2(m,n_2)\]
\[(r_1m_1+r_2m_2,n) - r_1(m_1,n) - r_2(m_2,n)\]

Lo que nos permite actuar con ellos de forma bilineal. La demostración se basa en usar
la propiedad universal de la proyección sobre ese cociente.

\item 2.2. Adjunción con Hom
\label{sec-4-4-2-2}
Dado un módulo $N$ de $R$, tenemos un funtor covariante $\otimes_R N$, que será \textbf{adjunto izquierdo}
a $Hom_{R-mod}(N,-)$. Podemos observar simplemente que una aplicación bilineal, al currificarse,
determina una función que va de $M$ a $Hom(N,P)$, y que es lineal. Sabiendo esto, es trivial
que:

\[ Hom_R(M, Hom_R(N,P)) \cong Hom_R(M \otimes N, P)\]

La naturalidad y el hecho de que es un isomorfismo se comprueban fácilmente. El hecho de
que exista una adjunción nos dice además que $\otimes_R N$, o $N\otimes_R$ por la isomorfía anterior,
son cocontinuos.

\begin{fact}
Para cualesquiera \(R\)-módulos, se tiene:

\[(M_1 \oplus M_2) \otimes N \cong (M_1 \otimes N) \oplus (M_2 \otimes N)\]

\[N \otimes (M_1 \oplus M_2) \cong (N \otimes M_1) \oplus (N \otimes M_2)\]

\[(\oplus_\alpha M_\alpha) \otimes N \cong \oplus_\alpha (M_\alpha \otimes N)\]
\end{fact}

Por cocontinuidad.

\begin{fact}
Para cualesquiera dos conjuntos $A,B$, se tiene:

\[R^{\oplus A} \otimes R^{\oplus B} \cong R^{\oplus A \times B}\]
\end{fact}

Teniendo \(R^{\oplus n} \otimes R^{\oplus m} \cong R^{\oplus nm}\). De hecho, la base del espacio producto
tensor la forman los vectores puros que emparejan elementos de las 
bases de cada uno de los espacios.

\begin{theorem}
\textbf{Producto tensor de cocientes}. Dado un $N$ módulo de $R$, e $I$ ideal,
tenemos:

\[\frac{R}{I}\otimes N \cong \frac{N}{IN}\]

Y desde ahí, aplicando además el tercer teorema de isomorfía, tenemos:

\[\frac{R}{I} \otimes \frac{R}{J} \cong \frac{R}{I+J}\]
\end{theorem}

Esto se deduce de aplicar el funtor $\_ \otimes N$ a la secuencia exacta del 
ideal:

\[I \longrightarrow R \longrightarrow \frac{R}{I} \longrightarrow 0\]

\[I \otimes N \longrightarrow N \longrightarrow \frac{R}{I} \otimes N \longrightarrow 0\]

Desde donde se obtiene $IN$ como inclusión de $I\otimes N$ en $N$.

\item 2.3. Exactitud y planitud
\label{sec-4-4-2-3}
\begin{definition}
\textbf{Módulo plano}. El módulo $N$ es \textbf{plano} si el funtor $\_ \otimes N$ es un
funtor exacto.
\end{definition}

Un \textbf{módulo libre} será siempre plano.

\item 2.4. Los funtores Tor
\label{sec-4-4-2-4}
\begin{definition}
\textbf{El funtor Tor}. Lo que se aleja de la exactitud el funtor $\_ \otimes N$
es medido por el funtor $Tor_1(\_,N)$. De hecho, si tenemos una secuencia
exacta:

\[0\longrightarrow A \longrightarrow B \longrightarrow C \longrightarrow 0\]

Obtenemos aplicando el funtor $\otimes N$ esta otra secuencia:

\[Tor_1(C,N) \longrightarrow A \otimes N \longrightarrow B \otimes N \longrightarrow C \otimes N \longrightarrow 0\]

Y de hecho, esta secuencia podrá extenderse aún más con \emph{funtores derivados},
que se definen como:

\[Tor_i^R(M,N) = H_i(M_{\bullet} \otimes N)\]
\end{definition}

Aquí entendemos $M_\bullet \otimes N$ como el complejo que se obtiene tomando una resolución
libre de $M$:

\[\dots \longrightarrow R^{\otimes S_2} \longrightarrow R^{\otimes S_1} 
 \longrightarrow R^{\otimes S_0} \longrightarrow M \longrightarrow 0}\]

Y retirando $M$ y tensando sobre $N$, para tener:

\[\dots \longrightarrow N^{\otimes S_2} \longrightarrow N^{\otimes S_1} 
 \longrightarrow N^{\otimes S_0} \longrightarrow 0}\]

Todo esto se obtendrá de manera natural aplicando el lema de la serpiente a una secuencia
de resoluciones compatibles, algo que, si los módulos fueran PID y tuvieran una resolución
de grado 2, sería de la forma:

\[ \begin{tikzcd}
    & 0 \dar & 0 \dar & 0 \dar &   \\
 0 \rar & R^{\oplus a_1}\rar\dar & R^{\oplus b_1} \rar\dar & R^{\oplus c_1} \rar\dar & 0 \\
 0 \rar & R^{\oplus a_0}\rar\dar & R^{\oplus b_0} \rar\dar & R^{\oplus c_0} \rar\dar & 0 \\
 0 \rar & A\rar\dar & B \rar\dar & C \rar\dar & 0 \\
  & 0 & 0 & 0 & 
 \end{tikzcd} \]

Tensando las dos filas superiores, que son libres, nos quedarían dos filas sobre las que aplicar
el lema de la serpiente y obtener los funtores derivados tal y como los hemos definido.
\end{itemize}

\subsubsection*{5. Funtor Hom y dualidad}
\label{sec-4-4-3}
\begin{itemize}
\item 5.1. Adjunciones, de nuevo
\label{sec-4-4-3-1}
Ya sabemos que el funtor $Hom(N,\_)$ es adjunto derecho a $\_\otimes N$, ahora
estudiamos el funtor $Hom(\_,N)$.

\begin{theorem}
\textbf{Adjunción de Hom contravariante}. El funtor $Hom(\_,N)$ es adjunto derecho
de su funtor opuesto, $Hom^{op}(\_,N)$.
\end{theorem}

Aplicando currificación tenemos trivialmente:

\[Hom(L,Hom(M,N)) \cong Hom(M,Hom(L,N))\]

Que, teniendo en cuenta que estamos usando la categoría opuesta, prueba la
adjunción.

\begin{proposition}
\textbf{Exactitud de Hom}. Ambos funtores $Hom$ son adjuntos derechos y por tanto,
exactos por la izquierda. Teniendo en cuenta que uno es contravariante, quiere
decir que:

\[ A \overset{}\longrightarrow B \overset{}\longrightarrow C \overset{}\longrightarrow 0\]

Lleva a:

\[ 0 \overset{}\longrightarrow Hom(C,N) \overset{}\longrightarrow 
 Hom(B,N) \overset{}\longrightarrow Hom(A,N)\]
\end{proposition}

\item 5.2. Módulos duales.
\label{sec-4-4-3-2}
\begin{definition}
\textbf{Módulo dual}. El dual de un R-módulo $M$ es el módulo $M^{\vee} = Hom_R(M,R)$.
\end{definition}

Tenemos que $Hom(M,R^n) \cong M^{\vee} \otimes R^n$.
\end{itemize}

\subsubsection*{6. Módulos proyectivos e inyectivos, y el funtor Ext}
\label{sec-4-4-4}
\begin{itemize}
\item 6.1. Proyectividad e inyectividad
\label{sec-4-4-4-1}
\begin{definition}
\textbf{Módulos proyectivos e inyectivos}. Un R-módulo es \emph{proyectivo} si $Hom(P,\_)$
es exacto; e \emph{inyectivo} si $Hom(\_,P)$ es exacto.
\end{definition}

Esto es equivalente a decir que cada epimorfismo $M \longrightarrow N$ lleva un
morfismo $P \longrightarrow N$ a $P \longrightarrow M$, en el caso de \emph{proyectividad}:

\[ \begin{tikzcd}
  & P \dlar[swap,dashed]{\exists p'} \dar[swap]{p} \drar{0} & \\
 M \rar & N \rar & 0
 \end{tikzcd} \]

O que cada monomorfismo $L \longrightarrow M$ lleva un morfismo $L \longrightarrow Q$ a
un monomorfismo $M \longrightarrow Q$, en el de la \emph{inyectividad}:

\[ \begin{tikzcd}
  & Q & \\
 0 \urar{0} \rar & N \rar \uar[swap]{q} & M \ular[dashed,swap]{\exists q'}
 \end{tikzcd} \]

Además, esto es equivalente a decir que un módulo $P$ es \emph{proyectivo} si toda secuencia

\[ 0 \overset{}\longrightarrow L \overset{}\longrightarrow M \overset{}\longrightarrow P \overset{}\longrightarrow 0 \]

es escindida, y $Q$ es \emph{inyectivo} si toda secuencia:

\[ 0 \overset{}\longrightarrow Q \overset{}\longrightarrow M \overset{}\longrightarrow N \overset{}\longrightarrow 0 \]

es escindida.

\item 6.2. Módulos proyectivos
\label{sec-4-4-4-2}
\begin{theorem}
\textbf{Caracterización de proyectividad}. Un módulo es proyectivo ssi es el sumando
directo de un módulo libre.
\end{theorem}

Así, la suma directa de dos módulos proyectivos es proyectiva; el producto tensor
de dos módulos proyectivos es proyectivo, y todo módulo proyectivo es plano.

\item 6.3. Módulos inyectivos
\label{sec-4-4-4-3}
\begin{theorem}
\textbf{Caracterización de inyectividad}. Un módulo es \textbf{inyectivo} ssi toda aplicación
$f : I \longrightarrow Q$ extiende a una aplicación $\hat f : R \longrightarrow Q$, donde I es ideal de R.
\end{theorem}

\item 6.4. El funtor Ext
\label{sec-4-4-4-4}
Existirían dos formas naturales de definir \textbf{Ext}, que coinciden no trivialmente:

\begin{definition}
\textbf{Funtor Ext}. Dado $M$ con una resolución proyectiva:

\[ \dots \overset{}\longrightarrow P_1 \overset{}\longrightarrow P_0 \overset{}\longrightarrow M \overset{}\longrightarrow 0 \]

aplicamos el funtor contravariante $Hom(\_,N)$ eliminando $M$ para obtener:

\[ 0 \overset{}\longrightarrow Hom(P_0,N) \overset{}\longrightarrow Hom(P_1,N) \overset{}\longrightarrow Hom(P_2,N) \overset{}\longrightarrow \dots \]

Y tomamos la cohomología de este complejo $Hom(M_\bullet,N)$, dejando como definición:

\[Ext^i_R(M,N) = H^i(Hom(M_\bullet,N))\]
\end{definition}

\begin{definition}
\textbf{Funtor Ext}. Dado $N$ con una resolución inyectiva:

\[ 0 \overset{}\longrightarrow N \overset{}\longrightarrow Q_0 \overset{}\longrightarrow Q_1 \overset{}\longrightarrow \dots \]

aplicamos el funtor covariante $Hom(M,\_)$ eliminando $N$ para obtener:

\[ 0 \overset{}\longrightarrow 
 Hom(M,Q_0) \overset{}\longrightarrow 
 Hom(M,Q_1) \overset{}\longrightarrow 
 Hom(M,Q_2) \overset{}\longrightarrow \dots \]

Y tomamos la cohomología de este complejo $Hom(M,N_\bullet)$, dejando como definición:

\[Ext^i_R(M,N) = H^i(Hom(M,N_\bullet))\]
\end{definition}
\end{itemize}

\subsection*{IX. Álgebra homológica}
\label{sec-4-5}
\subsubsection*{Complejos y homología, de nuevo}
\label{sec-4-5-1}
\begin{itemize}
\item 3.1. Recordatorio de definiciones básicas
\label{sec-4-5-1-1}
\begin{definition}
\textbf{Resolución}. La \emph{resolución} de un objeto $A$ es un complejo
exacto excepto en un punto, donde es isomorfa a $A$.
\end{definition}

Esto es equivalente a tener un complejo exacto de la forma:

\[ \dots \overset{}\longrightarrow 
 M_2 \overset{}\longrightarrow 
 M_1 \overset{}\longrightarrow 
 M_0 \overset{}\longrightarrow 
 A \longrightarrow
 0\]

\item 3.2. La categoría de los complejos
\label{sec-4-5-1-2}
\begin{definition}
\textbf{Categoría de complejos de cocadenas}. La categoría $C(A)$ tiene como objetos
los complejos de cocadenas en una categoría $A$; y como morfismos entre dos 
cocadenas,   $Hom(M^\bullet,N^\bullet)$, los diagramas conmutativos entre ellas. Por ejemplo:

\[ \begin{tikzcd}
 \dots \rar & M^{i-1} \rar\dar{\alpha^{i-1}} & M^{i} \rar\dar{\alpha^{i}} &  M^{i+1} \rar\dar{\alpha^{i+1}} & \dots \\
 \dots \rar & N^{i-1} \rar & N^{i} \rar & N^{i+1} \rar & \dots
 \end{tikzcd} \]

representa el morfismo $\alpha_\bullet$.
\end{definition}

Esta es una categoría abeliana. De ella definiremos además dos variantes:

\begin{itemize}
\item $C^+(A)$, subcategoría plena de los complejos acotados por debajo.
\item $C^-(A)$, subcategoría plena de los complejos acotados por arriba.
\end{itemize}
\end{itemize}
\section*{Taller de Geometría y Topología}
\label{sec-5}
\subsection*{1. Construcciones con regla y compás}
\label{sec-5-1}
\subsubsection*{1.1. Construcciones posibles}
\label{sec-5-1-1}
\begin{itemize}
\item 1.1.0. Construcciones elementales
\label{sec-5-1-1-1}
Axiomáticamente consideramos realizables las siguientes construcciones
elementales:

\begin{enumerate}
\item Dados dos puntos, puede construirse un \textbf{segmento} entre ellos.
\item Todo segmento puede extenderse.
\item Dados dos puntos, puede construirse un \textbf{círculo} con centro y radio.
\item Dadas dos rectas secantes, puede construirse su \textbf{punto} de corte.
\item Dados círculo y rectas, puede construirse su \textbf{punto} de corte.
\item Dados círculos tangentes, pueden construirse \textbf{puntos} de corte.
\end{enumerate}

\begin{itemize}
\item Equivalencia de compases\hfill{}\textsc{extra}
\label{sec-5-1-1-1-1}
Asumimos un compás colapsable, pero podríamos usar uno no colapsable
con el mismo efecto, por la \href{https://en.wikipedia.org/wiki/Compass_equivalence_theorem}{equivalencia de compases}.
\end{itemize}

\item 1.1.1. Construcciones básicas
\label{sec-5-1-1-2}
\begin{itemize}
\item 1. Triángulo equilátero sobre un segmento
\label{sec-5-1-1-2-1}
\item 2. Copiar un segmento
\label{sec-5-1-1-2-2}
\item 3. Cortar segmento de otro dado
\label{sec-5-1-1-2-3}
\item 4. Bisecar ángulo
\label{sec-5-1-1-2-4}
\item 5. Mediatriz de un segmento
\label{sec-5-1-1-2-5}
\item 6. Perpendicular a través de un punto en una recta
\label{sec-5-1-1-2-6}
\item 7. Perpendicular a través de un punto fuera de una recta
\label{sec-5-1-1-2-7}
\item 8. Triángulo con longitudes de lados dada
\label{sec-5-1-1-2-8}
\item 9. Copiar un segmento a un segmento dado
\label{sec-5-1-1-2-9}
\item 10. Copiar un ángulo a un rayo
\label{sec-5-1-1-2-10}
\item 14. Paralela a una recta a través de un punto exterior
\label{sec-5-1-1-2-11}
\end{itemize}
\item 1.1.2. Construcciones involucrando razones geométricas
\label{sec-5-1-1-3}
\begin{itemize}
\item 15. Cortar un segmento en n partes iguales
\label{sec-5-1-1-3-1}
\item 16. Cortar un segmento en un racional
\label{sec-5-1-1-3-2}
\item 17. Media geométrica de dos segmentos
\label{sec-5-1-1-3-3}
\end{itemize}
\item 1.1.3. Construcciones involucrando áreas
\label{sec-5-1-1-4}
\begin{itemize}
\item 19. Paralelogramo con igual área que un triángulo dado
\label{sec-5-1-1-4-1}
\end{itemize}
\item 1.1.4. Circunferencias destacadas
\label{sec-5-1-1-5}
\begin{itemize}
\item 24. Centro de una circunferencia
\label{sec-5-1-1-5-1}
\item 25. Circunferencia inscrita a un triángulo
\label{sec-5-1-1-5-2}
Usando bisectrices.
\item 26. Circunferencia circunscrita a un triángulo
\label{sec-5-1-1-5-3}
\end{itemize}
\end{itemize}
\subsubsection*{1.2. Construcciones imposibles}
\label{sec-5-1-2}
\begin{itemize}
\item 1.2.1. Elementos constructibles o realizables
\label{sec-5-1-2-1}
Un real $x$ es constructible cuando podemos construir puntos $C,D$ tales que
$\overline{CD} = x$.

\begin{itemize}
\item Subcuerpo de números constructibles
\label{sec-5-1-2-1-1}
Los números constructibles forman un subcuerpo de $\mathbb{R}$. Llamamos $\mathfrak{C}$ al
cuerpo de los constructibles.

\begin{itemize}
\item {\bfseries\sffamily TODO} Demostración
\label{sec-5-1-2-1-1-1}
\end{itemize}
\item Los racionales son constructibles
\label{sec-5-1-2-1-2}
Todo subcuerpo de $\mathbb{R}$ debe contener a los racionales.

\item Las raíces de constructible son constructibles
\label{sec-5-1-2-1-3}
Si $x \in\mathfrak{C}$, entonces $\sqrt{|x|} \in \mathfrak{C}$.
\end{itemize}

\item 1.2.1. Extensión cuadrática
\label{sec-5-1-2-2}
Dado $\mathbb{K}$ subcuerpo de $\mathbb{R}$ llamamos a:

\[
\mathbb{K}(\sqrt{e}) = \{ a+ b\sqrt{e} \mid a,b\in\mathbb{K}\}
\]

una extensión cuadrática de $\mathbb{K}$.

\item 1.2.1. Teorema de Descartes
\label{sec-5-1-2-3}
Un número real es constructible ssi está en alguna extensión cuadrática
iterada de $\mathbb{Q}$.

\begin{itemize}
\item {\bfseries\sffamily TODO} Demostración
\label{sec-5-1-2-3-1}
\end{itemize}
\end{itemize}

\subsection*{2. Geometría no euclídea}
\label{sec-5-2}
\subsubsection*{El Quinto Postulado de Euclides}
\label{sec-5-2-1}
\begin{itemize}
\item Quinto postulado de Euclides
\label{sec-5-2-1-1}
El Quinto Postulado de Euclides afirma que, dadas $r,s,t$ rectas
cortando $r$ a $t,s$ con ángulos $\alpha,\beta$; si $\alpha+\beta < \pi$, entonces $s \cap t \neq \varnothing$.

\item Teorema de Legendre
\label{sec-5-2-1-2}
El Quinto Postulado equivale a que la suma de los ángulos de un
triángulo es exactamente $\pi$. Si la suma de los ángulos de un triángulo
es $\pi$, se cumple el Quinto Postulado.

\item Cuadriláteros de Saccheri
\label{sec-5-2-1-3}
Un cuadrilátero $ABCD$ es \textbf{de Saccheri} cuando $\widehat{A},\widehat{B}$ son rectos y 
además $\overline{AD}=\overline{BC}$.

\item Cuadrilátero de Lambert
\label{sec-5-2-1-4}
Un cuadrilátero es \textbf{de Lambert} si tiene tres ángulos rectos.

\item Independencia del Quinto Postulado
\label{sec-5-2-1-5}
Los siguientes resultados son independientes del Quinto Postulado

\begin{itemize}
\item la suma de los ángulos de un triángulo es menor o igual que $\pi$.
\item los dos ángulos no rectos de un cuadrilátero de Saccheri son
iguales y menores o iguales que $\pi/2$.
\item el lado superior de un cuadrilátero de Saccheri es menor que 
el lado inferior.
\end{itemize}

\item Implicación al Quinto Postulado
\label{sec-5-2-1-6}
Los siguientes resultados equivalen al Quinto Postulado

\begin{itemize}
\item los ángulos de un triángulo suman $\pi$.
\item los ángulos de todos los triángulos suman $\pi$.
\item un cuadrilátero de Saccheri tiene todos los ángulos rectos.
\end{itemize}
\end{itemize}

\subsubsection*{Geometría hiperbólica}
\label{sec-5-2-2}
\begin{itemize}
\item Axioma de Lobachevsky
\label{sec-5-2-2-1}
Dada una recta $r$ y un punto $a \notin r$, existen al menos dos rectas $s_1,s_2$
distintas con $a \in s_1\cap s_2$, pero $s_1\cap r = s_2\cap r = \varnothing$.

\item Infinitas rectas
\label{sec-5-2-2-2}
Dada una recta $r$ y un punto $a \notin r$, existen infinitas rectas que pasan
por $a$ y no contienen a $r$.
\end{itemize}

\subsubsection*{Paralelas en geometría hiperbólica}
\label{sec-5-2-3}
\begin{itemize}
\item Existencia de rectas dado un ángulo
\label{sec-5-2-3-1}
Dada una recta $r$ y un punto $a \notin r$, dado cualquier ángulo $\alpha \in (0,\pi)$ y
siendo $a \in t_a \perp r$, existen dos rectas $l_1,l_2$ formando un ángulo $\alpha$ con
$t_a$.

\item Existencia de paralelas
\label{sec-5-2-3-2}
Dada una recta $r$ y un punto $a \notin r$, tomamos $t_{a}$ la perpendicular por $a$
y $l_{\alpha}$ la dada con ángulo $\alpha$ por el teorema anterior. Existe un único
$\beta \in (0,\pi/2)$ tal que

\begin{itemize}
\item $l_{\beta}\cap r = \varnothing$,
\item $\forall a \in (0,\beta): l_{\alpha}\cap r \neq \varnothing$.
\end{itemize}

En este caso, la recta $l_{\beta}$ es \textbf{paralela}.

\item El paralelismo es independiente del punto elegido
\label{sec-5-2-3-3}
Si una recta es paralela a otra por un punto, lo es por todos sus
puntos.

\item El ángulo de paralelismo sólo depende de la distancia
\label{sec-5-2-3-4}
El ángulo de paralelismo de una recta por un punto sólo depende de
la distancia de la recta al punto.

\item Existen exactamente dos paralelas
\label{sec-5-2-3-5}
Dada una recta y un punto exterior, existen exactamente dos paralelas
distintas que pasan por el punto.

\item Simetría del paralelismo
\label{sec-5-2-3-6}
Si $r$ es paralela a $s$, $s$ es paralela a $r$.

\item Ultraparalelas
\label{sec-5-2-3-7}
Dos rectas se dicen ultraparalelas si no son secantes ni paralelas.

\item Caracterización de ultraparalelas
\label{sec-5-2-3-8}
Dos rectas son ultraparalelas si y sólo si admiten una perpendicular
común.
\end{itemize}

\subsubsection*{Defecto de triángulos}
\label{sec-5-2-4}
\begin{itemize}
\item Defecto de un triángulo
\label{sec-5-2-4-1}
El \textbf{defecto} de un triángulo $ABC$ es

\[ \mathrm{def}(ABC) =
\pi - {\widehat A} - \widehat B - \widehat C > 0.
\]

\item Los defectos de triángulos son aditivos
\label{sec-5-2-4-2}
Dado un $D \in AB$ y $ABC$ un triángulo, se tiene

\[ \mathrm{def}(ABC) = \mathrm{def}(ACD) + \mathrm{def}(BCD).
\]

\item Dos triángulos son congruentes si tienen los mismos ángulos
\label{sec-5-2-4-3}
Dos triángulos con los mismos ángulos en geometría hiperbólica
deben ser congruentes.

\item Paralela a una y perpendicular a otra
\label{sec-5-2-4-4}
Dadas dos semirectas formando un ángulo agudo, existe una única
perpendicular a una y paralela a la otra.

\item El defecto es el área
\label{sec-5-2-4-5}
El defecto es igual al área del triángulo.
\end{itemize}

\subsubsection*{Semiplano de Poincaré}
\label{sec-5-2-5}
\begin{itemize}
\item Definición
\label{sec-5-2-5-1}
Llamamos $\mathbb{H}^2 = \left\{ (x,y) \in \mathbb{R}^2\mid y>0
\right\}$ y $\forall p = (x,y) \in \mathbb{H}^2$ y sobre él tomamos 
la métrica $g_p = \frac{1}{y^{2}}\left\langle \cdot,\cdot \right\rangle$.
\end{itemize}

\subsection*{Ejercicios}
\label{sec-5-3}
\subsubsection*{Formas del universo}
\label{sec-5-3-1}
\begin{itemize}
\item Ejercicio 1
\label{sec-5-3-1-1}
\begin{statement}
Como seres de dimensión 3 en un mundo de tres dimensiones es fácil
dibujar y entender posibles formas de Planilandia pero no de nuestro
propio Universo. Imaginar la forma de nuestro Universo en alguna de
las siguientes situaciones:

\begin{itemize}
\item Hacemos una expedición a una galaxia remota. Al llegar a ella
descubrimos estar de vuelta en la tierra.

\item Un astrónomo acaba de descubrir que los mismos objetos se
encuentran en posiciones distintas del Universo.

\item Buscando ondas de radio que detecten señales extraterrestres,
detectamos una señxal que procede de una galaxia lejana.
Investigamos y vemos que se trata de la señal de un programa de TV
emitido hace 50 años.
\end{itemize}

A las posibles formas de nuestro Universo les llamaremos
variedades de dimensión 3 y su estudio constituye la Topología
3-dimensional.
\end{statement}

En cualquiera de esos casos podríamos estar ante un toro tridimensional.
Los objetos se repetirían en el espacio una y otra vez, y todo el espacio
se repetiría con ellos. Podríamos interpretar que vivimos en un cubo en el
que están identificadas cada una de las caras con la opuesta. Nótese que
seguría siendo un universo orientable.

En un segundo ejemplo, podría ocurrir que tuviéramos un universo similar
al anterior pero en el que una de las paredes del cubo cambiara la 
orientación. Al pasar por ella volveríamos al mismo punto pero habríamos
intercambiado derecha e izquierda.

En un tercer ejemplo, podría ocurrir que el universo fuera infinito en
una dimensión pero en las otras dos se comportara como un toro plano.
Dependiendo de la dirección que tomáramos, volveríamos o no al punto
de partida.

\item Ejercicio 2
\label{sec-5-3-1-2}
\begin{statement}
\begin{enumerate}
\item ¿Cuáles de las siguientes superficies tienen la misma topología?

\includegraphics[width=.9\linewidth]{./img/formasuniverso1.png}

\item En Planilandia CP descubrió que dos caminos cerrados partiendo del mismo
punto en direcciones opuestas no tienen por qué volver a cruzarse en un
punto diferente. ¿Es esta propiedad geométrica o topológica?
\item Describir superficies con la misma topología pero diferente geometría.
\end{enumerate}
\end{statement}

\begin{enumerate}
\item Tienen la misma topología:
\begin{itemize}
\item la esfera y el objeto justo debajo.
\item el toro y la taza.
\item el resto de objetos.
\end{itemize}
\item Esta es una propiedad topológica. El hecho seguirá siendo cierto al
aplicar deformaciones continuas al espacio.
\item Por ejemplo de un ejercicio anterior tomamos el toro y la taza, que
tenían la misma topología pero se puede observar que al hacer la
deformación del toro en la taza han cambiado propiedades como área
que es diferente para ambas figuras.
\end{enumerate}

\item Ejercicio 3
\label{sec-5-3-1-3}
\begin{statement}
Distingue según su topología intrínseca y extrínseca.

\includegraphics[width=.9\linewidth]{./img/formasuniverso2.png}

\begin{itemize}
\item ¿Puedes forrar un cilidro con parte de una hoja de papel sin deformarla?,
¿y un cono?, ¿y un trozo de esfera?
\item ¿Coḿo pueden los planilandeses que vivan en mundos como los de la figura
conocer que sus geometrías intrínsecas son diferentes? ¿Qué pueden decir
acerca de sus topologías extrínsecas e intrínsecas?

\includegraphics[width=.9\linewidth]{./img/formasuniverso3.png}
\end{itemize}

$\quad$
\end{statement}

Todas las figuras tienen misma topología intrínseca. En el caso de la
topología extrínseca podemos distinguir tres grupos:

\begin{itemize}
\item las figuras azules junto a la morada adyacente a estas.
\item las figuras amarillas.
\item la morada que nos queda.
\end{itemize}

Podemos forrar el cilindro y el cono, por el contrario se crearían
pliegues en el papel al intentar cubrir el trozo de esfera.

En el trozo de esfera al medir los ángulos de un triángulo el
resultado sería mayor que PI, en el hiperboloide sería menor que PI, y
en el plano sería igual a PI.

\item Ejercicio 4
\label{sec-5-3-1-4}
\begin{statement}
\begin{itemize}
\item Justifica que el toro llano y la superficie de un donut son
topológicamente equivalentes.

\item Consigue en la siguiente figura un de un toro llano tres \texttt{X}
    en línea.

\includegraphics[width=5cm]{./img/formasuniverso4.png}

\item Cuáles de las posiciones siguientes serían equivalentes al jugar
unas "tres en línea" sobre un toro llano.

\includegraphics[width=.9\linewidth]{./img/formasuniverso5.png}

\item En el siguiente tablero de ajedrez sobre un toro llano, ¿qué figuras
están amenazadas por el caballo blanco?

\includegraphics[width=5cm]{./img/formasuniverso6.png}

\item ¿Qué figuras están amenazadas por el caballo y la reina blancos?

\includegraphics[width=5cm]{./img/formasuniverso7.png}

\item Las siguientes figuras muestran un toro llano de tres dimensiones.
Explica cómo se construye e imagina qué verías al mirar en una dirección
concreta.

\includegraphics[width=5cm]{./img/formasuniverso8.png}
\end{itemize}

$\quad$
\end{statement}

\begin{itemize}
\item Triangulando el toro llano y el donut llegamos fácilmente a la misma
triangulación.
\item Escribimos X en la casilla inferior central.
\item Todos los celestes por un lado; todos los amarillos excepto los de la
primera columna y el último de la tercera columna, y todos los demás.
\item Rey, afil, y dos caballos negros.
\item Todas.
\item Identificando las caras, veríamos nuestra espalda delante, nuestros pies
hacia arriba y nuestra cabeza hacia abajo.
\end{itemize}

\item Ejercicio 5
\label{sec-5-3-1-5}
\begin{statement}
Jugando en la botella de Klein:

\begin{itemize}
\item ¿Cuáles de las siguientes posiciones ganan en el juego del tres en
raya dentro de una botella de Klein?

\includegraphics[width=.9\linewidth]{./img/formasuniverso10.png}

\item Analiza cómo hacer tres en línea en la siguiente figura.

\includegraphics[width=.9\linewidth]{./img/formasuniverso11.png}
\end{itemize}
\end{statement}

De las tres posiciones del tres en raya

\begin{itemize}
\item la primera gana.
\item la segunda gana.
\item la tercera gana.
\end{itemize}

Si escribimos coordenadas del tres en raya como

\begin{center}
\begin{tabular}{lll}
0,0 & 0,1 & 0,2\\
1,0 & 1,1 & 1,2\\
2,0 & 2,1 & 2,2\\
\end{tabular}
\end{center}

se ganan las partidas con la posición

\begin{itemize}
\item primera: (2,2)
\item segunda: (0,0)
\item tercera: (0,2)
\item cuarta: (2,2)
\item quinta: (1,1)
\end{itemize}

\item Ejercicio 6
\label{sec-5-3-1-6}
\begin{statement}
Actividades:

\begin{itemize}
\item Si un planilandés viviendo en un plano proyectivo cruza el ecuador,
¿vuelve como su imagen especular?
\item Una familia de planilandeses vive en un plano proyectivo. Planean
edificar dos gasolineras separadas cuanto más mejor. ¿Dónde deberían
construirlas?
\item CP conoce que vive en una esfera o en un plano proyectivo. ¿Cómo
podría saber cuál de los dos es su mundo?
\item Un segundo planilandés sabe que su Universo es un plano proyectivo o
una botella de Klein, ¿qué podría hacer para conocer de cuál de los dos
se trata?
\end{itemize}

$\quad$
\end{statement}

\begin{enumerate}
\item Sí, el plano no es orientable.
\item Una en el centro y otra en el ecuador.
\item Cruzando la frontera y al volver, comparando su orientación.
\item Ampliar primero una zona segura en la que pudiera volver a cada punto
sin haber cambiado la orientación y luego comprobar si el resto del
universo ha quedado dividido en dos (suma de planos) o en uno (plano).
\end{enumerate}

\item Ejercicio 7
\label{sec-5-3-1-7}
\begin{statement}
Haciendo sumas conexas:

\begin{itemize}
\item Deducir que si a una cinta de Möbius le pagamos un disco por el borde,
obtenemos un plano proyectivo.
\item Corrobora las palabras de Klein: "La cinta de Möbius es divina, si pegas
dos por su borde obtienes mi botella".
\item Construye usando papel la suma conexa de una cinta de Möbius a un toro
y a una botella de Klein.
\item Muestra que la suma conexa de un toro con un plano proyectivo es 
topológicamente equivalente a la suma conexa de una botella de Klein con
un plano proyectivo.
\item Establece una correspondencia por equivalencia topológica entre las
superficies de los conjuntos A y B:

\[
   A = \{
   \mathbb{T}^2\#\mathbb{S}^2,
   \mathbb{P}^2\#\mathbb{P}^2\#\mathbb{P}^2,
   \mathbb{B}^2,
   \mathbb{S}^2\#\mathbb{S}^2\#\mathbb{S}^2,
   \mathbb{P}^2\#\mathbb{T}^2,
   \mathbb{B}^2\#\mathbb{T}^2\#\mathbb{P}^2
   \}
   \]

\[
   B = \{
   \mathbb{P}^2\#\mathbb{P}^2,
   \mathbb{B}^2\#\mathbb{P}^2,
   \mathbb{S}^2\#\mathbb{S}^2,
   \mathbb{P}^2\#\mathbb{P}^2\#\mathbb{P}^2\#\mathbb{B}^2,
   \mathbb{T}^2,
   \mathbb{T}^2\#\mathbb{P}^2
   \}
   \]
\end{itemize}
\end{statement}

\begin{enumerate}
\item Un plano proyectivo menos un disco es una banda de Möbius porque
podemos dibujar el plano proyectivo en un disco y quitarle un disco
que corte la frontera del disco.
\item Nótese que si partimos la botella de Klein por la mitad, lo que queda
en cada una de las mitades es una banda de Möbius.
\item Serían al final 5 planos proyectivos.
\item Usando la clasificación de superficies compactas sabemos que no
necesitamos mezclar asas con gorros cruzados $\times \circ = \times^3$.
\item Usaremos clasificación de superficies compacatas para escribir cada
una como suma de planos proyectivos o toros. La $\mathbb{S}^2$ es neutra bajo la
suma conexa.

Nos quedan

\begin{itemize}
\item $\mathbb{B}^2 \cong \mathbb{P}\#\mathbb{P}$,
\item $\mathbb{B}^2 \#\mathbb{P}^2 \cong \mathbb{P}^2 \# \mathbb{P}^2 \# \mathbb{P}^2$,
\item $\mathbb{S}^2\# \mathbb{S}^2 \cong \mathbb{S}^2 \# \mathbb{S}^2 \# \mathbb{S}^2$,
\item $\mathbb{P}^2\#\mathbb{P}^2\#\mathbb{P}^2\#\mathbb{B}^2 \cong \mathbb{B}^2\#\mathbb{T}^2\#\mathbb{P}^2$,
\item $\mathbb{T}^2 \cong \mathbb{T}^2\#\mathbb{S}^2$,
\item $\mathbb{T}^2\#\mathbb{P}^2 \cong \mathbb{P}^2\#\mathbb{T}^2$.
\end{itemize}
\end{enumerate}


Las geodésicas del modelo vienen dadas por intersecciones del hiperboloide
con subespacios lineales bidimensionales, que nunca serán vacíos
\end{itemize}
\section*{Teoría de Números y Criptografía}
\label{sec-6}
\subsection*{Factorización}
\label{sec-6-1}
\subsubsection*{Métodos básicos de factorización}
\label{sec-6-1-1}
\begin{itemize}
\item Fuerza bruta
\label{sec-6-1-1-1}
Probando a dividir cada número hasta $\sqrt{n}$.

\item Método de Fermat
\label{sec-6-1-1-2}
Escribimos un número como diferencia de cuadrados para factorizarlo como
$n = t^2 - s^2 = (t+s)(t-s)$.

\begin{itemize}
\item Soluciones en congruencias no triviales
\label{sec-6-1-1-2-1}
Si $(t,s)$ es una solución no trivial de $x^2 \equiv y^2$, entonces $gcd(n,t+s)\neq 1$
y $gcd(n,t-s) \neq 1$.
\end{itemize}
\end{itemize}

\subsubsection*{Método de factor base}
\label{sec-6-1-2}
Una \textbf{base} es un conjunto $B = \{p_0 = -1,p_1,\dots,p_h\}$ donde $p_0,p_1,\dots,p_n$
son enteros primos.

\begin{itemize}
\item Conjunto de candidatos B-números
\label{sec-6-1-2-1}
Un conjunto de candidatos B-números es $\mathbb{Z}_n$ escribiendo la
mitad de números más grandes dentro de la base, como:

\[
\mathbb{Z}_n =
\{0,1\dots,-2,-1\}
\]

Llamamos $\operatorname{abmod}$ a la clase de equivalencia del número en el conjunto 
de candidatos.

\item B-número
\label{sec-6-1-2-2}
Un número $b$ es B-número respecto de $n$ si $\operatorname{abmod}(b^2,n)$ factoriza por los 
elementos de la base $B$.

\item Alfa-vectores
\label{sec-6-1-2-3}
Dado $b$ un B-número, con:

\[
\operatorname{abmod}(b^2,n) = p_0^{e_0}\dots p_h^{e_h}
\]

definimos el \textbf{α-vector} como $\alpha vect(b) = (e_0,e_1,\dots,e_h)$.

\item {\bfseries\sffamily TODO} Idea del algoritmo
\label{sec-6-1-2-4}
Sea $S_0 = \{\alpha_0,\dots,\alpha_r\}$ un α-vector del conjunto de B-números.
\end{itemize}

\subsubsection*{Métodos de elección de la base B y los B-números}
\label{sec-6-1-3}
\begin{itemize}
\item Algoritmo: voy a tener suerte
\label{sec-6-1-3-1}
Se eligen los $h \in \mathbb{N}$ primeros números primos. Escogemos dos índices
$k_{max} \leq i_{max}$ y calculamos:

\begin{itemize}
\item $\lfloor \sqrt{n}\rfloor, \lfloor \sqrt{2n}\rfloor, \dots, \lfloor \sqrt{k_{max}n}\rfloor$
\item $\lfloor \sqrt{n}\rfloor+1, \lfloor \sqrt{2n}\rfloor+1, \dots, \lfloor \sqrt{k_{max}n}\rfloor+1$
\item $\dots$
\item $\lfloor \sqrt{n}\rfloor + i_{max}, \lfloor \sqrt{2n}\rfloor+i_{max}, \dots, \lfloor \sqrt{k_{max}n}\rfloor+i_{max}$
\end{itemize}

Entre estos, buscamos B-números y calculamos los α-vectores 
correspondientes.

\item Algoritmo: voy a forzar la suerte
\label{sec-6-1-3-2}
\end{itemize}
\subsubsection*{Fracciones continuas}
\label{sec-6-1-4}
\begin{itemize}
\item Fracción continua
\label{sec-6-1-4-1}
Notamos una fracción continua como:

\[
[a_0,a_1,a_2,\dots] =
a_0 + \frac{1}{a_1+\frac{1}{a_2+\dots}}
\]

\item Propiedades de las fracciones continuas
\label{sec-6-1-4-2}
Las fracciones continuas cumplen:

\begin{enumerate}
\item Todo racional se expresa como fracción continua finita.
\item Todo real se expresa como fracción continua.
\item Se cumple la fórmula:

\[\frac{a+\sqrt{d}}{b} = [a_0,[a_1,\dots,a_r]]\]

Si notamos $[a_0,[a_1,\dots,a_r]] = [a_0,a_1,\dots,a_r,a_1,\dots,a_r,\dots]$.
\end{enumerate}

\item Cálculo de la fracción continua de un real
\label{sec-6-1-4-3}
Sea $x \in \mathbb{R}$, tomamos $a_0 = \lfloor x \rfloor$ y podemos escribir recursivamente la 
fracción continua como:

\[
x = a_0 + \frac{1}{x_1^{-1}}
\]
\end{itemize}
\section*{Álgebra moderna}
\label{sec-7}
\subsection*{1. Construcción de anillos}
\label{sec-7-1}
\subsubsection*{1.1. Anillos}
\label{sec-7-1-1}
\begin{itemize}
\item Anillos
\label{sec-7-1-1-1}
Un \textbf{anillo} es $(R,+,\times,1)$ siendo:

\begin{enumerate}
\item $(R,+)$ un grupo aditivo abeliano.
\item $(R,\times,1)$ monoide multiplicativo.
\item $\times$ distributivo con $+$.
\end{enumerate}

Llamamos $0$ al elemento neutro de la suma.

\begin{itemize}
\item Anillos conmutativos
\label{sec-7-1-1-1-1}
Llamamos \textbf{anillo conmutativo} a un anillo con $\times$ conmutativo.

\item Propiedades en anillos
\label{sec-7-1-1-1-2}
Sea $r_1,r_2 \in R$ anillo:

\begin{itemize}
\item $0r=0=r0$
\item $r_1(-r_2) = -r_1r_2 = (-r_1)r_2$
\item $r(r_1-r_2) = rr_1-rr_2$
\item $r_1+r_2=r_2+r_1$
\end{itemize}

\begin{itemize}
\item Demostración
\label{sec-7-1-1-1-2-1}
Usando distributividad se prueban trivialmente.
\end{itemize}
\end{itemize}

\item Morfismos de anillos
\label{sec-7-1-1-2}
Un $f : R \to S$ es homomorfismo de anillos cuando:

\begin{itemize}
\item $f(r_1+r_2) = f(r_1)+f(r_2)$
\item $f(r_1r_2) = f(r_1)f(r_2)$
\item $f(1) = 1$
\end{itemize}

\begin{itemize}
\item Categoría de los anillos
\label{sec-7-1-1-2-1}
La composición de dos morfismos de anillos es morfismo de anillos y
la identidad es morfismo de anillos. Los anillos unitales forman así
una categoría $\mathtt{Ring}$.

\item Isomorfismos de anillos
\label{sec-7-1-1-2-2}
\end{itemize}
\item Subanillos
\label{sec-7-1-1-3}
\item Retículo de subanillos
\label{sec-7-1-1-4}
\item Ideales
\label{sec-7-1-1-5}
\item Ideales extendidos y contraidos
\label{sec-7-1-1-6}
\item Retículo de ideales
\label{sec-7-1-1-7}
\item Ejemplo: Matrices infinitas
\label{sec-7-1-1-8}
\item Ejemplo: Álgebra de Weyl
\label{sec-7-1-1-9}
Se llama \textbf{álgebra de Weyl} al anillo de operadores en los polinomios
generado por $X$ (multiplicación por la indeterminada) y $\frac{\partial}{\partial x}$ (diferenciación);
con la composición como producto.

\begin{itemize}
\item Caracterización
\label{sec-7-1-1-9-1}
El álgebra de Weyl es isomorfa a:

\[
\frac{K[X,Y]}{(YX-XY-1)}ñ
\]
\end{itemize}

\item Ejemplo: Anillo de un monoide
\label{sec-7-1-1-10}
Dado un monoide multiplicativo $M$, definimos $R[M]$ como los polinomios que
usan como exponentes los elementos de $M$. Es decir,

\[
\sum_i r_i[m_i]
\]

Y forma un álgebra definiendo:

\begin{itemize}
\item Suma: $\sum_i r_i[m_i] + \sum_i r_i'[m_i] = \sum_i (r_i+r_i')[m_i]$
\item Producto: $\left(\sum_i r_i[m_i]\right)\left(\sum_i r_i'[m_i]\right) = \sum_k \left(\sum_{m_im_j=m_k} (r_ir_j')[m_k]\right)$
\end{itemize}

\begin{itemize}
\item Generaliza al anillo de polinomios
\label{sec-7-1-1-10-1}
Nótese que generaliza al anillo de polinomios en una variable cuando 
el monoide es $\mathbb{N}$, y que generaliza al anillo de polinomios en varias 
variables cuando el monoide es $\mathbb{N}^n$.
\end{itemize}

\item {\bfseries\sffamily TODO} Monoide libre
\label{sec-7-1-1-11}
\end{itemize}
\subsubsection*{1.2. Construcción de anillos}
\label{sec-7-1-2}
\begin{itemize}
\item Anillo cociente
\label{sec-7-1-2-1}
\begin{itemize}
\item Proyección
\label{sec-7-1-2-1-1}
\end{itemize}
\item Propiedad universal del anillo cociente
\label{sec-7-1-2-2}
\item Primer teorema de isomorfía
\label{sec-7-1-2-3}
\item Segundo teorema de isomorfía
\label{sec-7-1-2-4}
\item Tercer teorema de isomorfía
\label{sec-7-1-2-5}
\item Producto directo
\label{sec-7-1-2-6}
\item Caracterización del producto por ortogonales centrales idempotentes
\label{sec-7-1-2-7}
\item Anillo opuesto
\label{sec-7-1-2-8}
\item Centro
\label{sec-7-1-2-9}
\item Propiedad universal del anillo de un monoide
\label{sec-7-1-2-10}
\item Anillo de polinomios
\label{sec-7-1-2-11}
\item Propiedad universal del anillo de polinomios
\label{sec-7-1-2-12}
\end{itemize}
\subsubsection*{1.3. Módulos}
\label{sec-7-1-3}
\begin{itemize}
\item R-módulos
\label{sec-7-1-3-1}
\begin{itemize}
\item Caracterización por anillo opuesto
\label{sec-7-1-3-1-1}
\end{itemize}
\item Morfismo de R-módulos
\label{sec-7-1-3-2}
\item Submódulos
\label{sec-7-1-3-3}
\begin{itemize}
\item Ideales como submódulos
\label{sec-7-1-3-3-1}
\end{itemize}
\item Módulo cociente
\label{sec-7-1-3-4}
\item Propiedad universal del módulo cociente
\label{sec-7-1-3-5}
\item Retículo de submódulos
\label{sec-7-1-3-6}
\begin{itemize}
\item Intersección de submódulos
\label{sec-7-1-3-6-1}
\item Suma de submódulos
\label{sec-7-1-3-6-2}
\end{itemize}
\item Submódulos maximales
\label{sec-7-1-3-7}
\end{itemize}
\subsubsection*{{\bfseries\sffamily TODO} 1.4. Categorías y funtores}
\label{sec-7-1-4}
\subsubsection*{1.5. La categoría Mod-R}
\label{sec-7-1-5}
\begin{itemize}
\item Caracterización de monomorfismos y epimorfismos
\label{sec-7-1-5-1}
\item Primer teorema de isomorfía
\label{sec-7-1-5-2}
\item Segundo teorema de isomorfía
\label{sec-7-1-5-3}
\item Tercer teorema de isomorfía
\label{sec-7-1-5-4}
\item Producto directo
\label{sec-7-1-5-5}
\item Suma directa
\label{sec-7-1-5-6}
\item Límites
\label{sec-7-1-5-7}
\item Colímites
\label{sec-7-1-5-8}
\item Ejemplos de límite
\label{sec-7-1-5-9}
\item Cambios de anillo
\label{sec-7-1-5-10}
\url{https://en.wikipedia.org/wiki/Change_of_rings}
\end{itemize}
\subsection*{2. Construcción de módulos}
\label{sec-7-2}
\subsubsection*{2.1. Producto tensor}
\label{sec-7-2-1}
\begin{itemize}
\item Aplicaciones bilineales
\label{sec-7-2-1-1}
Sean $M$ un R-módulo derecho y $N$ un R-módulo izquierdo. Un homomorfismo de
grupos es R-bilineal si:

\begin{itemize}
\item $\varphi(m_1+m_2,n) = \varphi(m_1,n) + \varphi(m_2,n)$
\item $\varphi(m,n_1+n_2) = \varphi(m,n_1) + \varphi(m,n_2)$
\item $\varphi(mr,n) = \varphi(m,rn)$
\end{itemize}

\item Producto tensor
\label{sec-7-2-1-2}
Construimos el grupo producto tensor como el grupo libre generado por
los elementos del producto cartesiano, dividido por el grupo generado 
por las relaciones de bilinealidad:

\[
M \otimes_R N = \frac{\langle
(m,n) \mid m \in M, n \in N
\rangle}{B}
\]

Donde $B$ está generado por:

\begin{itemize}
\item $(m_1+m_2,n) - (m_1,n) - (m_2,n)$
\item $(m,n_1+n_2) - (m,n_1) - (m,n_2)$
\item $(mr,n)-(m,rn)$
\end{itemize}

Nótese que además tenemos la proyección $b : M \times N \to M \otimes N$.

\item Propiedad universal del producto tensor
\label{sec-7-2-1-3}
Sean $M_R, _RN$ módulos con $f : M \times N \to X$ bilineal, existe un
único homomorfismo de grupos $\overline{f} : M \oplus_R N \to X$ tal que conmuta:

\[\begin{tikzcd}
M \times N \rar{b}\drar[swap]{f} & 
M \otimes_R N \dar[dashed]{\exists! \overline{f}} \\
& X
\end{tikzcd}\]

\begin{itemize}
\item {\bfseries\sffamily TODO} Demostración
\label{sec-7-2-1-3-1}
\end{itemize}

\item Neutro del producto tensor
\label{sec-7-2-1-4}
Se cumple $M \otimes R \cong M$ y $R \otimes N \cong N$.

\item {\bfseries\sffamily TODO} Producto tensor en anillos conmutativos
\label{sec-7-2-1-5}
\item Producto tensor de álgebras
\label{sec-7-2-1-6}
Si $R,S$ son dos A-álgebras, su producto tensor lo es con el producto
dado por:

\[
(r_1 \otimes s_1)(r_2 \otimes s_2) = (r_1r_2)\otimes(s_1s_2)
\]

\begin{itemize}
\item {\bfseries\sffamily TODO} Inclusiones en el producto tensor de álgebras
\label{sec-7-2-1-6-1}
\end{itemize}

\item {\bfseries\sffamily TODO} Producto tensor de álgebras como coproducto
\label{sec-7-2-1-7}
\end{itemize}
\subsubsection*{2.2. Módulos a dos lados}
\label{sec-7-2-2}
\begin{itemize}
\item Módulos a dos lados
\label{sec-7-2-2-1}
Un $M$ R-módulo izquierda y S-módulo derecha se llama \textbf{(R;S)-módulo a dos lados}
si cumple:

\[
r(ms)=(rm)s
\]

\item {\bfseries\sffamily TODO} Caracterización de módulos a dos lados
\label{sec-7-2-2-2}
\item {\bfseries\sffamily TODO} Módulos a dos lados balanceados y fieles
\label{sec-7-2-2-3}
\item {\bfseries\sffamily TODO} Propiedad universal del producto tensor como módulo a dos lados
\label{sec-7-2-2-4}
\end{itemize}
\subsubsection*{2.3. El retículo de submódulos}
\label{sec-7-2-3}
\begin{itemize}
\item Categoría de los conjuntos parcialmente ordenados
\label{sec-7-2-3-1}
Tomamos la \textbf{categoría de los conjuntos parcialmente ordenados} ${\cal P}$, siendo 
sus morfismos las aplicaciones crecientes:

\[
x \leq y \implies f(x) \leq f(y)
\]

\begin{itemize}
\item Submódulos como conjuntos parcialmente ordenados
\label{sec-7-2-3-1-1}
Existe el funtor covariante retículo ${\cal L} : \mathtt{Mod-}R \to {\cal P}$ que lleva cada módulo 
en su retículo de submódulos y cada homomorfismo de módulos lo aplica sobre
cada submódulo del retículo:

\[
{\cal L}(f)(N) = f(N)
\]

Y existe el funtor contravariante retículo ${\cal L} : \mathtt{Mod-}R \to {\cal P}$ que lleva cada
módulo en su retículo y cada homomorfismo de módulos lo aplica de manera
inversa sobre cada submódulo del retículo:

\[
{\cal L}(f)(N) = f^{-1}(N)
\]
\end{itemize}

\item Retículos
\label{sec-7-2-3-2}
Un \textbf{retículo} es un conjunto parcialmente ordenado donde todo par de
elementos tiene supremo e ínfimo, llamados $a \vee b$ y $a \wedge b$.

\item Propiedades del retículo
\label{sec-7-2-3-3}
En todo retículo $({\cal L}, \leq)$ se verifican:

\begin{enumerate}
\item Idempotencia, $a \vee a = a$
\item Conmutatividad, $a \vee b = b \vee a$
\item Asociatividad, $a \vee (b \vee c) = (a \vee b) \vee c$
\item Absorción, $a \vee (a \wedge c) = a$
\end{enumerate}

Y sus duales:

\begin{enumerate}
\item Idempotencia, $a \wedge a = a$
\item Conmutatividad, $a \wedge b = b \wedge a$
\item Asociatividad, $a \wedge (b \wedge c) = (a \wedge b) \wedge c$
\item Absorción, $a \wedge (a \vee c) = a$
\end{enumerate}

\item Retículo abstracto
\label{sec-7-2-3-4}
Llamamos \textbf{retículo abstracto} a un conjunto $L$ con operaciones $\vee,\wedge$ que
cumplen las propiedades de retículo.

\begin{itemize}
\item Orden en un retículo abstracto
\label{sec-7-2-3-4-1}
Un retículo abstracto determina una relación de orden, y además
se cumple en él:

\[
a \vee b = b \iff a \wedge b = a
\]

\begin{itemize}
\item {\bfseries\sffamily TODO} Demostración
\label{sec-7-2-3-4-1-1}
\begin{itemize}
\item Relación de orden
\label{sec-7-2-3-4-1-1-1}
\item Propiedad
\label{sec-7-2-3-4-1-1-2}
\end{itemize}
\end{itemize}
\item Homomorfismo de retículos abstractos
\label{sec-7-2-3-4-2}
Un homomorfismo de retículos abstractos es una aplicación preservando
supremos e ínfimos.

\[
f(a\wedge b) = f(a) \wedge f(b) \qquad f(a\vee b) = f(a) \vee f(b)
\]

\item Categoría de retículos abstractos
\label{sec-7-2-3-4-3}
La categoría de retículos abstractos contiene a los retículos y los
homomorfismos de retículos entre ellos. Es una categoría isomorfa
a la categoría de conjuntos parcialmente ordenados.
\end{itemize}

\item Retículo de submódulos de un módulo
\label{sec-7-2-3-5}
Los submódulos forman un retículo con:

\begin{itemize}
\item $N \vee N' = N + N'$
\item $N \wedge N' = N \cap N'$
\end{itemize}

\item Retículos acotados
\label{sec-7-2-3-6}
Un retículo con cero y uno se llama acotado, donde:

\begin{itemize}
\item El elemento cero cumple: $a \wedge 0 = 0$
\item El elemento uno cumple:  $a \vee 1 = 1$
\end{itemize}

\item Retículos modulares
\label{sec-7-2-3-7}
Llamamos retículo modular al que cumple la \textbf{ley modular}:

\[
N_1 \vee (N_2 \wedge N_3) = (N_1 \vee N_2) \wedge N_3
\]

\begin{itemize}
\item El retículo de submódulos es modular
\label{sec-7-2-3-7-1}
Los submódulos forman retículos modulares.

\begin{itemize}
\item {\bfseries\sffamily TODO} Demostración
\label{sec-7-2-3-7-1-1}
\end{itemize}
\end{itemize}

\item Retículos completos
\label{sec-7-2-3-8}
Un retículo en el que existe el supremo e ínfimo de cualquier familia de
submódulos se dice \textbf{completo}.

\begin{itemize}
\item El retículo de submódulos es completo
\label{sec-7-2-3-8-1}
El retículo de submódulos es completo, siendo el supremo e ínfimo de
cada familia $\{ N_i \mid i \in I\}$:

\begin{itemize}
\item $\bigvee N_i = \sum N_i$

\item $\bigwedge N_i = \bigcap N_i$
\end{itemize}
\end{itemize}

\item {\bfseries\sffamily TODO} Retículos superiormente continuos y compactamente generados
\label{sec-7-2-3-9}
\end{itemize}
\subsubsection*{2.5. Módulos finitamente generados}
\label{sec-7-2-4}
\begin{itemize}
\item {\bfseries\sffamily TODO} Módulos finitamente generados
\label{sec-7-2-4-1}
\item {\bfseries\sffamily TODO} Construcción de finitamente generados
\label{sec-7-2-4-2}
\item {\bfseries\sffamily TODO} Submódulo maximal
\label{sec-7-2-4-3}
\item {\bfseries\sffamily TODO} Caracterización de finitamente generados
\label{sec-7-2-4-4}
\item {\bfseries\sffamily TODO} Homomorfismos a la suma directa
\label{sec-7-2-4-5}
\item {\bfseries\sffamily TODO} Conmutación con sumas directas
\label{sec-7-2-4-6}
\item {\bfseries\sffamily TODO} Compacidad
\label{sec-7-2-4-7}
\item {\bfseries\sffamily TODO} Módulos noetherianos
\label{sec-7-2-4-8}
\item {\bfseries\sffamily TODO} Módulos noetherianos: propiedades
\label{sec-7-2-4-9}
\end{itemize}
\subsubsection*{{\bfseries\sffamily TODO} 2.6. Módulos noetherianos}
\label{sec-7-2-5}
\subsection*{3. Sumas directas y productos directos de módulos}
\label{sec-7-3}
\subsubsection*{3.1. Biproducto de módulos}
\label{sec-7-3-1}
\begin{itemize}
\item Biproducto de módulos
\label{sec-7-3-1-1}
Se llama \textbf{biproducto de módulos} a la terna $(M_1\oplus M_2, \{p_1,p_2\}, \{q_1,q_2\})$,
con las proyecciones e inclusiones de producto y coproducto.

\begin{itemize}
\item Propiedades del biproducto
\label{sec-7-3-1-1-1}
Las composiciones de proyecciones e inyecciones cumplen:

\begin{itemize}
\item $p_1q_1 = id_1$
\item $p_2q_2 = id_2$
\item $p_1q_2 = 0$
\item $p_2q_1 = 0$
\item $q_1p_1 + q_2p_2 = id$
\end{itemize}
\end{itemize}
\end{itemize}

\subsubsection*{{\bfseries\sffamily TODO} 3.2. Independencia y sumas directas}
\label{sec-7-3-2}
\subsubsection*{{\bfseries\sffamily TODO} 3.3. Módulos libres}
\label{sec-7-3-3}
\subsubsection*{{\bfseries\sffamily TODO} 3.4. Descomposición de anillos}
\label{sec-7-3-4}
\subsection*{Ejercicios}
\label{sec-7-4}
\subsubsection*{Semana 1}
\label{sec-7-4-1}
\begin{statement}
Prueba que los ideales (biláteros) del anillo $M_n(R)$ son de la forma
$M_R(\mathfrak{a})$, para un ideal (bilátero) $\mathfrak{a} \subseteq R$.
\end{statement}

Llamamos $E_{ij}$ a la matriz que tiene todas sus entradas nulas excepto
la entrada $i,j$. Dada $N$, una matriz con elementos $N = (n_{ij})_{i,j}$, el
producto de matrices es:

\[
E_{ia}NE_{bj} = n_{ab}E_{ij}
\]

Es decir, si una matriz $N$ está en un ideal bilátero $J$, todas las 
matrices de la forma $n_{ab}E_{ij}$ estarán también en el ideal.

Esto nos da por un lado que los elementos que aparecen en matrices 
del ideal forman un ideal $I$, si $a,b$ están en el ideal, $(a+\alpha b)E_{ij}$ 
estará en el ideal. Y además, cualquier matriz de la forma $xE_{ij}$ para
$x \in I$ está en el ideal. Así, el ideal es de la forma:

\[
J = M_R(\mathfrak{a})
\]

\subsubsection*{Semana 2}
\label{sec-7-4-2}
\begin{statement}
Prueba que $A[|X|]$, el anillo de las series formales de potencias en una
indeterminada $X$, es isomorfo al límite inverso del sistema de anillos
dirigido inferiormente $(\{\frac{A[X]}{(X^n)}\}, \{f_{n,m}\}_{n \geq m})$, donde $f_{n,m} : \frac{A[X]}{(X^n)} \to \frac{A[X]}{(X^m)}$ es
el homomorfismo de anillos definido por $f_{n,m}(\overline{X}) = \overline{X}$, si $n \geq m$.
\end{statement}

\begin{itemize}
\item Subanillo isomorfo
\label{sec-7-4-2-1}
Empezamos definiendo un subanillo del producto de los anillos $\frac{A[X]}{(X^n)}$.
Nótese que es subanillo por ser las funciones $f_{i,j}$ morfismos de anillos:

\[
H = \left\{
(p_i)_i \in \prod_i \frac{A[X]}{(X^i)}
\;\middle|\;
f_{i,j}(m_i) = m_j
\right\}
\]

Comprobaremos que es isomorfo a $A[|X|]$; para ello definimos la función
siguiente:

\[
g(a_0+a_1X+a_2X^2+\dots) = (a_0,a_0+a_1X,a_0+a_1X+a_2X^2,\dots)
\]

Es trivialmente inyectiva porque si $p \neq q$, se diferenciarán en el primer
polinomio en el que tengan un coeficiente distinto. Es trivialmente
sobreyectiva porque si tengo un elemento de la forma $(u_0,u_1,\dots) \in H$,
se debe tener $u_n = u_{n-1} + a_n X_n$, para $u_{n-1}$ de grado $n-1$. Esto asegura
que el elemento será de la forma:

\[
(a_0,a_0+a_1X, a_0+a_1X+a_2X^2,\dots) = g(a_0+a_1X+a_2X^2+\dots)
\]

\item Límite inverso
\label{sec-7-4-2-2}
Para probar que es límite inverso, probaremos que si existiera un $Z$ con
morfismos $\phi_n : Z \to \frac{A[X]}{(X^n)}$ cumpliendo $\phi_m = f_{n,m} \circ \phi_n$, existiría un único 
morfismo $Z \to H$ haciendo conmutar el diagrama:

\[\begin{tikzcd}
\frac{A[X]}{(X^0)} \rar &
\frac{A[X]}{(X^1)} \arrow{rr} &&
\frac{A[X]}{(X^2)} \rar &
\dots \\
&&
H \arrow{ull} \ular \urar \arrow{urr}
& & \\
& &
Z
\arrow[bend left]{uull} \arrow[bend left]{uul}
\arrow[bend right]{uurr} \arrow[bend right]{uur}
\uar[dashed]{!\exists}
&&
\end{tikzcd}\]

Ahora bien, dado $Z$ y los morfismos $\phi_n$, por propiedad universal del
producto directo, tenemos que existe un único morfismo $h: Z \to \prod_i \frac{A[X]}{(X^i)}$,
cumpliendo además que $\phi_n = \pi_n \circ h$. Aplicando $f_{n,m}$ tenemos:

\[
f_{n,m} \circ \pi_n \circ h = f_{n,m} \circ \phi_n =
\phi_m = \pi_m\circ h
\]

Lo que nos da que $im(h) \subseteq H$ y por tanto el morfismo buscado, que hereda
la unicidad.
\end{itemize}

\subsubsection*{Semana 3}
\label{sec-7-4-3}
\begin{itemize}
\item Ejercicio 1.5.
\label{sec-7-4-3-1}
\begin{statement}
Sea $R$ un anillo y $0 \neq e \in R$ un elemento idempotente; llamamos $f = 1-e$.

\begin{enumerate}
\item Prueba que $eRe$ y $fRf$ son anillos.
\item Prueba que $eRf$ es un $(eRe;fRf)$ módulo, y $fRe$ es un $(fRf,eRe)$ módulo.
\item Prueba que existe un homomorfismo inyectivo de anillos

\$$\lambda$ : R \longrightarrow 
\begin{pmatrix}
eRe \& eRf \\
    fRe \& fRf 
\end{pmatrix}\$, definido: $\lambda(r) = \begin{pmatrix} ere&erf\\fre&frf \end{pmatrix}$ para cada $r \in R$.

\item Prueba que existe un isomorfismo de grupos abelianos
$Hom_R(eR_R,fR_R) \cong fRe$, y un isomorfismo de anillos $End_R(eR_R) \cong eRe$.
\item Prueba que:

\[
    R \overset{\beta}\cong End_R(R_R) = 
    End_R((eR\oplus fR)_R) \cong \begin{pmatrix} eRe&eRf\\fRe&fRf \end{pmatrix}
    \]

siendo $\beta(r)(x) = rx$. Como consecuencia $\lambda$ es un isomorfismo de anillos.
\end{enumerate}

$\quad$
\end{statement}

\begin{itemize}
\item Punto 1
\label{sec-7-4-3-1-1}
Por propiedad distributiva, son cerrados con la misma suma que $R$. Son
trivialmente cerrados con el producto y nos falta comprobar que contiene
un elemento unidad, que es $e$ y es neutro gracias a ser idempotente:

\[
e(ere) = ere = (ere)e
\]

Nótese que $f$ es también idempotente y se repite el razonamiento.

\item Punto 2
\label{sec-7-4-3-1-2}
Comprobamos que $eRf$ es cerrado para la suma, y además:

\begin{itemize}
\item $(ese)(erf) = e(ser)f$
\item $(erf)(ftf) = e(rft)f$
\end{itemize}

Por lo que es un módulo a izquierda para $eRe$ y a derecha para $fRf$.

Nótese que el caso de $fRe$ es simétrico.

\item Punto 3
\label{sec-7-4-3-1-3}
Nótese que $\lambda$ preserva sumas trivialmente. Debemos comprobar que respeta
la unidad y el producto. Notamos primero gracias a que $ef=0$ tenemos:

\[
\lambda(1) = \begin{pmatrix}e&0\\0&f\end{pmatrix}
\]

Que se comprueba trivialmente que es el uno de su anillo, ya que es neutro
respecto al producto:

\[\begin{pmatrix}e&0\\0&f\end{pmatrix}\begin{pmatrix}er_1e&er_2f\\fr_3e&fr_4f\end{pmatrix} =\begin{pmatrix}er_1e&er_2f\\fr_3e&fr_4f\end{pmatrix}\]

Por último comprobamos que el producto se preserva:

\[\begin{pmatrix}ere&erf\\fre&frf\end{pmatrix}\begin{pmatrix} ese & esf \\ fse & fsf \end{pmatrix}
= \begin{pmatrix}erse&ersf\\frse&frsf\end{pmatrix}\]

Donde usamos crucialmente que $erese+erfse=er(e+f)se=erse$.

\item Punto 4
\label{sec-7-4-3-1-4}
\begin{itemize}
\item Isomorfismo de grupos abelianos
\label{sec-7-4-3-1-4-1}
Suponiendo que se consideran los homomorfismos como módulos a derecha
de $R$, podemos llevar cada homormofismo $\lambda$ a $\lambda(e)e$ y cada elemento $fre$
al homomorfismo $\psi(x) = (fre)x$.

Comprobamos que esto da una biyección para elemento cualquiera $fre \in fRe$
y $\lambda \in Hom(eR,fR)$ comprobando que la composición es la identidad:

\[
fre \mapsto \psi_{fre} \mapsto \psi_{fre}(e)e = freee = fre
\]
\[
\lambda(ex) \mapsto \lambda(e)e \mapsto \lambda(e)e(ex) = \lambda(ex)
\]

Donde hemos usado en el último paso que $\lambda$ es homomorfismo de R-módulos
a derecha. Que esto preserva la suma es trivial.

\item Isomorfismo de anillos
\label{sec-7-4-3-1-4-2}
En este caso tenemos un isomorfismo de grupos abelianos dado por el
caso anterior. Además, es operación multiplicativa al tenerse:

\[
(\psi\circ\varphi)(e)e = \psi(e)\varphi(e)e
\]

Por ser homomorfismos de módulos a derecha. Y es unital por tenerse:

\[
id(e)e = e
\]
\end{itemize}

\item Punto 5
\label{sec-7-4-3-1-5}
\begin{itemize}
\item Primer isomorfismo
\label{sec-7-4-3-1-5-1}
Trivialmente $\beta$ es inyectivo porque $\beta(r)$ aplica la unidad en $r$.
Que es sobreyectivo es trivial porque cada función está determinada
por dónde lleva la unidad. Por ser homomorfismo de R-módulos:

\[
\varphi(r) = \varphi(1)r
\]

\item Segundo isomorfismo
\label{sec-7-4-3-1-5-2}
Nótese que dada una $\varphi \in End_R(eR\oplus fR)$, podemos descomponer su aplicación
a cualquier elemento como:

\[
\varphi(er+fr) = \varphi(er)+\varphi(fr) = e\varphi(er)+f\varphi(er)+
e\varphi(fr)+f\varphi(fr)
\]

Por lo que queda determinada por dos endomorfismos entre $eR$ y $fR$ y
dos homomorfismos de $eR$ a $fR$ y de $fR$ a $eR$; y se puede escribir como:

\[
\varphi(ex+fy) = \begin{pmatrix}f_1&f_2\\f_3&f_4\end{pmatrix}\begin{pmatrix}ex\\fy\end{pmatrix}
\]

Con los isomorfismos anteriores tenemos lo buscado.

\item Isomorfismo de anillos
\label{sec-7-4-3-1-5-3}
Notamos trivialmente que el isomorfismo así determinado es $\lambda$.
Dado $r$, podemos ver que se divide como:

\[
rx = erex + erfx + frex + frfx
\]

Donde cada elemento pertenece al buscado.
\end{itemize}
\end{itemize}

\item Ejercicio 1.6.
\label{sec-7-4-3-2}
\begin{statement}
Sea $R$ un anillo, $0\neq e \in R$ un elemento idempotente, y $f = 1 - e$. Para cada
R-módulo derecha $M$ se define $Me = \{me \mid m \in M\}$, y $Mf = \{mf \mid m \in M\}$.

\begin{enumerate}
\item Prueba que $Me$ es un $eRe$ módulo derecha y $Mf$ un $fRf$ módulo derecha.
\item Prueba que $Me \times Mf$ es un $\begin{pmatrix}eRe&eRf\\fRe&fRf\end{pmatrix}$ módulo derecha con estructura
dada por,

$\backslash$[
(m$_{\text{1e}}$, m$_{\text{2f}}$)
\begin{pmatrix}em$_{\text{11}}$e\&em$_{\text{12}}$f$\backslash$\fm$_{\text{21}}$e\&fm$_{\text{22}}$ff\end{pmatrix} =
(m$_{\text{1em}}$$_{\text{11}}$e + m$_{\text{2fm}}$$_{\text{21}}$e, m$_{\text{1em}}$$_{\text{12}}$f + m$_{\text{2fm}}$$_{\text{22}}$f)
$\backslash$]

\item Prueba que $h : M \longrightarrow Me \times Mf$, definido $h(m) = (me,mf)$, es un isomorfismo
de R-módulos derecha, donde la estructura de $Me \times Mf$ está dada vía $\lambda$.
Observa que $Me$ y $Mf$ son subgrupos de $M$, pero no necesariamente submódulos.
\end{enumerate}

$\quad$
\end{statement}

\begin{itemize}
\item Punto 1
\label{sec-7-4-3-2-1}
Siendo $me \in Me$, tenemos que $me(ere) = (mer)e \in Me$, donde usamos que
$M$ es módulo a derecha. De la misma forma se cumple para $f$, que es
idempotente.

\item Punto 2
\label{sec-7-4-3-2-2}
Simplemente tenemos que comprobar que la aplicación de multiplicar por
la matriz es lineal en $(m_1,m_2)$, y además, que los elementos vuelven
a estar en $Me \times Mf$ por escribirse como:

\begin{itemize}
\item $(m_1em_{11})e + (m_2fm_{21})e$
\item $(m_1em_{12})f + (m_2fm_{22})f$
\end{itemize}

Usando de nuevo que $M$ es módulo a derecha.

\item Punto 3
\label{sec-7-4-3-2-3}
Tenemos que cada elemento se escribe de forma única como $m = me+mf$.
Si tuviéramos otra suma $m = ae + bf$, se tendría $me=ae$ y $mf=bf$ al
multiplicar por cada uno de los idempotentes.

Tenemos por tanto una biyección, que además es lineal y preserva la
multiplicación por la derecha:

\[
(mre,mrf) = (me,mf)\begin{pmatrix}ere&erf\\fre&frf\end{pmatrix}
\]

Observamos que $Me$ y $Mf$ son cerrados para la suma. Pero no tienen por
qué ser cerrados como módulo. Nótese que puede darse el caso de que
$mer \notin Me$, como ocurre en las matrices, donde hay idempotentes no
centrales:

\[ e\begin{pmatrix}
a & b \\ c & d
\end{pmatrix} = \begin{pmatrix}
1 & 0 \\ 0 & 0
\end{pmatrix} \begin{pmatrix}
a & b \\ c & d
\end{pmatrix} = \begin{pmatrix}
a & b \\ 0 & 0
\end{pmatrix}\]

Que no puede pertenecer al $Me$ porque cambia al multiplicarla a la derecha
por $e$.
\end{itemize}
\end{itemize}
\subsubsection*{Semana 4}
\label{sec-7-4-4}
\begin{itemize}
\item Ejercicio 1.7.
\label{sec-7-4-4-1}
\begin{statement}
Si $K$ es un cuerpo, se considera el anillo:

\[
R = \begin{pmatrix}
K & K \\ 0 & K
\end{pmatrix}
\]

\begin{enumerate}
\item Estudia los ideales derecha de $R$.
\item Estudia los ideales izquierda de $R$.
\item Estudia los ideales biláteros de $R$.
\end{enumerate}

Ver también ejercicios anteriores.
\end{statement}

\begin{itemize}
\item Ideales derecha
\label{sec-7-4-4-1-1}
La multiplicación por un elemento del ideal sería:

\[\begin{pmatrix}
a & b \\ 0 & d
\end{pmatrix}\begin{pmatrix}
k_1 & k_2 \\ 0 & k_3
\end{pmatrix}  = \begin{pmatrix}
k_1a & k_2a+k_3b \\ 0 & k_3d
\end{pmatrix}\]

Estudiando cada combinación de $a,b,d$ nulos o no nulos, se obtienen
los ideales siguientes:

\begin{itemize}
\item El ideal total, $\begin{pmatrix}K & K \\ 0 & K\end{pmatrix}$.

\item Suponiendo $a=0$, $\langle\begin{pmatrix}0 & k \\ 0 & 1\end{pmatrix}\rangle$.

\item Suponiendo $a=0,d=0$, $\begin{pmatrix}0 & K \\ 0 & 0\end{pmatrix}$.

\item Suponiendo $a=0,b=0$, $\begin{pmatrix}0 & 0 \\ 0 & K\end{pmatrix}$.

\item Suponiendo $d=0$, $\begin{pmatrix}K & K \\ 0 & 0\end{pmatrix}$.

\item El ideal trivial, $\begin{pmatrix}0 & 0 \\ 0 & 0\end{pmatrix}$.
\end{itemize}

\item Ideales izquierda
\label{sec-7-4-4-1-2}
La multiplicación por un elemento del ideal es:

\[\begin{pmatrix}
k_1 & k_2 \\ 0 & k_3
\end{pmatrix} \begin{pmatrix}
a & b \\ 0 & d
\end{pmatrix} = \begin{pmatrix}
k_1a & k_1b+k_2d \\ 0 & k_3d
\end{pmatrix}\]

Estudiando cada combinación de $a,b,d$ nulos o no nulos, se obtienen
los ideales siguientes:

\begin{itemize}
\item El ideal total, $\begin{pmatrix}K & K \\ 0 & K\end{pmatrix}$.

\item Suponiendo $d=0$, $\begin{pmatrix}K & K \\ 0 & 0\end{pmatrix}$.

\item Suponiendo $a=0$, $\begin{pmatrix}0 & K \\ 0 & K\end{pmatrix}$.

\item Suponiendo $a=0,d=0$, $\begin{pmatrix}0 & K \\ 0 & 0\end{pmatrix}$.

\item Suponiendo $a=0,b=0$ $\begin{pmatrix}K & 0 \\ 0 & 0\end{pmatrix}$.

\item El ideal trivial, $\begin{pmatrix}0 & 0 \\ 0 & 0\end{pmatrix}$.
\end{itemize}

\item Ideales biláteros
\label{sec-7-4-4-1-3}
Buscamos los ideales que lo son a izquierda y derecha:

$\begin{pmatrix}K & K \\ 0 & K\end{pmatrix}$ $\begin{pmatrix}K & K \\ 0 & 0\end{pmatrix}$ $\begin{pmatrix}0 & K \\ 0 & K\end{pmatrix}$ $\begin{pmatrix}0 & K \\ 0 & 0\end{pmatrix}$ $\begin{pmatrix}0 & 0 \\ 0 & 0\end{pmatrix}$
\end{itemize}

\item Ejercicio 1.8.
\label{sec-7-4-4-2}
\begin{statement}
Estudia los ideales derecha e izquierda del anillo:

\[
R = \begin{pmatrix}\mathbb{Q}&\mathbb{R}\\0&\mathbb{R}\end{pmatrix}
\]

\begin{enumerate}
\item Prueba que $R$ es un anillo artiniano derecha y noetheriano derecha.
\item Prueba que $R$ no es un anillo artiniano izqierda ni noetheriano izquierda.
\end{enumerate}

$\quad$
\end{statement}

\begin{itemize}
\item Ideales derecha
\label{sec-7-4-4-2-1}
Los ideales no triviales a la derecha son los siguientes:

\begin{itemize}
\item $\begin{pmatrix} 0 & 0 \\ 0 & \mathbb{R} \end{pmatrix}$

\item $\langle\begin{pmatrix} 0 & k \\ 0 & 1 \end{pmatrix}\rangle$

\item $\begin{pmatrix} \mathbb{Q} & \mathbb{R} \\ 0 & 0 \end{pmatrix}$

\item $\begin{pmatrix} \mathbb{Q} & \mathbb{R} \\ 0 & \mathbb{R} \end{pmatrix}$
\end{itemize}

Habiendo sólo una cantidad finita de ideales, el anillo será artiniano
y noetheriano.

\item Ideales izquierda
\label{sec-7-4-4-2-2}
Considerando de nuevo los casos y teniendo esta vez en cuenta que el
primer coeficiente está en $\mathbb{Q}$.

\begin{itemize}
\item $\begin{pmatrix} \mathbb{Q} & \mathbb{K} \\ 0 & 0 \end{pmatrix}$, para cualquier $\mathbb{K}$ extensión de cuerpos $\mathbb{Q}\subset \mathbb{K}\subset\mathbb{R}$.

\item $\begin{pmatrix} 0 & \mathbb{R} \\ 0 & \mathbb{R} \end{pmatrix}$

\item $\begin{pmatrix} \mathbb{Q} & \mathbb{R} \\ 0 & \mathbb{R} \end{pmatrix}$
\end{itemize}

Comprobamos que no es artiniano ni noetheriano porque podemos crear
cadenas que rompen la condición de cadena ascendente y descendente.
Sabiendo que los reales tienen dimensión infinita sobre los racionales
como espacio vectorial, creamos ambas cadenas añadiendo y retirando
progresivamente vectores de la base.
\end{itemize}
\end{itemize}

\subsubsection*{Semana 6}
\label{sec-7-4-5}
\begin{statement}
Sea $M$ un grupo abeliano finitamente generado y libre de torsión.
Prueba que $M$ es un grupo libre.
\end{statement}

Si no fuera libre, cada conjunto de generadores
$\left\{ m_1,\dots,m_{n} \right\}$ debería cumplir ecuaciones de la forma

\[n_1m_1+ \dots + n_tm_t = 0,\]

y entre todos los posibles conjuntos de generadores de cardinalidad
mínima y combinaciones, podemos elegir una que minimice $|n_1|+\dots + |n_t|$.
Ahora, si hay una combinación en la que $|n_i|<|n_j|$ para dos $i\neq j$,
podemos usar que

\[
n_im_i + n_jm_j = (n_i-n_j)m_i + n_{j}(m_j-m_i)
\]

para reescribir la relación, teniendo otro sistema de generadores
equivalente $\left\langle m_1,\dots,m_i,m_j,\dots,m_{t} \right\rangle = \left\langle m_1,\dots,m_i,m_j-m_i,\dots,m_t \right\rangle$ que
tiene un $|n_1|+\dots + |n_t|$ menor. Así, en el mínimo, $|n_i| = d$ para cualquier
índice. Pero este $d$ no puede ser mayor que $1$ porque si no se tendría un
elemento de torsión

\[
d \left( \frac{n_1}{d}m_1 + \dots + \frac{n_t}{d}m_t \right) = 0.
\]

Así, hemos llegado a una relación en la que un generador puede ponerse
como suma y diferencia de los otros, 

\[
m_1 = \pm m_2 \pm m_3 \pm \dots \pm m_{t},
\]

contraviniendo la minimalidad del
sistema de generadores

\subsubsection*{Semana 7}
\label{sec-7-4-6}
\begin{statement}
Sea $R$ un anillo con un único ideal izquierda maximal $\mathfrak{a}$.

\begin{enumerate}
\item Prueba que $\mathfrak{a}$ es un ideal bilátero.
\item Prueba que $\mathfrak{a}$ es el único ideal derecha maximal.
\item Prueba que $R/\mathfrak{a} = {\cal U}(R) = R^{\times}$, el conjunto de los elementos 
invertibles de $R$.
\end{enumerate}

Un anillo con un único ideal derecha maximal se llama \textbf{anillo local}.
\end{statement}

Nótese que por Teorema de Krull, todo ideal propio (izquierda,
derecha, bilátero) está contenido en un ideal maximal (izquierda,
derecha, bilátero). Todo elemento no unidad está contenido en un
ideal maximal (izquierda, derecha, bilátero).

\begin{itemize}
\item Primer punto
\label{sec-7-4-6-1}
Si $x \in \mathfrak{a}$, sabemos que no puede ser unidad; así, $xy$ tampoco puede serlo 
para ningún $y \in R$, y como no puede serlo, debe estar contenido en algún
ideal maximal izquierda, que debe ser $\mathfrak{a}$.

\item Segundo punto
\label{sec-7-4-6-2}
Usando el tercer punto, cualquier elemento que estuviera en un ideal
derecha maximal que no estuviera en el único ideal bilátero que existe
debería ser una unidad.

\item Tercer punto
\label{sec-7-4-6-3}
Si hubiera algún $x + \mathfrak{a}$ no invertible, se tendría que $\left\langle x \right\rangle$ generaría un
ideal propio que debería estar contenido en un maximal. Este maximal
debería ser $\mathfrak{a}$, y por tanto $x = 0$.
\end{itemize}

\subsubsection*{Semana 8}
\label{sec-7-4-7}
\begin{statement}
Sea $R$ un anillo y $e \in R$ un elemento idempotente

\begin{enumerate}
\item para cada ideal derecha $\mathfrak{a} \subseteq R$ prueba que $\mathfrak{a} \cap Re = \mathfrak{a}e$.
\item para cada ideal $\mathfrak{A} \subseteq R$ prueba que $\mathfrak{A} \cap Re = \mathfrak{A}e$.
\item $eRe$ es un anillo y $\mathfrak{A} \mapsto e\mathfrak{A}e$ define una aplicación sobreyectiva que respeta
el orden del retículo de los ideales de $R$ en el retículo de los ideales de
$eRe$.
\item prueba que tenemos un funtor $\text{Mod-}R \to \text{Mod-}eRe$, definido $M \mapsto Me$.
\item si $M$ es un $R\text{-módulo}$ derecha simple, prueba que $Me = 0$ ó $Me$ es un
$eRe\text{-módulo}$ derecha simple.
\item ¿se conservan los $R\text{-módulos}$ derecha proyectivos?
\item ¿se conservan los $R\text{-módulos}$ izquierda inyectivos?
\end{enumerate}
\end{statement}

\begin{itemize}
\item Punto 1
\label{sec-7-4-7-1}
Sea $x \in \mathfrak{a} \cap Re$, entonces $x = xe \in \mathfrak{a}e$. Sea $ae \in \mathfrak{a}e$, es trivial que $ae \in \mathfrak{a} \cap Re$.

\item Punto 2
\label{sec-7-4-7-2}
Un ideal es un ideal derecha.

\item Punto 3
\label{sec-7-4-7-3}
Se comprueba que $eRe$ es anillo con unidad $e$. El producto de dos elementos
sigue siendo bilineal con $ere \cdot ese = erese$. Si $S \subseteq R$, es claro que $eSe \subseteq eRe$.
Es sobreyectiva porque si $I$ es $eRe\text{-ideal}$, podemos comprobar que $RIR$ es un
$R\text{-ideal}$ y que $eRIRe = eReIeRe = I$.

\item Punto 4
\label{sec-7-4-7-4}
El funtor llevará $f \colon M \to N$ a $\widetilde f \colon Me \to Ne$ definida como

\[\widetilde f(me) = f(m)e.\]

Es funtor por cumplir $\widetilde g \circ \widetilde f (me) = (g \circ f(m))e$.

\item Punto 5
\label{sec-7-4-7-5}
$Me$ sería un submódulo, así que podría ser $Me = 0$ o $Me = M$.
En el segundo caso sería un $eRe\text{-módulo}$ por ser un $R\text{-módulo}$,
y en ese caso, como $M = Me$, se tendría que si hubiera un
submódulo $A$ de $M$ como $eRe\text{-módulo}$, sería de $M$ como $R\text{-módulo}$
por tenerse $AR = ((Ae)R)e = A(eRe) = A$.

\item Punto 6
\label{sec-7-4-7-6}
Si tenemos $P \oplus H \cong R^{(I)}$, multiplicando, $Pe \oplus He \cong (Re)^{(I)}$. Pero
sabemos que $Re \cong eRe$ como $eRe\text{-módulo}$.

\item Punto 7
\label{sec-7-4-7-7}
Si $Q$ es un submódulo izquierda inyectivo, para cualquier $R\text{-módulo}$ $M$
con $Q \leq M$ existe un $Q \leq K$ tal que $K \oplus Q = M$, como producto directo
interno.

Sea ahora un $eRe\text{-módulo}$ $N$ tal que $Qe \leq N$. Tenemos que $Ne = N$ por
ser $e$ unidad del anillo. Como $Q$ es inyectivo, existe un $K$ tal que
$Q \cap K = \left\{ 0 \right\}$ y $Q + K = Q + N$. Si multiplicamos por $e$ tenemos

\[
Qe + Ke = Qe + N = N.
\]

De aquí se tiene que $Ke \cap N = Ke$. Entonces, $Ke \subset K \cap N$ y dado $l \in K \cap N$,
se tiene que como $l \in N$, $le=l$, luego $l \in Ke$. Así, $Ke = K \cap N$, tenemos

\[
Qe + K \cap N = N,
\]

y como $Q \cap K = \left\{ 0 \right\}$, se tiene $Qe \cap K = \left\{ 0 \right\}$ y por tanto $Qe \cap (K\cap N) = \left\{ 0 \right\}$.
\end{itemize}

\subsubsection*{Semana 9}
\label{sec-7-4-8}
\begin{statement}
Sea $R$ un anillo y $M$ un $R\text{-módulo}$ derecha. Se considera el anillo $S=M_n(R)$ y
el grupo abeliano $M^n$.

\begin{enumerate}
\item Prueba que $M^n$ es un $S\text{-módulo}$ derecha con acción dada por
$(m_i)_i(a_{ij})_{ij} = \left( \sum_i m_ia_{ij} \right)_j$.
\item Prueba que $M \mapsto M^n$, extendiendo para homomorfismos en la forma obvia,
define un functor $F \colon \text{Mod-}R \to \text{Mod-}S$.
\item Se considera el idempotente $e_{11} \in M_n(R)$, la matriz que tiene $1$ en el
lugar $(1,1)$, y $0$ en el resto. Observa que $e_{11}Se_{11} \cong R$. Tenemos entonces un
funtor $G \colon \text{Mod-}S \to \text{Mod-}e_{11}Se_{11} = \text{Mod-}R$.
\item Prueba que para cada $R\text{-módulo}$ derecha $M$ se tiene un isomorfismo
$\theta_M\colon M \cong GF(M)$, y que si $f\colon M_1\to M_2$ es un homomorfismo de $R\text{-módulos}$,
entonces tenemos un cuadrado conmutativo de homomorfismos de $R\text{-módulos}$.

\[\begin{tikzcd}
    M_1 \rar{f} \dar[swap]{\theta_{M_1}} & M_2 \dar{\theta_{M_2}} \\
    GF(M_1) \rar{GF(f)} & GF(M_2) 
     \end{tikzcd}\]

\item Prueba que para cada $S\text{-módulo}$ derecha $N$ se tiene un isomorfismo
$\nu_N\colon N \cong FG(N)$, y que si $g\colon N_1 \to N_2$ es un homomorfismo de $S\text{-módulos}$,
entonces tenemos un cuadrado conmutativo de homomorfismos de $S\text{-módulos}$.

\[\begin{tikzcd}
    N_{1} \rar{g} \dar[swap]{\nu_{N_1}} & N_{2} \dar{\nu_{N_2}} \\
    FG(N_{1}) \rar{FG(g)} & FG(N_{2})
    \end{tikzcd}\]

\item Prueba que si $M$ es un $R\text{-módulo}$ derecha simple (resp. proyectivo,
inyectivo), también $F(M)$ lo es.
\end{enumerate}
\end{statement}

\begin{itemize}
\item Punto 1
\label{sec-7-4-8-1}
Comprobaremos que cumple la definición de módulo, es decir,

\begin{itemize}
\item hay una \textbf{identidad} dada por $(m_i)_i(\delta_{ij})_{ij} = (m_{j})_{j}$.
\item el \textbf{producto} es bilineal
$(m_i+n_i)_i(a_{ij})_{ij} = (\sum_i (m_i+n_i)a_{ij})_j = (\sum m_ia_{ij} + \sum n_ia_{ij})_j$ y
$(m_i)_i(a_{ij}+b_{ij})_{ij} = (\sum m_i(a_{ij}+b_{ij}))_j = \sum m_ia_{ij} + \sum m_ib_{ij}$.
\item y es \textbf{asociativa} con el producto de matrices usual
$((m_i)_ia_{ij})(b_{ij}) = (\sum_j (\sum_i m_ia_{ij})b_{jk})_{k} = (\sum_i m_i\sum_{j}a_{ij}b_{jk})_k$.
\end{itemize}

\item Punto 2
\label{sec-7-4-8-2}
Si los extendemos de forma obvia aplicando el homomorfismo a cada una de las
entradas de la matriz, es obvio que se conserva la composición de funciones
como

\[
g(f((m_{i})_i)) = (gf(m_i))_i,
\]

y que la indentidad se preserva por la extensión.

\item Punto 3
\label{sec-7-4-8-3}
Notamos que podemos llevar cada matriz a su única entrada $e_{11}(r_{ij})e_{11} \mapsto r_{11}$.
La suma es por componentes y por tanto se respeta por la aplicación; el
producto de matrices de una entrada coincide con el producto del anillo.

\item Punto 4
\label{sec-7-4-8-4}
Nótese que $F(M) = M^n$ y que $GF(M) = (M\ 0\ 0\ \dots )$, donde además hay un
isomorfismo $e_{ii}Se_{11} \cong R$. El isomorfismo de módulos lleva $m$ en $(m\ 0\ 0\ \dots)$,
y se comprueba trivialmente que la multiplicación funciona de la misma
manera.

Dado un homomorfismo de módulos, tenemos que $GF(f)$ aplicará el homomorfismo
sobre el único elemento llevando $GF(f)(m\ 0\ 0\ \dots) = (f(m)\ 0\ 0\ \dots)$.

\item Punto 5
\label{sec-7-4-8-5}
Tenemos por ser idempotente que $G(Ne_{11}) = G(N)$, pero 

\[FG(N) \cong FG(Ne_{11}) \cong Ne_{11} \oplus Ne_{22} \oplus \dots \cong N\]

por ser $Ne_{11}\cong Ne_{22}$ como $R\text{-módulos}$ y ser $\left\{ e_{1},\dots,e_{n} \right\}$ un conjunto de
idempotentes centrales.

Una función $g\colon N_1 \to N_2$ está unívocamente determinada por cómo actúa
sobre cada sumando directo, por lo que conmuta su actuación antes y
después de aplicarla explícitamente sobre cada sumando directo.

\item Punto 6
\label{sec-7-4-8-6}
Los dos puntos anteriores han definido dos isomorfismos naturales
que constituyen una equivalencia de categorías.
\end{itemize}

\subsubsection*{Semana 10}
\label{sec-7-4-9}
\begin{statement}
Se considera la categoría de grupos abelianos; en este caso $R = \mathbb{Z}$.

\begin{enumerate}
\item Prueba que $\mathbb{Z}$ es un grupo abeliano uniforme. Determina todos los grupos
cíclicos uniformes.
\item Prueba que el grupo $\mathbb{Z}_{p^{\infty}}$ es un grupo uniforme y no es un grupo cíclico.
Se consideran $\mathbb{Q}$ y $\mathbb{R}$; ¿es alguno uniforme?
\item Determina todos los grupos abelianos inyectivos indescomponibles.
\item Si $M$ es un grupo abeliano finitamente generado sabemos que 
$M \cong \left( \bigoplus^t_{i=1} \mathbb{Z}_{p^{n_i}} \right) \oplus \mathbb{Z}^n$, para $n,n_1,\dots,n_t \in \mathbb{N}$. ¿Cuál es la descomposición
de $E(M)$ como suma de inyectivos indescomponibles?
\end{enumerate}
\end{statement}

\begin{itemize}
\item Punto 1
\label{sec-7-4-9-1}
Dados dos submódulos de $\mathbb{Z}$, que estarán generados por dos enteros, podemos
comprobar que se intersecarán en su mínimo común múltiplo.

\item Punto 2
\label{sec-7-4-9-2}
Para comprobar que el grupo $\mathbb{Z}_{p^{\infty}}$ es uniforme tomamos un módulo no 
nulo que tenga al menos un elemento $a/p^n$ y otro con $b/p^m$; 
para $a,b$ coprimos con $p$.  Existirán $x,y,p$ tales que
$xp a/p^n = 1/p^{n+p} = y b/p^{m}$, por lo que será uniforme.

Comprobamos que $\mathbb{R}$ no es uniforme porque tiene, por ejemplo $(\pi) \cap (1) = 0$.
Sin embargo $\mathbb{Q}$ sí lo es porque si tenemos dos módulos y cada uno contiene
al menos un elemento no nulo $a/b$ y $c/d$; tenemos que $cb \cdot a/b = ad \cdot c/d$.

\item Punto 3
\label{sec-7-4-9-3}
Los inyectivos indescomponibles están en correspondencia con los ideales
primos. Tenemos para $p$ primo que $E(\mathbb{Z}_p) = \mathbb{Z}_{p^{\infty}}$, ya que es extensión esencial
y es inyectivo. Para el ideal primo $\{0\}$ tenemos a su vez que $E(\mathbb{Z}) = \mathbb{Q}$, ya
que es inyectivo y uniforme.

\item Punto 4
\label{sec-7-4-9-4}
Como $\mathbb{Z}$ es noetheriano, tenemos $\bigoplus E(M_i) = E \left( \bigoplus M_i \right)$, así que la suma
debe ser

\[
E(M) = \left( \bigoplus_{i=1}^t \mathbb{Z}_{p^{\infty}} \right) \oplus \mathbb{Q}^n
\]

considerando los sumandos con exponente no nulo.
\end{itemize}

\subsection*{Trabajos}
\label{sec-7-5}
\subsubsection*{Funtores adjuntos}
\label{sec-7-5-1}
\begin{itemize}
\item Transformaciones naturales
\label{sec-7-5-1-1}
\begin{definition}
Dados dos funtores $S,T : A \to B$, una \textbf{transformación natural} $\tau : S \Longrightarrow T$ 
es una función asignando a cada objeto $a \in A$ un morfismo $Sa \to Ta$ y 
cumpliendo el siguiente diagrama conmutativo:

\[\begin{tikzcd}
a \dar{f} & & Sa \rar{\tau_a}\dar{Sf} & Ta \dar{Tf} \\
a' & & Sa' \rar{\tau_{a'}} & Ta'
\end{tikzcd}\]

En este caso, decimos que $\tau_a$ es \emph{natural en} $a$.
\end{definition}

\begin{definition}
Llamamos \textbf{isomorfismo natural} a la transformación natural en la que
cada componente $\tau_a$ tiene una inversa. Podemos definir una transformación
natural inversa $\tau^{-1}$ que tiene por componentes a cada una de las inversas.
\end{definition}

\begin{itemize}
\item Composición vertical de transformaciones naturales
\label{sec-7-5-1-1-1}
\begin{definition}
Dados funtores $R,S,T : {\cal A} \to {\cal B}$ y transformaciones naturales $\tau : S \Longrightarrow T$
y $\sigma : R \Longrightarrow S$, podemos componerlas componente a componente para formar
una \textbf{transformación naturalcomposición vertical} $\tau \circ \sigma$.

\[\begin{tikzcd}
Rc \rar{Rf}\dar{\sigma_c}\arrow[dd,bend right=90] &
Rc' \dar{\sigma_{c'}} \arrow[dd,bend left=90] \\
Sc \rar{Sf} \dar{\tau_c} & Sc' \dar{\tau_{c'}} \\
Tc \rar{Tf} & Tc' 
\end{tikzcd}
\]
\end{definition}

Nótese que la naturalidad se preserva, ya que si los dos cuadrados
pequeños son conmutativos, conmuta todo el diagrama.

\item Composición horizontal de transformaciones naturales
\label{sec-7-5-1-1-2}
\begin{definition}
Dados funtores $S,T : {\cal A} \longrightarrow {\cal B}$ y $S',T' : {\cal B} \longrightarrow {\cal C}$ y dadas transformaciones
naturales $\tau : S \Longrightarrow T$ y $\tau' : S' \Longrightarrow T'$, podemos crear una transformación
natural entre los funtores compuestos, $\tau' \ast \tau$:

\[\begin{tikzcd}
S'Sx \arrow{rr}{(\tau' \ast \tau)_x} \dar &&
T'Tx \dar \\
S'Sy \arrow{rr}{(\tau' \ast \tau)_y} &&
T'Ty
\end{tikzcd}\]

Cada componente se crea aprovechando la siguiente igualdad:

\[
(\tau' \ast \tau) = T'\tau \circ \tau' = \tau' \circ S'\tau
\]
\end{definition}

Y puede comprobarse que constituye una transformación natural.

\item Categoría de los funtores
\label{sec-7-5-1-1-3}
\begin{definition}
Dadas ${\cal A},{\cal B}$ categorías, los funtores entre ellas forman una \textbf{categoría de funtores} 
que llamaremos $Funct({\cal A},{\cal B})$ y que tiene como morfismos a las transformaciones 
naturales con la composición vertical:

\[
Nat(S,T) = \{ \tau \mid \tau : S \Longrightarrow T \}
\]
\end{definition}

Nótese que la composición es asociativa y que consta de una identidad
en la transformación natural de cada funtor consigo mismo que tiene
como componentes identidades en cada objeto.
\end{itemize}

\item Definición de funtores adjuntos por naturalidad
\label{sec-7-5-1-2}
\begin{definition}
Una \textbf{adjunción} entre categorías ${\cal A}$ y ${\cal B}$ es un par de funtores $F:{\cal A} \to {\cal B}$ y
$G: {\cal B} \to {\cal A}$ con una familia de isomorfismos $\varphi_{a,b} : Hom(Fa,b) \cong Hom(a,Gb)$
que determinan transformaciones naturales en ambas componentes.
\end{definition}

Notamos al par de funtores adjuntos como $F \dashv G$. Llamamos a $F$ adjunto
izquierdo y a $G$ adjunto derecho.

\begin{itemize}
\item Condiciones de naturalidad
\label{sec-7-5-1-2-1}
Las condiciones de naturalidad de esa familia de isomorfismos equivalen
a que los siguientes diagramas conmuten:

$\backslash$[\begin{tabular}{cc} \begin{tikzcd}
Hom(Fa,b) \rar\{$\varphi$$_{\text{a,b}}$\} \dar[swap]{f_\ast} \& 
Hom(a,Gb) \dar{(Gf)_\ast} \\
Hom(Fa,b') \rar\{$\varphi$$_{\text{a,b'}}$\} \&
Hom(a,Gb')
\end{tikzcd} \&
\begin{tikzcd}
Hom(Fa,b) \rar\{$\varphi$$_{\text{a,b}}$\} \dar[swap]{(Fg)^\ast} \& 
Hom(a,Gb) \dar{g^\ast} \\
Hom(Fa',b) \rar\{$\varphi$$_{\text{a,b'}}$\} \&
Hom(a',Gb)
\end{tikzcd} \end{tabular}$\backslash$]

Nótese que cada uno de ellos expresa la naturalidad entre los dos bifuntores
cuando se fija un argumento. Es decir, hay dos isomorfismos naturales

\begin{enumerate}
\item $Hom(F-,b) \Longrightarrow Hom(-,Gb)$.
\item $Hom(Fa,-) \Longrightarrow Hom(a,G-)$.
\end{enumerate}
\end{itemize}

\item Definición por unidad y counidad
\label{sec-7-5-1-3}
\begin{definition}
Una \textbf{adjunción} entre categorías ${\cal A}$ y ${\cal B}$ es un par de funtores $F:{\cal A} \to {\cal B}$ y
$G: {\cal B} \to {\cal A}$ con dos transformaciones naturales:

\begin{itemize}
\item La \textbf{unidad}:   $\eta : 1_{\cal A} \Longrightarrow GF$
\item La \textbf{counidad}: $\epsilon: FG \Longrightarrow 1_{\cal B}$
\end{itemize}

Cumpliendo que las composiciones siguientes dan la identidad:

\begin{itemize}
\item $F \overset{F \eta} \Longrightarrow FGF \overset{\varepsilon F}\Longrightarrow F$
\item $G \overset{\eta G} \Longrightarrow GFG \overset{G \varepsilon}\Longrightarrow G$
\end{itemize}
\end{definition}

Demostraremos que esta definición es equivalente a la anterior.

\begin{itemize}
\item Equivalencia de definiciones: desde familia de isomorfismos a unidades
\label{sec-7-5-1-3-1}
\begin{theorem}
Dada una adjunción en términos de una familia de isomorfismos, podemos
construir una adjunción en términos de unidad y counidad.
\end{theorem}

\begin{proof}
\emph{Paso 1: Construcción de la unidad y la counidad.}

Supongamos que tenemos la familia de transformaciones naturales
$\varphi_{a,b} : Hom(Fa,b) \to Hom(a,Gb)$. Particularizaremos los cuadrados de
naturalidad en los dos casos $b = Fa$ y $a = Gb$ para crear la unidad y
la counidad.

\[\begin{tikzcd}
Hom(Fa,Fa) \arrow{d}[swap]{(Ff)^\ast} \arrow{r}{\varphi} & 
Hom(a,GFa) \arrow{d}{(f)^\ast} \\
Hom(Fa', Fa) \arrow{r}{\varphi} &
Hom(a',GFa)
\end{tikzcd}\]

Si tomamos la identidad $1_{Fa}$ y llamamos $\eta_a = \varphi(1_{Fa})$, tenemos que
$\eta \circ f = \varphi(Ff)$ por conmutatividad.

Si damos la vuelta al isomorfismo $\varphi$ para tomar $\varphi^{-1}$, llamarlo de la
misma forma y repetir el mismo proceso:

\[\begin{tikzcd}
Hom(FGb,b) \arrow{d}[swap]{(Ff)^\ast} & 
Hom(Gb,Gb) \arrow{d}{(f)^\ast} \lar[swap]{\varphi} \\
Hom(FGb', b) &
Hom(Gb',Gb) \lar{\varphi}
\end{tikzcd}\]

Nótese que aquí usamos $\varphi$ para notar un isomorfismo y su inversa;
dependerá sólo del contexto determinar cuál estamos usando.
Si tomamos la identidad $1_{Gb}$ y llamamos $\varepsilon_b = \varphi(1_{Gb})$, tenemos que
$\varepsilon \circ Ff = \varphi(f)$.

Aplicamos el mismo proceso al segundo cuadrado natural.

\[\begin{tikzcd}
Hom(Fa,Fa) \arrow{d}[swap]{g_\ast} \arrow{r}{\varphi} & 
Hom(a,GFa) \arrow{d}{(Gg)_\ast} \\
Hom(Fa, Fa') \arrow{r}{\varphi} &
Hom(a,GFa')
\end{tikzcd}\]

Y volvemos a tomar la identidad para tener $\varphi(g) = Gg \circ \eta$. Volviendo a
dar la vuelta a los isomorfismos llegamos a:

\[\begin{tikzcd}
Hom(FGb,b) \arrow{d}[swap]{(g)_\ast} & 
Hom(Gb,Gb) \arrow{d}{(Gg)_\ast} \lar[swap]{\varphi} \\
Hom(FGb,b') &
Hom(Gb,Gb') \lar{\varphi}
\end{tikzcd}\]

Que nos da, tomando la identidad, $\varphi(Gg) = g \circ \varepsilon$.

\emph{Paso 2: Naturalidad de la unidad y la counidad.}

Una vez tenemos definidas la unidad y la counidad, podemos comprobar
su naturalidad desde las ecuaciones que hemos obtenido:

\[\begin{aligned}
\eta     \circ f        &= \varphi(Ff) \\
g        \circ \epsilon &= \varphi(Gg) \\
\epsilon \circ Ff       &= \varphi(f) \\
Gg       \circ \eta     &= \varphi(g) \\
\end{aligned}\]

Y la naturalidad de $\eta$ y $\varepsilon$ se deduce desde ahí por la conmutatividad de los
siguientes diagramas, con $\eta \circ f = GFf \circ \eta$ y $g \circ\varepsilon = \varepsilon\circ FGg$:

\[\begin{tabular}{cc}\begin{tikzcd}
GFa  \arrow{r}{GFf} & 
GFb \\
a \arrow{u}[swap]{\eta_X} \arrow{r}[swap]{f} & 
b \arrow{u}{\eta_Y}
\end{tikzcd} & \begin{tikzcd}
FGX \arrow{d}[swap]{\epsilon_X} \arrow{r}{FGg} & FGY \arrow{d}{\epsilon_Y}\\
X \arrow{r}[swap]{g} & Y
\end{tikzcd}\end{tabular}\]

\emph{Paso 3: Comprobar la condición de composición.}

Por último tenemos los dos triángulos siguientes, cuya conmutatividad
equivale a la condición de que la composición debía ser la identidad.

\[\begin{tabular}{cc} \begin{tikzcd}
F \arrow{r}{F \eta_X} \arrow{dr}{id} & FGF \arrow{d}{\epsilon_{FX}} \\
 & F
\end{tikzcd} & \begin{tikzcd}
G \arrow{r}{\eta_{GX}} \arrow{dr}{id} & GFG \arrow{d}{G\epsilon_X} \\
 & G
\end{tikzcd}\end{tabular}
\]

Para ello usamos las identidades anteriores comprobando que:

\[\begin{aligned}
\epsilon \circ F\eta &= \varphi(\eta) = 1 \\
G\epsilon \circ \eta &= \varphi(\epsilon) = 1
\end{aligned}\]

$\quad$
\end{proof}

\item Equivalencia de definiciones: desde unidades a familia de isomorfismos
\label{sec-7-5-1-3-2}
\begin{theorem}
Dada una adjunción en términos de unidad y counidad, podemos construir
una adjunción en términos de familia de isomorfismos.
\end{theorem}

\begin{proof}
\emph{Paso 1: Definición de los isomorfismos.}

Por las condiciones sobre la composición de unidad y counidad,
tenemos:

\[\begin{aligned}
\varepsilon \circ F\eta &= 1 \\
G\varepsilon \circ \eta &= 1
\end{aligned}\]

Y por las condiciones de naturalidad de ambas transformaciones, se
tiene:

\[\begin{aligned}
\eta \circ f &= GFf \circ \eta \\
g \circ \varepsilon &= \varepsilon \circ FGg
\end{aligned}\]

Definimos el isomorfismo y su inversa, que seguimos notando igual,
como:

\[\begin{aligned}
\varphi(f) &= \varepsilon \circ Ff \\
\varphi(g) &= Gg \circ \eta
\end{aligned}\]

Se comprueba trivialmente que es isomorfismo por las condiciones
anteriores. Tenemos así las igualdades:

\[\begin{aligned}
\eta     \circ f        &= \varphi(Ff) \\
g        \circ \epsilon &= \varphi(Gg) \\
\epsilon \circ Ff       &= \varphi(f) \\
Gg       \circ \eta     &= \varphi(g) \\
\end{aligned}\]

\emph{Paso 2: Naturalidad de los isomorfismos.}

Demostraremos que el isomorfismo es natural en cada una de sus
componentes. La naturalidad aquí se deduce de que la definición
de $\varphi$ nos da las siguientes ecuaciones para cualesquiera $f,g,h$:

\[\begin{aligned}
\varphi(f \circ h)   &= Gf \circ \varphi(h) \\
\varphi(h \circ Fg) &= g \circ \varphi(h)
\end{aligned}\]

Que nos dan la naturalidad de $\varphi$ en ambas componentes.
\end{proof}
\end{itemize}

\item Unicidad del adjunto
\label{sec-7-5-1-4}
\begin{theorem}
El adjunto es esencialmente único, es decir,
si tenemos funtores $F : {\cal A} \to {\cal B}$ y $G,G' : {\cal B} \to {\cal A}$ y son ambos adjuntos por la
derecha al primero, $F \dashv G, F \dashv G'$; entonces existe un isomorfismo natural
$\tau : G \cong G'$.
\end{theorem}

\begin{proof}
\emph{Paso 1: Definiendo el isomorfismo natural.}

Por ser ambas adjunciones, tenemos un isomorfismo natural en ambas
variables $X,Y$ dado por $\varphi : Hom(X,GY) \cong Hom(X,G'Y)$, ya que
ambos eran isomorfos a $Hom(FX,Y)$.

Tomamos para cada $A$, la componente de nuestro isomorfismo natural
en $A$ como $\tau_A = \varphi_A(id_{GA})$.

\emph{Paso 2: Probando la naturalidad.}

Aplicamos dos veces la naturalidad de $\varphi$ para tener, dado un
$f : A \to B$:

$\backslash$[\begin{tabular}{cc}
\begin{tikzcd}
Hom(GA,GA)\rar{\varphi} \dar[swap]{f^\ast} \& Hom(GA,G'A) \dar{(Gf)^\ast} \\
Hom(GA,GB)\rar{\varphi} \& Hom(GA,G'B)
\end{tikzcd} \& \begin{tikzcd}
Hom(GB,GB)\rar{\varphi} \dar[swap]{(Gf)_\ast} \& Hom(GB,GB') \dar{(Gf)_\ast} \\
Hom(GA,GB)\rar{\varphi} \& Hom(GA,G'B)
\end{tikzcd}\end{tabular}$\backslash$]

Obtenemos, tomando de la identidad en ambos diagramas, que $\tau \circ Gf = \varphi(Gf)$
y que $G'f \circ \tau = \varphi(Gf)$. Y uniendo ambas igualdades tenemos la condición de
naturalidad de la transformación $\tau$. Por ser la imagen por un isomorfismo 
natural del isomorfismo identidad, todas sus componentes son isomorfismos.
\end{proof}
\item Continuidad
\label{sec-7-5-1-5}
\begin{theorem}
Todo funtor que es un adjunto derecho (equivalentemente, que tiene un
adjunto izquierdo) es \textbf{continuo}; es decir, preserva límites
categóricos. Por otro lado, todo functor que es un adjunto izquierdo
es \textbf{cocontinuo} y preserva colímites categóricos.
\end{theorem}

\begin{proof}
Sea $a$ el límite de un funtor en la categoría $A$ y sea $G : A \to B$ un
funtor con adjunto a la izquierda $F \dashv G$. Comprobaremos que si existiera
otro cono desde $x$, descompondría de forma única por $Ga$, haciéndolo límite.

$\backslash$[\begin{tabular}{ccc}\begin{tikzcd}
a \dar[d]\dar[d,shift left=1, bend left]\dar[d,shift right=1, bend right] \\
\dots{}
\end{tikzcd} \&
$\Longrightarrow$
\&
\begin{tikzcd}
x
\arrow[in=70, out=290]{dd}
\arrow[bend left, shift left=1]{dd}
\arrow[bend right, shift right=1]{dd} \\
Ga 
\dar[d]\dar[d,shift left=1, bend left]
\dar[d,shift right=1, bend right] \\
G\dots{}
\end{tikzcd}\end{tabular}$\backslash$]

Pero entonces, por la adjunción, por cada $x \to Gi$ tenemos un $Fx \to i$, y estas
aplicaciones generan un cono que conmuta con el diagrama por tenerse

$\backslash$[\begin{tabular}{ccc}\begin{tikzcd}[column sep=0.5em]
\& x \dlar[swap]{\alpha}\drar{\beta} \& \\
Gi \arrow{rr}{Gf} \& \& Gj
\end{tikzcd} \&
$\Longrightarrow$
\&
\begin{tikzcd}[column sep=0.5em]
\& Fx \dlar[swap]\{\overline{\alpha}\}\drar\{\overline{\beta}\} \& \\
i \arrow{rr}{f} \& \& j
\end{tikzcd}\end{tabular}$\backslash$]

y por las condiciones de naturalidad de la transformación
$Hom(F-,-) \cong Hom(-,G-)$ tenemos que

\[
\beta = Gf \circ \varphi(\overline{\alpha}) = 
\varphi(f \circ \overline{\alpha}) = \varphi(\overline{\beta})
.\]

Así, como $a$ es límite, tenemos un único $Hom(Fx,a)$ que hace conmutar a los
diagramas. Como sólo existe uno, sólo existe un $Hom(x,Ga)$, lo que conlleva
que sea $Ga$ efectivamente el límite.

El caso de cocontinuidad se obtiene aplicándolo a la categoría dual.
cite:lane78categories
\end{proof}

\item Ejemplos
\label{sec-7-5-1-6}
\begin{itemize}
\item Módulos libres
\label{sec-7-5-1-6-1}
Un \textbf{funtor de olvido} es aquel que proyecta estructuras en una
categoría de estructuras más generales, "olvidando" en el proceso
parte de su estructura. En nuestro caso particular de R-módulos,
tenemos el funtor de olvido que lleva cada módulo a su conjunto
subyacente y cada homomorfismo a su aplicación de conjuntos:

\[
U : R\mathtt{-Mod} \longrightarrow \mathtt{Set}
\]

Sobre cada conjunto puede generarse un R-módulo libre, y cada
aplicación de conjuntos puede extenderse directamente por linealidad
a todo el módulo libre. Esto nos da el \textbf{funtor de módulo libre}:

\[
F : \mathtt{Set} \longrightarrow R\mathtt{-mod}
\]

Definido como, 

\[F(S) = <S> \qquad F(f)\left(\sum rx\right) = \sum rf(x)\]

Hay una \textbf{adjunción} entre el funtor libre y el funtor de olvido
$F \dashv U$, ya que tenemos la correspondencia natural entre homomorfismos
dada por, para un conjunto $X$ y un R-módulo $M$:

\[
Hom(FX,M) \cong Hom(X,UM)
\]

Que hace corresponder a cada aplicación entre conjuntos su extensión
lineal, que está biunívocamente determinada.

La naturalidad se tiene por tenerse para cada $x \in X$:

\[\begin{aligned}
\varphi(g\circ f)(x) = g(f(x)) =& Ug \circ f(x) \\
\varphi(Ff \circ g)(x) = Ff(g(x)) =& f(\varphi(g)(x))
\end{aligned}\]

\item Otros funtores libres y de olvido
\label{sec-7-5-1-6-2}
De la misma forma que funciona el funtor de olvido entre
módulos y conjuntos, funciona con otras estructuras algebraicas,
como por ejemplo:

\begin{itemize}
\item Grupos a conjuntos.
\item Grupos abelianos a grupos.
\item K-álgebras a K-módulos.
\end{itemize}

\item Funtor diagonal
\label{sec-7-5-1-6-3}
\begin{itemize}
\item Categoría producto
\label{sec-7-5-1-6-3-1}
\begin{definition}
Dada una categoría ${\cal C}$ con productos y coproductos ($\mathtt{Set}$, por ejemplo) 
definimos ${\cal C}\times{\cal C}$ como la categoría que tiene como objetos a pares de 
objetos de ${\cal C}$ y morfismos a pares de morfismos que se componen componente
a componente:

\[
(f,g)\circ(h,i) = (f\circ h, g\circ i)
\]
\end{definition}

En la categoría producto, tenemos un \textbf{funtor diagonal} $\Delta : {\cal C} \to {\cal C}\times{\cal C}$, que 
lleva cada objeto $A$ a $A\times A$ y cada morfismo $f$ a $(f,f)$.

\item Producto como adjunto derecho
\label{sec-7-5-1-6-3-2}
Si definimos el \textbf{funtor producto}, $\times : {\cal C}\times{\cal C} \to {\cal C}$, lleva $(A,B)$ en $A\times B$
y cada par de morfismos $f : A \to C$ y $g : B \to D$, en el morfismo producto
x$f \times g : A\times B \to C \times D$, dado por el único que hace conmutar:

\[\begin{tikzcd}
& A\times B \dlar[swap]{\pi}\drar{\pi}\dar[dashed] & \\
A\dar[swap]{f} & C \times D \dlar[swap]{\pi}\drar{\pi} & B\dar{g} \\
C & & D \\
\end{tikzcd}\]

Este funtor es adjunto derecho al funtor diagonal. Nótese que se
tiene:

\[
Hom(\Delta A, (B,C)) \cong Hom(A, B \times C)
\]

Y utilizamos la propiedad universal del producto para llevar dos
morfismos $A \to B$ y $A \to C$ a un morfismo al producto $A \to B \times C$.
Y puede comprobarse la naturalidad.

\item Coproducto como adjunto izquierdo
\label{sec-7-5-1-6-3-3}
Si definimos el \textbf{funtor coproducto} $\coprod : {\cal C}\times{\cal C} \to {\cal C}$, lleva $(A,B)$ en
$A \coprod B$ y cada par de morfismos $f : A \to C$ y $g : B \to D$, en el morfismo
coproducto, dado por el único que hace conmutar:

\[\begin{tikzcd}
A\dar{f}\drar{i} & & B\dar{g}\dlar[swap]{i} \\
C\drar{i} & A \coprod B \dar[dashed] &D\dlar[swap]{i} \\
& C \coprod D &
\end{tikzcd}\]

Y este funtor es adjunto izquierdo al funtor diagonal. Teniéndose
el isomorfismo siguiente y la naturalidad por la propiedad universal
del coproducto:

\[
Hom\left(A \coprod B, C\right) \cong Hom((A, B), \Delta C)
\]
\end{itemize}

\item Tensor-Hom
\label{sec-7-5-1-6-4}
Existe una adjunción entre los funtores \textbf{tensor y Hom}. cite:kan58adjoint
Si $R,S$ son dos anillos y fijamos un (R;S)-módulo $X$, tenemos los dos funtores

\[\begin{aligned}
F : \mathtt{Mod-}R \longrightarrow \mathtt{Mod-}S
&\qquad&
F(Y) = Y \otimes_R X\\
G : \mathtt{Mod-}S \longrightarrow \mathtt{Mod-}R
&\qquad&
G(Z) = \mathrm{Hom}(X,Z)
\end{aligned}\]

y tenemos el isomorfismo natural

\[
\mathrm{Hom}_{S}(Y \otimes_{R} X, Z) \cong
\mathrm{Hom}_{R}(Y, \mathrm{Hom}_{S}(X,Z))
\]

dado por $\widehat{f}(y)(x) = f(y \otimes x)$.

bibliographystyle:unsrt
bibliography:math.bib
\end{itemize}
\end{itemize}
\subsubsection*{Retículos}
\label{sec-7-5-2}
\subsubsection*{Módulos libres}
\label{sec-7-5-3}

\begin{itemize}
\item Definición de módulo libre
\label{sec-7-5-3-1}
\begin{definition}
Definimos el \textbf{R-módulo libre} cite:aluffi09$_{\text{rings}}$ sobre $A$ como un módulo $F^R(A)$ 
con una inclusión $j : A \to F^R(A)$ como aquel que cumple que para cualquier 
aplicación $f : A \to M$ a un R-módulo, existe un único homomorfismo de 
R-módulos $\varphi:F^R(A) \to M$ que hace conmutar el diagrama

\[\begin{tikzcd}
F^R(A) \rar[dashed]{\exists!\varphi} & M \\
A \uar{j}\urar[swap]{f} &
\end{tikzcd}\]
\end{definition}

Sabemos por ser una propiedad universal que si existe, será único salvo
isomorfías y que $j\colon A \to F^R(A)$ será inyectivo.

\begin{itemize}
\item Construcción
\label{sec-7-5-3-1-1}
\begin{definition}
Dado un conjunto $A$, definimos la \textbf{suma directa indexada} sobre él como
las aplicaciones de soporte finito

\[
N^{\oplus A}
=
\left\{ \alpha\colon A \to N 
\mid 
\alpha(a) \neq 0 \text{ sólo para un número finito de elementos} \right\}
\]

a las que les damos estructura de R-módulo con $(r\alpha)a = r\alpha(a)$.
\end{definition}

Además, existe una inclusión $j\colon A \to R^{\oplus A}$ definida como

\[
j(a)(x) = \left\{\begin{array}{ll} 
1 & \mbox{if } x = a  \\
0 & \mbox{if } x \neq a 
\end{array} 
\right. .
\]

\begin{theorem}
El así definido es el módulo libre sobre $A$. Es decir, $F^R(A) \cong R^{\oplus A}$.
\end{theorem}

\begin{proof}
Nótese que podemos escribir realmente los elementos de esta suma directa
indexada como

\[
\sum_{a \in A} r_aa,
\]

además de forma única y en un número finito de sumandos, uno para cada
elemento en el que la aplicación sea no nula. Esto que nos lleva a que,
una vez definida la imagen de cada elemento $a$, queda definida la imagen
que debe tener $\varphi$ sobre toda el anillo de forma única.
\end{proof}
\end{itemize}

\item Independencia lineal y bases
\label{sec-7-5-3-2}
\begin{definition}
Decimos que un conjunto indexado $i \colon I \to M$ es \textbf{linealmente independiente}
(respectivamente \textbf{sistema generador}) si el homomorfismo natural desde su
módulo libre, $\varphi\colon F^{R}(I) \to M$, haciendo conmutar

\[\begin{tikzcd}
F^{R}(I) \rar{\varphi} & M \\
I \uar{j}\urar[swap]{i} &
\end{tikzcd}\]

es inyectivo (respectivamente sobreyectivo). cite:aluffi09$_{\text{linear}}$
\end{definition}

\begin{definition}
Un conjunto indexado $I \to M$ es una base cuando es linealmente
independiente y genera $M$.
\end{definition}

\begin{lemma}
Un conjunto indexado $B \to M$ es una base si y sólo si el homomorfismo
natural desde su módulo libre es un isomorfismo $R^{\oplus B} \cong M$. Así, un
$R\text{-módulo}$ es libre si y sólo si admite una base.
\end{lemma}

\begin{proof}
Trivial si combinamos las definiciones de linealmente independiente y
sistema generador.
\end{proof}

\item Caso de los espacios vectoriales
\label{sec-7-5-3-3}
\begin{theorem}
Los módulos sobre un cuerpo son necesariamente libres. Podemos
probarlo usando la caracterización anterior por bases. De hecho, dado un
subconjunto de vectores linealmente independientes en un espacio
vectorial, existe una base del espacio conteniéndolos.
\end{theorem}

La noción de dimensión de un espacio vectorial nos permite recuperar
la cardinalidad de la base sobre la que es módulo libre.

\item Clasificación de módulos libres en dominios de integridad
\label{sec-7-5-3-4}
\begin{theorem}
Sea $M$ un $R\text{-módulo}$ libre para $R$ dominio de integridad con
$B$ un conjunto linealmente independiente maximal. Para cualquier $S$
linealmente independiente,

\[ \# S \leq \# B.
\]

En particular, cualesquiera dos conjuntos linealmente independientes
maximales tienen la misma cardinalidad.
\end{theorem}

\begin{proof}
Empezamos tomando cuerpos de fracciones y pasamos a considrear el caso
de $R$ un cuerpo y $M$ un espacio vectorial.

Comprobaremos que podemos ir reemplazando elementos de $B$ por
elementos de $S$ sucesivamente para ir creando sucesivos $B'$ y
seguir manteniendo independencia lineal y la maximalidad. Si tomamos
$B' \cup \{v\}$ para algún $v \in S$, por maximalidad tenemos una
dependencia lineal

\[
c_0v + c_1b_1 + \dots + c_tb_t = 0
\]

con $c_0 \neq 0$ para no contravenir la independencia de $B$; además, no sólo
pueden existir elementos no nulos de $S$, porque contravendría su independencia.
Debe existir un $c_1 \neq 0$ con $b_1 \in B' \setminus S$ y podemos intercambiar $b_1$ por $v$
teniendo de nuevo un conjunto linealmente independiente maximal, ya que

\[
v = -c_0^{-1}c_1b_1 - \dots - c_0^{-1}c_tb_t.
\]

Si aplicamos inducción transfinita bajo una buena ordenación de $S$, podemos
asegurar que se llega a un conjunto de cardinalidad $\#B$ que contiene a
los elementos de $S$.
\end{proof}

\begin{corollary}
Para $R$ un dominio de integridad y dos conjuntos $A,B$,

\[
F^R(A) \cong F^R(B) \iff A \cong B.
\]
\end{corollary}

\begin{corollary}
Para $R$ un dominio de integridad se satisface la propiedad IBN

\[
R^m \cong R^n \iff m = n.
\]
\end{corollary}

\item Clasificación de módulos libres en dominios de ideales principales
\label{sec-7-5-3-5}
\begin{lemma}
Sea $R$ un dominio de ideales principales y $F$ un módulo libre finitamente
generado sobre él. Entonces existen $a\in R, x\in F, y\in M$ con $y = ax$ y
$M' \subseteq M,F' \subset F$ con $M' = F' \cap M$ submódulos cumpliendo

\[
F = \left\langle x \right\rangle \oplus F',
\qquad
M = \left\langle y \right\rangle \oplus M'.
\]
\end{lemma}

\begin{proof}
La familia de ideales $\left\{ \varphi \in \mathrm{Hom}(F,R) \mid \varphi(M) \right\}$ es no vacía. Como los PID son
noetherianos, tiene un elemento maximal $\alpha(M) = (a)$, para algún $\alpha(y) = a$.

Dado cualquier $\varphi(y)$, si tomamos el generador $(b) = (a,\varphi(y))$ tenemos que

\[
b = ra + s\varphi(y)
\]

y que si definimos $\psi = r\alpha + s\varphi$, tenemos $b = \psi(y) \in \psi(M)$, luego
$(a) \subseteq (b) \subseteq \psi(M)$, y por maximalidad, $a \mid \varphi(y)$.

Si vemos $y = \left( s_1,\dots,s_n \right)$ como elemento de $F\cong R^{\oplus n}$, tenemos $a \mid \pi_i(y) = s_i$,
así que sabemos $s_i = ar_i$, y definimos

\[
x = \left( r_1,\dots,r_n \right).
\]

Ahora tomamos $F' = \mathrm{ker}(\alpha)$ y comprobamos las sumas directas.
\end{proof}

\begin{proposition}
Sea $R$ un dominio de ideales principales y $F$ un módulo libre finitamente
generado sobre él. Todo submódulo $M \subset F$ será libre.
\end{proposition}

\begin{proof}
Aplicamos el lema anterior a los sucesivos $M^{(i)}$ que genere. Tendremos
que eventualmente $M^{(i)} = 0$, ya que los $y^{(i)}$ son independientes y $F$ es
finitamente generado.
\end{proof}

\begin{itemize}
\item Resoluciones en PIDs
\label{sec-7-5-3-5-1}
\begin{proposition}
Sea $R$ dominio de integridad. Será dominio de ideales principales si y sólo
si para cualquier epimorfismo a un módulo finitamente generado

\[
R^{m_0} \overset{\pi_0} \longrightarrow M \longrightarrow 0,
\]

existe un módulo libre haciendo exacta la secuencia

\[
0 \longrightarrow 
R^{m_1} \overset{\pi_1} \longrightarrow
R^{m_0} \overset{\pi_0} \longrightarrow
M \longrightarrow
0.\]
\end{proposition}
\end{itemize}

\item Anillo de endomorfismos
\label{sec-7-5-3-6}
\begin{proposition}
Los endomorfismos de un $R\text{-módulo}$ $F$, $\mathrm{End}(F)$ forman un álgebra con la
composición.
\end{proposition}

\begin{itemize}
\item Semejanza
\label{sec-7-5-3-6-1}
\begin{definition}
Dos matrices $A,B \in {\cal M}_n(R)$ son \textbf{semejantes} si representan el mismo
endomorfismo $F \to F$, diferenciándose en la elección de la base.
\end{definition}

\begin{proposition}
Dos matrices $A,B$ son semejantes si y sólo si

\[
B = PAP^{-1}.
\]
\end{proposition}

\begin{definition}
Dos endomorfismos $\alpha,\beta \colon F \to F$ son \textbf{semejantes} si existe un
automorfismo $\pi \colon F \to F$ cumpliendo

\[
\beta = \pi \circ \alpha \circ \pi^{-1}.
\] cite:aluffi09$_{\text{linear}}$
\end{definition}

\item Semejanza y acciones de anillos de polinomios
\label{sec-7-5-3-6-2}
\begin{proposition}
Una transformación lineal de $F$ es exactamente lo mismo que una estructura
como $R[X]\text{-módulo}$ compatible con la estructura de $R\text{-módulo}$.
\end{proposition}

Si tenemos una transformación lineal $\alpha$, podemos definir la acción de
un polinomio como

\[
\left( r_mt^m + \dots + r_1t + r_0 \right)(v) =
r_{m}\alpha^m(v) + \dots r_1\alpha(v) + r_0v.
\]

Y por la propiedad universal del anillo de polinomios, toda estructura
de $R[t]\text{-módulo}$ quedará determinada por el endomorfismo que asignemos a $t$.

\begin{lemma}
Dadas transformaciones lineales $\alpha,\beta$ de $F$; las estructuras como $R[t]\text{-módulo}$
son isomorfas si y sólo si $\alpha$ y $\beta$ son semejantes.
\end{lemma}
\begin{proof}
Si llamamos $F_{\alpha}$, $F_{\beta}$ a las dos estructuras como $R[t]\text{-módulo}$, tendremos que
un isomorfismo $\pi\colon F_{\alpha}\to F_{\beta}$ será lo mismo que una transformación invertible
$\pi\colon F \to F$ cumpliendo $\beta = \pi\circ\alpha\circ\pi^{-1}$.

Nótese de hecho que un isomorfismo entre módulos debe comportarse como

\[
\pi\circ\alpha (v) = \pi(tv) = t\pi(v) = \beta\circ\pi(v),
\]

por lo que $\pi\circ\alpha = \beta\circ\pi$ es la condición que lo distingue de cualquier
otra transformación lineal.
\end{proof}

\begin{corollary}
Hay una correspondencia biyectiva entre clases de semejanza de transformaciones
lineales de un $R\text{-módulo}$ libre $F$ y clases de isomorfía de estructuras de
$R[t]\text{-módulo}$ en $F$.
\end{corollary}

Nótese que esto se expande a las matrices en el caso finito-dimensional.
\end{itemize}

\item Proyectividad
\label{sec-7-5-3-7}
\begin{theorem}
Todo módulo libre es proyectivo.
\end{theorem}
\begin{proof}
Supongamos que tenemos un módulo libre $F$ sobre el conjunto $A$. Dado
un epimorfismo $\varphi\colon M \to N$, tendremos la situación siguiente, donde
podemos definir una aplicación de $A$ a $M$ por ser $\varphi$ epimorfismo.
Dado cualquier $\alpha\colon F \to N$ se tiene

\[\begin{tikzcd}
& A\dar{i}\ar{ddl} \\
& F\dar{\alpha}\dlar[dashed] \\
M\rar{\varphi} & N \\
\end{tikzcd}\]

tales que conmutan el triángulo exterior y el superior. Así tenemos
que ambas funciones coinciden sobre la base y por tanto coinciden
para todo el módulo libre.
\end{proof}
\item Referencias
\label{sec-7-5-3-8}
bibliographystyle:unsrt
bibliography:math.bib
\end{itemize}
\subsubsection*{Categorías abelianas}
\label{sec-7-5-4}
\begin{itemize}
\item Objeto nulo
\label{sec-7-5-4-1}
\begin{definition}
En una categoría, un \textbf{objeto nulo} es aquel que es a la vez inicial y final.
\end{definition}

Nótese que no todas las categorías tienen por qué tener un objeto nulo.
La categoría $\mathtt{Set}$, por ejemplo, tiene objetos inicial y final no isomorfos.

\begin{definition}
En una categoría con objeto nulo llamamos \textbf{morfismo cero} entre dos
objetos, $0_{a,b}\colon a \to b$, al que resulta de componer el único morfismo $a \to 0$ con 
el único morfismo $0 \to b$.
\end{definition}

\item Núcleos y conúcleos
\label{sec-7-5-4-2}
\begin{definition}
En una categoría con objeto nulo, el \textbf{núcleo} de un morfismo
$f \colon a \to b$ es un morfismo $k \colon \mathrm{ker}(f) \to a$ tal que $f\circ k = 0$ y que es universal 
respecto a esa propiedad; es decir, para cualquier otro $h$ cumpliendo 
que $f \circ h = 0$, se tiene el diagrama

\[\begin{tikzcd}
c \dar[dashed]{\exists! h'}\ar[bend right=90,swap]{dd}{h}\arrow[bend left=45]{ddr}{0} &   \\
\mathrm{ker}(f) \dar{k}\drar{0} &   \\
a\rar{f} & b & .\\
\end{tikzcd}\]
\end{definition}

De otra forma, podríamos definirlo como el \textbf{ecualizador} del morfismo $f$ con
el morfismo cero, es decir, como el universal respecto al diagrama

\[\begin{tikzcd} \mathrm{ker}(f) \rar{k} & 
a \rar[bend left]{f}\rar[bend right,swap]{0} & b
\end{tikzcd},\]

y por tanto, es un límite finito y es único salvo isomorfismo. cite:aluffi09$_{\text{linear}}$

\begin{definition}
En una categoría con objeto nulo, se define el \textbf{conúcleo}, $c \colon b \to \mathrm{coker}(f)$
de manera dual al núcleo, como universal según el siguiente diagrama 
conmutativo

\[\begin{tikzcd}
c  &   \\
\mathrm{coker}(f)  \uar[dashed]{\exists! h'} &   \\
a \ar[bend left=90]{uu}{0}\uar{0} \rar{f} & b \ular[swap]{c} \arrow[bend right=45,swap]{uul}{h} \\
\end{tikzcd}\]
\end{definition}

\begin{itemize}
\item Propiedades del núcleo
\label{sec-7-5-4-2-1}
\begin{proposition}
Cualquier núcleo es un monomorfismo. Dualmente, cualquier conúcleo es
un epimorfismo.
\end{proposition}
\begin{proof}
Si se tienen dos $m,n\colon d \to \mathrm{coker}(f)$, entonces sabemos que $m \circ k$ y $n \circ k$,
por propiedad universal, hacen que exista un único $h \circ k = m\circ k=n\circ k$.
Debe tenerse por tanto $h=m=n$.
\end{proof}

Nótese que el converso no tiene por qué ser cierto. En general, no todo
monomorfismo es núcleo ni todo epimorfismo es conúcleo.

\item Ejemplo: grupos
\label{sec-7-5-4-2-2}
En la categoría $\mathtt{Grp}$, el objeto cero es el grupo trivial. El núcleo de
cualquier morfismo es lo que llamamos usualmente núcleo, como se puede
comprobar trivialmente. Nótese que todos los núcleos son normales en un 
grupo pero que no todas las inclusiones lo son como subgrupo normal, por 
lo que no todos los monomorfismos serán aquí núcleos.
\end{itemize}

\item Categorías preaditivas
\label{sec-7-5-4-3}
\begin{definition}
Una \textbf{categoría preaditiva} es aquella en la que cada conjunto de morfismos
$\mathrm{hom}(a,b)$ es un grupo abeliano y la composición es bilinear respecto a la
operación de grupo.
\end{definition}

\begin{proposition}
Para un objeto en una categoría preaditiva, $z \in {\cal A}$, equivalen:

\begin{enumerate}
\item $z$ es inicial.
\item $z$ es final.
\item $\mathrm{id}_z$ es el elemento neutro de $\mathrm{hom}(z,z)$.
\item $\mathrm{hom}(z,z)$ es el grupo trivial.
\end{enumerate}
\end{proposition}
\begin{proof}
Si $z$ es inicial o final, se tiene un único $\mathrm{id}_z = 0$, que da el grupo trivial.
Si se tiene $\mathrm{id}_z=0$, entonces para cualquier morfismo $f\colon a \to z$, se tendrá

\[
f = \mathrm{id}_z \circ f = 0\circ f = 0
\]

por la bilinealidad de la composición. Dualmente se verá que es inicial.
\end{proof}

\begin{itemize}
\item Biproductos
\label{sec-7-5-4-3-1}
\begin{definition}
Un \textbf{biproducto} para dos objetos en una categoría preaditiva $a,b \in A$ es un
$c$ con morfismos

\[\begin{tikzcd}
a \rar[bend right,swap]{i_{1}} &
c \rar[bend left]{p_2} \lar[bend right,swap]{p_1} &
b \lar[bend left]{i_2}
\end{tikzcd}\]

cumpliendo las identidades $p_1i_1 = \mathrm{id}_a$, $p_2i_2 = \mathrm{id}_b$, y $i_1p_1 + i_2p_2 = \mathrm{id}_{c}$.
\end{definition}

\begin{theorem}
Dos objetos en una categoría preaditiva $a,b \in A$ tienen producto (o coproducto) 
si y sólo si tienen un biproducto, que será a su vez producto y coproducto.
\end{theorem}
\end{itemize}

\item Categorías abelianas
\label{sec-7-5-4-4}
\begin{definition}
Una \textbf{categoría abeliana} es una categoría preaditiva cumpliendo que

\begin{enumerate}
\item tiene un objeto nulo.
\item tiene biproductos finitos.
\item todo morfismo tiene núcleo y conúcleo.
\item todo monomorfismo es núcleo y todo epimorfismo es conúcleo.
\end{enumerate}
\end{definition}

\begin{itemize}
\item Factorización de un morfismo
\label{sec-7-5-4-4-1}
\begin{proposition}
En una categoría abeliana, cada morfismo se factoriza como $f = m\circ e$,
donde $m = \mathrm{ker}(\mathrm{coker}(f))$ y $e = \mathrm{coker}(\mathrm{ker}(f))$. 
Además, esta factorización cumple que, dada cualquier otra factorización
de la forma $f' = m'e'$ con $m'$ monomorfismo, $e'$ epimorfismo y con morfismos
de la forma

\[\begin{tikzcd}
\cdot \rar{f}\dar[swap]{g} & \cdot \dar{h} \\
\cdot \rar[swap]{f'} & \cdot &,
\end{tikzcd}\]

existe un único $k$ cumpliendo

\[\begin{tikzcd}
\cdot \arrow[bend left]{rr}{f}\dar[swap]{g} \rar{e}& 
\cdot\rar{m}\dar{k} & \cdot \dar{h} \\
\cdot \arrow[bend right,swap]{rr}{f'} \rar{e'} & \cdot\rar{m'} & \cdot
\end{tikzcd}\]
\end{proposition}
\begin{proof}
Tomamos $m = \ker(\operatorname{coker} f)$. Como $(\operatorname{coker} f)\circ f = 0$, por propiedad universal
del núcleo sabemos que $f$ se escribe como $f = me$ para algún $e$. Como puede
demostrarse que $e$ será epimorfismo, luego $e = \operatorname{coker}(\ker f)$.

Dadas $f=me$ y $f'=m'e'$ con $g,h$ del diagrama, consideramos
$u = \ker f = \ker e$ y entonces tenemos que $0 = hfu = m'e'gu$, luego $e'gu = 0$.
Por ser $u$ núcleo, $e'g$ factoriza en $e = \operatorname{coker}(u)$ como $e'g = ke$ para algún
$k$ que además debe ser único. Así, $m'ke = hme$ y $m'k = hm$, dando la
conmutatividad del diagrama.
\end{proof}

\begin{definition}
La \textbf{imagen} y \textbf{coimagen} de un morfismo $f = me \colon a \to b$ se definen como

\begin{itemize}
\item $\operatorname{im} f = m$
\item $\operatorname{coim} f = e$
\end{itemize}
\end{definition}

La proposición anterior se usa para comprobar que son únicas salvo
isomorfismo.
\end{itemize}

\item Secuencias exactas
\label{sec-7-5-4-5}
\begin{itemize}
\item Exactitud
\label{sec-7-5-4-5-1}
\begin{definition}
Un par de morfismos componibles es \textbf{exacto} en el objeto que comparten
cuando $\operatorname{im} f = \operatorname{ker} g$. Equivalentemente, cuando $\operatorname{coker} f = \operatorname{coim} g$.
\end{definition}

\item Complejos de cadenas
\label{sec-7-5-4-5-2}
\begin{definition}
En una categoría abeliana, un \textbf{complejo de cadenas} es una secuencia

\[\begin{tikzcd}
\dots \rar &
c_{n+1} \rar{\partial_{n+1}} &
c_n \rar{\partial_n} &
c_{n-1} \rar &
\dots
\end{tikzcd}\]

cumpliendo que $\partial_n\partial_{n+1} = 0$.
\end{definition}

\item Secuencias exactas cortas
\label{sec-7-5-4-5-3}
\begin{definition}
Una \textbf{secuencia exacta corta} es un diagrama

\[
0 \longrightarrow
a \overset{f}\longrightarrow
b \overset{g}\longrightarrow
c \longrightarrow
0
\]

que es exacto en $a$,$b$ y $c$.
\end{definition}

\begin{definition}
Un \textbf{morfismo de secuencias exactas cortas} está formado por tres morfismos
$f,g,h$ que hacen conmutar el diagrama

\[\begin{tikzcd}
0 \rar& 
\cdot \rar{m}\dar{f}& 
\cdot \rar{e}\dar{g}& 
\cdot \rar\dar{h}& 
0 \\
0 \rar& 
\cdot \rar{m'}& 
\cdot \rar{e'}& 
\cdot \rar& 
0 & .\\
\end{tikzcd}\]

Las secuencias exactas cortas de una categoría abeliana $A$ con estos 
morfismos forman la categoría $\mathtt{Ses}(A)$, que se hace preaditiva sumando
las tres componentes de cada morfismo.
\end{definition}
\end{itemize}

\item Resultados en categorías abelianas
\label{sec-7-5-4-6}
\begin{itemize}
\item Manipulación elemental en categorías abelianas
\label{sec-7-5-4-6-1}
\begin{proposition}
Si dados dos morfismos $f,g$ hacia $c$ calculamos su producto fibrado
(\emph{pullback}) tendremos que $f$ epimorfismo nos da $f'$ epimorfismo en

\[\begin{tikzcd}
s\rar{f'} \dar[swap]{g'} & d \dar{g} \\
b\rar{f} & c
\end{tikzcd}\]

donde además, el núcleo de $f$ factoriza como $\mathrm{ker}(f) = g'\circ \mathrm{ker}(f')$.
\end{proposition}
\begin{proof}
El producto fibrado se construye formando la secuencia exacta

\[\begin{tikzcd}
0\rar&s\rar{m}&b\oplus d\rar{fp_1-gp_2}& c
\end{tikzcd}\]

y tomando $g' = p_1m$ y $f'=p_2m$. Probaremos que $fp_1-gp_2$ es un
epimorfismo, para lo que basta comprobar que si $h(fp_1-gp_2) = 0$
entonces

\[
0 = h(fp_1-gp_2)i_1 = hfp_1i_1 = hf,
\]

y por ser epimorfismo $f$, $h = 0$. Ahora probaremos que $f'$ es
epimorfismo; si $uf'=up_2m=0$, por exactitud, es de la forma
$up_2 = u'(fp_1-gp_2)$. Ahora tenemos

\[
0 = up_2i_1 = u'f
\]

llegándose a $u'=0$ por ser $f$ epimorfismo.
\end{proof}

\begin{definition}
Definimos un \textbf{miembro} de $a$ como un morfismo con codominio $a$. Existe
una equivalencia $x \equiv y$ entre dos miembros cuando existen epimorfismos
$u,v$ tales que $xu=yv$.
\end{definition}
\begin{proof}
Para demostrar la transitividad de esta relación de equivalencia,
debemos aplicar la proposición anterior al diagrama siguiente,

\[\begin{tikzcd}
\cdot \rar\dar & 
\cdot \rar\dar& 
\cdot \dar{x}\\
\cdot \rar\dar& 
\cdot \rar{y}\dar{y}&
a \\
\cdot \rar{z} &
a &&,\\
\end{tikzcd}\]

donde probamos que si $x \equiv y$ y $y \equiv z$, entonces $x \equiv z$.
\end{proof}

Dado un morfismo $f \colon a \to b$, cada $x \in a$ da lugar a $f \circ x \in b$; y además,
$x \equiv y$ implica $f x \equiv f y$. Gracias a esto, podemos tratar a los miembros
de un objeto en una categoría abeliana de la misma manera de la que
tratamos a los elementos de un conjunto. La aplicación de funciones
se comporta de la misma manera y preserva la relación de equivalencia
de los miembros.

\begin{proposition}
En cualquier categoría abeliana cite:lane78categories

\begin{enumerate}
\item $f \colon a \to b$ es \emph{monomorfismo} ssi para $x \in a$, $f(x) \equiv 0 \implies x \equiv 0$.
\item $f \colon a \to b$ es \emph{monomorfismo} ssi para $x,y \in a$, $f(x) \equiv f(y) \implies x\equiv y$.
\item $g\colon b \to c$ es \emph{epimorfismo} ssi para $z\in c$, existe $y \in b$ con $g(y) \equiv z$.
\item $h\colon r \to s$ es \emph{nulo} ssi para $x \in r$, $hx \equiv 0$.
\item $a \overset{f}\to b\overset{g}\to c$ es \emph{exacta} ssi $gf = 0$ y para cada $g(y)\equiv 0$ existe un
$x \in a$ tal que $f(x) \equiv v$.
\item Si existen $g(x) = g(y)$, existe $g(z) = 0$; además cualquier $f(x) \equiv 0$
    implica $f(y) \equiv f(z)$ y cualquier $h(y)\equiv 0$ implica $h(x) \equiv -h(z)$.
\end{enumerate}
\end{proposition}
\begin{proof}
Se tienen (1) y (2) por definición de monomorfismo. Se tiene además
(3) por construcción del producto fibrado y (4) por definición.

Si factorizamos $f = me$, por exactitud se tendrá $\operatorname{ker} g = m$. Si $g y \equiv 0$,
$y \equiv my'$, y si construimos el producto fibrado

\[\begin{tikzcd}
\cdot\dar[dashed]{y''}\rar[dashed]{e'} & \cdot\dar{y'}\drar[bend left=45]{y} \\
\cdot\rar{e} & \cdot\rar{m} & \cdot &,
\end{tikzcd}\]

como $e'$ es epimorfismo, $y \equiv fy''$.

A la inversa, si para $y \in b$ existe $k = \ker g$, entonces $k \in b$ y $gk \equiv 0$.
Existe entonces $x \in a$ con $fx \equiv k$, es decir, $ku \equiv mexv$. Esto lleva
a $\operatorname{im} f \geq \ker g$ y a $gf = 0$, la exactitud.
\end{proof}

\item Lema de los cinco
\label{sec-7-5-4-6-2}
\begin{theorem}
En un diagrama conmutativo con filas exactas

\[\begin{tikzcd}
a_1 \rar{g_1} \dar{f_1} & 
a_2 \rar{g_2} \dar{f_2} &
a_3 \rar{g_3} \dar{f_3} & 
a_4 \rar{g_4} \dar{f_4} & 
a_5 \dar{f_5} \\
b_1 \rar{h_1} &
b_2 \rar{h_2} &
b_3 \rar{h_3} &
b_4 \rar{h_4} &
b_5 & ,
\end{tikzcd}\]

si $f_2,f_4$ son isomorfismos, $f_1$ es epimorfismo y $f_5$ es monomorfismo, $f_3$ es isomorfismo.
\end{theorem}
\begin{proof}
Usando la manipulación de diagramas cuyas reglas hemos escrito en
la proposición anterior, demostraremos que $f_3$ es monomorfismo.
La dualidad servirá para demostrar a su vez que es epimorfismo.

Explícitamente, si hubiera un elemento en $a_3$ que diera un cero en
$b_3$, habría un cero en $b_4$, por ser isomorfismo, habría un cero en
$a_4$, y entonces existiría, por exactitud, un elemento en $a_2$ cuya
imagen sería el elemento original en $a_3$. Por isomorfismo, este
debería dar un elemento en $b_2$ cuya imagen sería cero, así que
por exactitud existiría un elemento en $b_1$ del que sería imagen.
Como $f_1$ es epimorfismo, existiría un elemento en $a_1$ del que
sería imagen, y entonces el elemento original sería la imagen
por la composición de dos morfismos en secuencia exacta del
primer elemento. Debería ser cero, quedando probado $f_3$ como
monomorfismo.
\end{proof}

\item Lema de la serpiente
\label{sec-7-5-4-6-3}
\begin{theorem}
Dado un morfismo de secuencias exactas cortas $f,g,h$; existe un morfismo
$\delta \colon \operatorname{ker} h \to \operatorname{coker} f$ tal que la secuencia siguiente es exacta

\[\begin{tikzcd}
0 \rar &
\mathrm{ker}(f) \rar{m} &
\mathrm{ker}(g) \rar{e} &
\mathrm{ker}(h) \arrow[out = 0,in =180,swap]{dll}{\delta} \\&
\mathrm{coker}(f) \rar{m'} &
\mathrm{coker}(g) \rar{e'} &
\mathrm{coker}(h) \rar &
0
\end{tikzcd}\]
\end{theorem}
\begin{proof}
El diagrama extendido que tenemos es

\[ \begin{tikzcd}
	& 0 \dar              & 0 \dar            & 0 \dar           &   \\
 0 \rar & ker(f) \rar \dar  & ker(g) \rar \dar    & ker(h) \dar \ar[out=355, in=175,looseness=1, overlay, swap]{dddll}{\delta}       &   \\
 0 \rar & a \rar{m} \dar{f}  & b \rar{e} \dar{g} & c \rar \dar{h}        & 0 \\
 0 \rar & a' \rar{m'} \dar & b' \rar{e'} \dar & c' \rar \dar        & 0 \\
	& coker(f) \rar \dar & coker(g) \rar \dar  & coker(h) \rar \dar & 0 \\
	& 0                   & 0                 & 0                &
 \end{tikzcd} \]

y desde él, manipulando de nuevo el diagrama podemos construir primero
el morfismo $\delta$ y demostrar después que efectivamente es exacto.

Explícitamente, lo que haríamos sería tomar un elemento en $\mathrm{ker}(h)$.
Este elemento pasaría a $c$ y luego, como un cero a $c'$. Como $e$ es
sobreyectiva, existiría un elemento en $b$, y luego uno en $b'$ que
haría conmutar el diagrama. Pero como este elemento iría hacia
un cero al aplicar $e'$, debería estar en la imagen de $m'$, así sólo
deberíamos pasar de $a$ a $\mathrm{coker}(f)$ para terminar la construcción de
$\delta$.

Nótese que si el elemento procede de $\mathrm{ker}(g)$, entonces sería nulo en $b'$,
y de ahí sería nulo en $a'$ y en $\mathrm{coker(f)}$; y que la imagen de un
elemento que hubiera llegado desde $\delta$, al pasar a $\mathrm{coker}(g)$ debería ser
$0$ por provenir desde $b$. Esto demuestra la exactitud del diagrama.
\end{proof}
\end{itemize}

\item Referencias
\label{sec-7-5-4-7}
bibliographystyle:unsrt
bibliography:math.bib
\end{itemize}

\section*{Affine group schemes seminar}
\label{sec-8}
\subsection*{I. Álgebras de Hopf}
\label{sec-8-1}
\subsubsection*{1. Definiciones}
\label{sec-8-1-1}
\begin{itemize}
\item Álgebra de Hopf
\label{sec-8-1-1-1}
Un \textbf{álgebra de Hopf} es una biálgebra (álgebra y coálgebra) con un 
antiautomorfismo llamado \emph{antípoda}.

\begin{itemize}
\item Explícitamente
\label{sec-8-1-1-1-1}
Tenemos $(H, m, \eta, \Delta, \varepsilon, S)$ como componentes del álgebra de Hopf sobre
un cuerpo $k$, donde:

\begin{itemize}
\item $H$ es el álgebra.
\item $m : H \otimes H \to H$ es el producto.
\item $\eta : k \to H$ es la unidad.
\item $\Delta : H \to H \otimes H$ es la comultiplicación.
\item $\varepsilon : H \to k$ es la counidad.
\item $S : H \to H$ la antípoda.
\end{itemize}

Bajo ciertas condiciones de compatibilidad.
\end{itemize}

\item Group-like elements
\label{sec-8-1-1-2}
Elementos no nulos cumpliendo $\Delta(x) = x \otimes x$. Forman un grupo con la inversa
dada por la antípoda.
\end{itemize}

\subsection*{II. Introduction to affine group schemes}
\label{sec-8-2}
\subsubsection*{1. Definition and examples}
\label{sec-8-2-1}
\begin{itemize}
\item Affine group scheme
\label{sec-8-2-1-1}
An \textbf{affine group scheme} over $k$ is a representable functor $\mathtt{Alg}_k \to \mathtt{Grp}$.
More precisely, the composition of the functor with $\mathtt{Grp}\to\mathtt{Set}$ is
representable.

\item Connection with affine algebraic varieties
\label{sec-8-2-1-2}
If $V$ is an affine algebraic variety, the we can define the corresponding
affine scheme as $Alg_k(K[V], -)$, where $K[V]$ is the coordinate algebra.

\item Algebraic affine scheme
\label{sec-8-2-1-3}
An affine scheme is said to be \textbf{algebraic} if its representing object is
finitely generated as a k-algebra.
\end{itemize}

\subsection*{III. Esquemas diagonalizables y constantes}
\label{sec-8-3}
\subsubsection*{1. Introducción}
\label{sec-8-3-1}
\begin{itemize}
\item Álgebra grupo
\label{sec-8-3-1-1}
Dado un grupo $G$ y un cuerpo $k$, el álgebra grupo $k[G]$ está formada como el
espacio vectorial libre sobre $G$ con el producto que induce el grupo.

\begin{itemize}
\item Estructura de álgebra de Hopf
\label{sec-8-3-1-1-1}
Este álgebra tiene estructura de álgebra de Hopf si extendemos linealmente
las siguientes aplicaciones:

\begin{itemize}
\item $\Delta(x) = x \otimes x$
\item $\varepsilon(x) = 1$
\item $S(x) = x^{-1}$
\end{itemize}
\end{itemize}
\end{itemize}
\section*{EUTypes Summer School}
\label{sec-9}
\subsection*{Introduction to type theory}
\label{sec-9-1}
\subsubsection*{Bibliography}
\label{sec-9-1-1}
HP. Barendregt. Lambda calculus: syntax and semantics.
F. Cardone, JR. Hindley. History of lambda-calculus and combinatory logic.
Statman. Lambda calculus with types.
Benjamin Pierce. Types and programming languages.
JL Krivine. Lambda calculus, types and models.

\subsubsection*{Introduction to type theory I}
\label{sec-9-1-2}
\begin{itemize}
\item Introduction
\label{sec-9-1-2-1}
\begin{itemize}
\item Gentzen
\label{sec-9-1-2-1-1}
Gentzen: natural deduction/sequent calculus/axiomatic system.
\item Functions
\label{sec-9-1-2-1-2}
We can give multiple notions of function

\begin{itemize}
\item functions as black boxes.
\item set-theoretical definition.
\end{itemize}

We can define a domain and codomain for functions. This notion leads
to the notion of the type of a function.
\end{itemize}

\item Untyped lambda calculus
\label{sec-9-1-2-2}
\begin{itemize}
\item Informal syntax
\label{sec-9-1-2-2-1}
Lambda terms are divided in

\begin{itemize}
\item variables, which can be bound or free. There is a countable set of
variables.
\item application of terms, function application.
\item lambda-abstractions, function generation by binding a variable.
\end{itemize}

An example is $\lambda x. x+42$. We see the application as left-associative.

\[
M ::= x \mid MM \mid \lambda x.M
\]

\item Examples
\label{sec-9-1-2-2-2}
Those are examples of lambda combinators.

\begin{itemize}
\item $I = \lambda x.x$
\item $K = \lambda xy. x$
\item $\Delta = \lambda x . xx$
\item $Y = \lambda f.(\lambda x. f(xx))(\lambda x. f(xx))$
\item $\Omega = (\lambda x.xx)(\lambda x.xx)$.
\end{itemize}

\item Free variables and closed terms
\label{sec-9-1-2-2-3}
We define the set of free variables recursively. A closed term or
\textbf{combinator} has no free variables.

\item $\alpha$-conversion
\label{sec-9-1-2-2-4}
Renaming of bound variables. This could also be done by using \textbf{De
Bruijn} notation. We apply the Barendregt's convention of renaming
variables that would be bound after a $\beta$-reduction.

\[
\lambda x.M \longrightarrow_{\alpha} \lambda y.M[y/x]
\]

\item $\beta$-reduction
\label{sec-9-1-2-2-5}
It represents function application of functions in lambda calculus.

\[
(\lambda x.M)N \longrightarrow_{\beta} M[N/x]
\]

\begin{itemize}
\item Substitution as a meta notion
\label{sec-9-1-2-2-5-1}
Substitution is an implicit meta notion that can be defined
recursively over terms

\begin{itemize}
\item $x[M/x] := M$
\item $\dots$
\end{itemize}
\end{itemize}

\item $\eta$-conversion
\label{sec-9-1-2-2-6}
It represents function extensionality

\[
\lambda x.(Mx) \longrightarrow_{\eta} M
\]

\item Normal form
\label{sec-9-1-2-2-7}
A term is in \textbf{normal form} if beta-reduction cannot be applied.
For example

\begin{itemize}
\item $I$ is in normal form.
\item 4$KI(KII)$ is strongly normalizing (SN) to $I$.
\item $KI\Omega$ normalizing term.
\item $Y$ is only \textbf{head-normalizable}, or solvable.
\end{itemize}

Evaluation order is important; $KI\Omega$ stops or enters an infinite loop
depending on the evaluation order; this is a normalizing but not strongly
normalizing term.

\item Confluence and the Church-Rosser theorem
\label{sec-9-1-2-2-8}
If $M \to N$ and $M \to P$, then there exists $S$ such that $N \longrightarrow S$
and $P \longrightarrow S$. The proof is not trivial.

\begin{itemize}
\item Corollaries
\label{sec-9-1-2-2-8-1}
The order of applied reductions is arbitrary. The Normal form is
unique if it exists.
\end{itemize}

\item Normalisation therem
\label{sec-9-1-2-2-9}
A term is in head-normal form if its head is a lambda abstraction.
A term is in normal form if there are no $\beta$ nor $\eta$ redexes.

The normalisation theorem says that the leftmost strategy results
in the normal form of $M$ if and only if it has a normal form.

\item Fixed-point theorem
\label{sec-9-1-2-2-10}
There is a fixed point combinator

\[
Y \equiv \lambda f. (\lambda x.f(xx))(\lambda x.f(xx))
\]

such that $\forall F. YF \equiv F(YF)$.

\item Church encoding
\label{sec-9-1-2-2-11}
Logic and arithmetic can be encoded in lambda calculus via
Church numerals.

\[
n :\equiv \lambda fx. f^n x
\]

\item Expressiveness of lambda calculus
\label{sec-9-1-2-2-12}
In the 1930s

\begin{itemize}
\item Kleene: it is equivalent to recursive funtions.
\item Church
\item Curry
\end{itemize}
\end{itemize}

\item Typed lambda calculus
\label{sec-9-1-2-3}
\item Intersection types
\label{sec-9-1-2-4}
\end{itemize}
\subsubsection*{Introduction to type theory II}
\label{sec-9-1-3}
\begin{itemize}
\item Disadvantages of untyped lambda calculus
\label{sec-9-1-3-1}

\begin{itemize}
\item There exist lambda terms without normal form.
\item Meaningless expressions.
\end{itemize}

This motivates two typing paradigms

\begin{itemize}
\item Implicit type assignment: lambda calculus with types.
\item Explicit type assignment: typed lambda calculus.
\end{itemize}
\item Sintatic definition of typed lambda calculus
\label{sec-9-1-3-2}
Type assignments $M : \sigma$, declarations $x : \sigma$ and environments
$\Gamma = \left\{ x_1:\sigma_1,\dots, x_s:\sigma_s \right\}$. With rules

\begin{prooftree}
\AxiomC{}
\UnaryInfC{$\Gamma, x:\sigma \vdash x:\sigma$ }
\end{prooftree}


\begin{prooftree}
\AxiomC{$\Gamma \vdash M : \sigma \to \tau$}
\AxiomC{$\Gamma \vdash N : \sigma$}
\BinaryInfC{$\Gamma \vdash \lambda x . M : \sigma \to \tau$ } 
\end{prooftree}

\begin{prooftree}
\AxiomC{$\Gamma, x:\sigma\vdash y:\tau$}
\UnaryInfC{$\Gamma \vdash \lambda x.y : \sigma \to \tau$}
\end{prooftree}

It can be defined as a natural deduction system with introduction
and elimination rules.

\item Typing example
\label{sec-9-1-3-3}
There are non-typable normal forms.

\begin{itemize}
\item $I : \sigma \to \sigma$
\item $K : \sigma \to \tau \to \sigma$
\item $\Delta$, $Y$ or $\Omega$ can not be typed
\end{itemize}

\item Type preservation
\label{sec-9-1-3-4}
If $M \longrightarrow P$ and $M:\sigma$, then $P:\sigma$.

\item Generation and substitution lemmas
\label{sec-9-1-3-5}
If $\Gamma \vdash \lambda x. M: \varphi$, then $\varphi = \sigma \to \tau$ and $\Gamma, x:\sigma \vdash M : \tau$.
If $\Gamma, x:\sigma \vdash M : \tau$ and $\Gamma \vdash N:\tau$, then $\Gamma \vdash M[x/N]:\tau$.

\item Strong normalization
\label{sec-9-1-3-6}
If $M : \sigma$, then $M$ is strongly normalizing. This was proven by
Tait in 1967.
\item Typability and inhabitation
\label{sec-9-1-3-7}
Questions on lambda calculus

\begin{itemize}
\item \textbf{Typability:} iven a term, find a type for it.
\item \textbf{Inhavitation:} given a type, construct a term of that type.
\item \textbf{Type checking:} check the type of a term.
\end{itemize}

Typability is decidable in simply typed lambda calculus. It is
decidable in second order lambda calculus with the Hindley-Milner
algoritm.

Inhabitation is equivalent to the intuitionistic logic of Gentzen's
natural deduction. The rules of typed lambda calculus are the rules
of natural deduction if we do not use the terms.

\item Curry-Howard correspondence
\label{sec-9-1-3-8}
A formula is provable in IL iff it is inhabited in simply-typed lambda
calculus. This is also the language of Cartesian Closed Categories
(Lambek, 1970).

BHK interpretation of logical connectives is formalized by the 
Curry-Howard correspondence.

\item Consistency/Completeness/Decidability
\label{sec-9-1-3-9}
Intuitionistic propositional logic (IL) is consistent, complete and
decidable. Due to Curry-Howard, inhabitation is decidable in STLC.
\item Lambda cube
\label{sec-9-1-3-10}
If any $M$ is typable, $M$ is strongly normalizing.
The \href{https://en.wikipedia.org/wiki/Lambda_cube}{lambda cube} represent multiple type systems.

\[\begin{tikzcd}
& & & \\
\lambda 2 & & \lambda P2 & \\
& \lambda & & \\
\lambda_{\to}& & & & \\
\end{tikzcd}\]
\item Intersection types
\label{sec-9-1-3-11}
In our current system, $\Delta$ is not typeable. We are going to introduce
intersection types with elimination rules

\begin{prooftree}
\AxiomC{$\Gamma \vdash M : \sigma \cap \tau$}
\UnaryInfC{$\Gamma \vdash M : \sigma$}
\end{prooftree}

\begin{prooftree}
\AxiomC{$\Gamma \vdash M : \sigma \cap \tau$}
\UnaryInfC{$\Gamma \vdash M : \tau$}
\end{prooftree}

and a introduction rule

\begin{prooftree}
\AxiomC{$M:\sigma$}
\AxiomC{$M:\tau$}
\BinaryInfC{$M:\sigma\cap\tau$}
\end{prooftree}

In general, the Curry-Howard correspondance is lost here. We create
a new system $\lambda\cap$, but in this system, $\sigma \to \tau \to \sigma \cap \tau$ is provable while
it is not inhabited.

\begin{itemize}
\item Now self application is typable
\label{sec-9-1-3-11-1}
The self application can be of type $\lambda x.xx :((\sigma \to \tau) \cap \sigma) \to \tau$.
\end{itemize}

\item Characterization of strong normalization on intersection types
\label{sec-9-1-3-12}
A term is typable iff it is strongly normalizing.

For example, $KI\Omega$ is not typable, even if $I$ is.

\item Typability and inhabitation are undecidable with intersection types
\label{sec-9-1-3-13}
\item Models of lambda-calculus
\label{sec-9-1-3-14}
We can prove completeness of type assignment. It is a theorem that

\[
\Gamma \vdash M:\sigma \iff \Gamma \models M : \sigma
\]
\end{itemize}
\subsubsection*{Pure type systems I}
\label{sec-9-1-4}
\textbf{alx@minuw.edu.pl}

\begin{itemize}
\item Simple type systems
\label{sec-9-1-4-1}
Differences between logics and type systems:

\begin{itemize}
\item focus on computation instead of consistency.
\item it is meaningful to have two assumptions of the same type.
\item it is meaningful to use wh same assumption twice.
\end{itemize}

Combinatory logic, with combinators S,K, defines \emph{minimal logic}.
With them, deduction theorem is provable. But we can add other
combinators such as B,C or W.
\item More complex type systems
\label{sec-9-1-4-2}
We can add polymorphism with type variables. We get SystemF (aka
$\lambda 2$). We need more complex typing rules and beta reduction for types.
$\lambda P$ was proposed by deBruijn, Harper, Longo and Moggi.

\item Properties of interest of a pure type system
\label{sec-9-1-4-3}

\begin{enumerate}
\item Church-Rosser property. Values are computed deterministically.
\item Subject reduction property. Types are invariants of the reduction.
\item Strong normalisation property. Computation terminates.
\end{enumerate}

Those properties prove consistency of the logical system.
\item Examples of PTSs
\label{sec-9-1-4-4}
$\lambda_{\to}, \lambda 2, \lambda P, \lambda \omega, \lambda C, \lambda \ast, \lambda U$

\item Lemmas for PTSs
\label{sec-9-1-4-5}
\begin{itemize}
\item Free variables
\label{sec-9-1-4-5-1}
\item Transitivity of contexts
\label{sec-9-1-4-5-2}
\item Substitution
\label{sec-9-1-4-5-3}
\item Weakening
\label{sec-9-1-4-5-4}
\item Generation lemma
\label{sec-9-1-4-5-5}
\item Condensing lemma
\label{sec-9-1-4-5-6}
\end{itemize}
\item Properties of PTSs
\label{sec-9-1-4-6}
\begin{itemize}
\item Church-Rosser property
\label{sec-9-1-4-6-1}
There is a PTS extended with a number of axioms which does not have
the Church-Rosser property.

\item Geuvers theorem
\label{sec-9-1-4-6-2}
All functional strongly normalizing PTSs have the Church-Rosser
property.

\item Functionality
\label{sec-9-1-4-6-3}
A PTS (S,A,R) is functional when A is a function from S to S and
R is a function from $S \times S$ to $S$.

\item Uniqueness of types lemma
\label{sec-9-1-4-6-4}
\end{itemize}
\end{itemize}
\subsubsection*{Pure type systems II}
\label{sec-9-1-5}
\begin{itemize}
\item The type inhabitation problem
\label{sec-9-1-5-1}
\end{itemize}
\subsection*{Dependently typed programming}
\label{sec-9-2}
\subsubsection*{Milner's coincidence}
\label{sec-9-2-1}
Milner's Coincidence on Hindley-Milner's type systems

\begin{center}
\begin{tabular}{ll}
\hline
Terms & Types\\
\hline
what we write & we don't write these\\
what we read & invisible (except errors)\\
what gets compiled & what gets erased\\
non dependent $\lambda$ & polymorphism over types with $\Lambda$\\
\hline
\end{tabular}
\end{center}

In the late 90s, this was the accepted unquestioned way of thinking
about types.
\subsection*{Homotopy type theory}
\label{sec-9-3}
The Tao of Types - Thorsten Altenkirch

\begin{itemize}
\item A topological model of HoTT.
\end{itemize}

\subsubsection*{HoTT 1}
\label{sec-9-3-1}
There are multiple implementations of type theory (Coq, Agda, \ldots{})

\begin{itemize}
\item Extensionality vs intensionality
\label{sec-9-3-1-1}
\item Set theory vs type theory
\label{sec-9-3-1-2}
In set theory, we would write $3 \in \mathbb{N}$, and this is a proposition of the
language; we can write things like $x \in A \longrightarrow x \in B$. In type theory,
$3 : \mathbb{N}$ is instead a judgement. Statements such as $\mathbb{B}\cap \mathbb{N}$ are intensional:
they depend on the encoding.

Sometimes, we want to talk about intensional properties

\item Univalence
\label{sec-9-3-1-3}
Two types in a one-to-one correspondence are equal.

\item Propositions as types explanation/Curry-Howard equivalence
\label{sec-9-3-1-4}
\begin{itemize}
\item Example
\label{sec-9-3-1-4-1}
We will prove that $P \times Q \to R \iff P \to (Q \to R)$. We will define two
functions from and to the types

\begin{verbatim}
f :: ((a,b) -> c) -> (a -> b -> c)
f h x y = h (x,y)

g :: (a -> b -> c) -> ((a,b) -> c)
g h (x,y) = h x y
\end{verbatim}

these are curry/uncurry functions.
\end{itemize}

\item Products and sums
\label{sec-9-3-1-5}
Products are created as

\begin{prooftree}
\AxiomC{$a:A$}
\AxiomC{$b:B$}
\BinaryInfC{$(a,b) : A \times B$}
\end{prooftree}

and sums as

\begin{prooftree}
\AxiomC{$a:A$}
\UnaryInfC{$left(a) : A +B$}
\AxiomC{$b:B$}
\UnaryInfC{$right(b):A+B$}
\noLine
\BinaryInfC{}
\end{prooftree}

\item Definitional equality
\label{sec-9-3-1-6}
Equality given by the definition of the terms. These equalities are
static.

$\equiv$
\item Recursor
\label{sec-9-3-1-7}
The recursor is a non-dependent eliminator. It gives us the ability
of doing pattern-matching on types. For example, if we want to define
a function from a pair type using the recursor for the product

\[ \mathtt{rec}^{\times} :
(A \to B \to C) \to (A \times B) \to C
\]

or a recursor for the sum type

\[ \mathtt{rec}^+ :
(A \to C) \to (B \to C) \to (A + B \to C)
\]

the recursor for the empty type

\[ \mathrm{rec}^\bot : \bot \to C
\]

is implemented without using anything because of the nature of the
empty type.
\end{itemize}
\subsubsection*{HoTT 2}
\label{sec-9-3-2}
\begin{itemize}
\item What is a type?
\label{sec-9-3-2-1}
We allow the following judgements,

\begin{itemize}
\item $a:A$, type declarations.
\item $a \equiv_{A} b$, definitional equality.
\end{itemize}

and we define a universe of types ${\cal U}$, a type whose elements are types.

\begin{itemize}
\item Is type a type?
\label{sec-9-3-2-1-1}
If we set $Type : Type$, we can encode a version of Russell's paradox
using trees of types. Being of a type is a judgement, so we can not
encode the traditional Russell paradox.

\item Type universes
\label{sec-9-3-2-1-2}
We are going to use type universes $Type_0: Type_1: Type_2 : \dots$,
constructing a \textbf{predicative} hierarchy. It is a cumulative hierarchy,
where we can lift a type $A : Type_{i}$ to $\lceil A\rceil : Type_{i+1}$ and any function
$A \to B$ to $\lceil A\rceil \to \lceil B\rceil$.
\end{itemize}

\item Dependent types
\label{sec-9-3-2-2}
A \textbf{dependent type} depends on a term. An example is $Fin : \mathbb{N} \to Type$.
Another example is $Vec : Type \to \mathbb{N} \to Type$ or $Prime : \mathbb{N} \to Type$.
\item Pi-types
\label{sec-9-3-2-3}
Pi types are a generalization of function types allowing the codomain
to depend on the domain.

\begin{itemize}
\item Example: zeroes
\label{sec-9-3-2-3-1}

\[
zeroes : \prod_{n:\mathbb{N}} Vec\ \mathbb{N}\ n
\]

where

\[
zeroes\ n = (0,0,\dots,0)
\]

\item Example: theorems on naturals
\label{sec-9-3-2-3-2}

\[
pluszero : \prod_{n : \mathbb{N}} n+0 =_{\mathbb{N}} n
\]
\end{itemize}

\item Sigma-types
\label{sec-9-3-2-4}
Sigma tupes generalize product types to the case where the type of
second depends on the first. They work as dependent pairs.

\begin{itemize}
\item Example: lists
\label{sec-9-3-2-4-1}

\[
\sum_{n:\mathbb{N}} Vec\ A\ n
\]
\end{itemize}

\item Particular cases
\label{sec-9-3-2-5}
The function type is a particular case of a pi-type, while the 
product type is a particular case of a sigma-type.

\begin{itemize}
\item $\prod_{a:A} B \text{ is } A \to B$
\item $\sum_{a:A} B \text{ is } A \times B$
\end{itemize}

\item Example of predicate logic
\label{sec-9-3-2-6}
We have the following logic equivalence in predicate logic

\[
\left(\sum_{x:A}  P\ x\right)\to Q \iff \prod_{x:A} P\ x \to Q
\]

and the proof is similar to that of $((P,Q) \to R) \to (P \to Q \to R)$.
Yesterday we talked only about propositional logic.
\item Numerical interpretation
\label{sec-9-3-2-7}
If $f : n \to \mathbb{N}$, and we take $\overline{n}$ to be a type with $n$ elements

\[
\sum_{i:\overline{n}} \overline{f(i)}
=
\overline{\sum_{i=0}^{n-1} f(i)
\]

and the same is true for pi-types, they are related to a product.
\item Sum as a sigma type
\label{sec-9-3-2-8}
We can define $A + B$ using $\sum_{x:2}\text{if x then } A \text{ else } B$; and the same trick
can be used for products, taking $A \times B$ to be $\prod_{x:2}\text{if x then } A \text{ else } B$.

\[\begin{tikzcd}
    & $\sum$ &          & $\prod$ &       \\
$+$  &        & $\times$ &         & $\to$
\end{tikzcd}\]

\item Eliminators of dependent types
\label{sec-9-3-2-9}
The eliminator of the sum type is

\[
R^+ : (A \to B) \to (A \to C) \to A + B \to C
\]

and we can define a dependent version of the eliminator

\[
R^+ : \left( \prod_{x:A} C(inl(x)) \right) \to 
\left(\prod_{y:B} C(inr(y))\right) \to
\prod_{z: A+B} C(z)
\]

of which the first is a particular case.
\end{itemize}
\subsubsection*{HoTT 3}
\label{sec-9-3-3}
\begin{itemize}
\item Intensional equality
\label{sec-9-3-3-1}
\item Uniqueness of equality proofs
\label{sec-9-3-3-2}

\[
uep : \prod_{x,y:A}\prod_{p,q: x=y} p=q
\]

\begin{itemize}
\item Proof and the need for K
\label{sec-9-3-3-2-1}
It has been proved that this does not depend on J using
countermodels. We need to add another eliminator called K.
If we have

\[
C : \prod_{x:A} x =x \to Type
\]

then

\[
K_C : \prod_{x:A} C\ x\ (refl\ x) \to \prod_{x:A}\prod_{p:x=x} C\ x\ p
\]

\item What can we proof without K?
\label{sec-9-3-3-2-2}
The groupoid structure of paths can be proven wihtout K.

\[
\prod_{x,y:A}\prod_{p : x=y}
trans\ p\ refl = p
\]
\end{itemize}

\item Extensionality
\label{sec-9-3-3-3}
We need extensionality to prove

\[
\lambda x. x+0 = \lambda x.0+x
\]

using that $f x = g x$ for all $x$ implies $f = g$.

\begin{itemize}
\item Product
\label{sec-9-3-3-3-1}
The equality of a product is the product of two equalities;
the equality of a coproduct is a coproduct, and so on.

\item Equality of types
\label{sec-9-3-3-3-2}
Equivalence or isomorphism of types can be defined with two
mutually inverse functions between them. They give us a one-to-one
correspondence between types. This is written as $A \simeq B$.

We would need

\[
\eta : \prod_{x:A} g(f(x)) = x
\]

and

\[
\varepsilon : \prod_{y:B} f(g(y)) = y
\]

We could use J to prove

\[
\prod_{A,B: {\cal U}} A=B \to A \simeq B
\]

but this is not provable.

\item Automorphisms of Bool
\label{sec-9-3-3-3-3}
There are two proofs of equality of Bool to Bool.

There are two ways of proving $f(g(f(x))) = f(x)$ with the previous
definition.

We fix that with

\[
\tau : \prod_{x:A} f(\eta (x)) = \varepsilon f(x)
\]

\item Definition of equivalence
\label{sec-9-3-3-3-4}
This definition of isomorphism 

\[
isequiv(f) = \prod_{b:B} iscontractible \left( \sum_{a:A} f(a) = b \right)
\]

is equivalent to our previous definition of equivalence.
\item Isomorphisms and equivalence
\label{sec-9-3-3-3-5}
There are more isomorphisms than equivalences, but for every
isomorphism, we can build an equivalence

\[
A \simeq B \iff A\cong B
\]

The previous definition of univalence was unsound because it
made isomorphisms and equivalences equal.
\end{itemize}
\end{itemize}
\subsubsection*{HoTT 4}
\label{sec-9-3-4}
\begin{itemize}
\item What is a proposition?
\label{sec-9-3-4-1}
\item Axiom of choice
\label{sec-9-3-4-2}
Diaconescu; from the set-theoretical axiom of choice, we get that,
for all propositions the LEM holds, $\prod_{P:Prop} P \vee \neg P$.
\item Sets and propositions
\label{sec-9-3-4-3}

\[ \mathtt{isSet}\ A \equiv
\prod_{x,y:A} \mathtt{isProp}(x =_{A} y)
\]

$Type_0$ is an example of something that is not a set. There are two
different proofs of the equality $Bool = Bool$.

\item n-Types
\label{sec-9-3-4-4}

\[
isntype(A) \equiv
\prod_{x,y:A} is(n-1)type(x = y)
\]

An n-type is also an (n+1)-type.

\begin{center}
\begin{tabular}{rl}
-2 & Contractible type\\
-1 & Proposition\\
0 & Set\\
1 & Groupoid\\
2 & 2-Groupoid\\
\ldots{} & \ldots{}\\
\end{tabular}
\end{center}

The sphere $\mathbb{S}^2$ is not an n-type for any n.

\item Hedberg's theorem
\label{sec-9-3-4-5}
Given $A:Type$ with decidable equality

\[
d : \forall x,y : A.\quad x=y \vee x \neq y
\]

it is a set, $\mathtt{isSet}(A)$.
\end{itemize}
\subsubsection*{HoTT 5}
\label{sec-9-3-5}
\begin{itemize}
\item Negative translation of classical logic
\label{sec-9-3-5-1}

\begin{itemize}
\item $A \vee B \mapsto \neg (\neg P \wedge \neg Q)$
\item $\exists_{x:A}B(x) \mapsto \neg \prod_{x:A} \neg B(x)$
\end{itemize}
\end{itemize}
\section*{Informática gráfica}
\label{sec-10}
\subsection*{Detalles}
\label{sec-10-1}
\texttt{curena@ugr.es}
\texttt{lsi.ugr.es/curena}
\url{http://lsi.ugr.es/doce/ig/17-18}

1 punto por trabajo aparte
Defensa de prácticas con 1 semana antelación

\subsection*{Tema 1}
\label{sec-10-2}
\subsubsection*{Sección 2. Proceso de visualización}
\label{sec-10-2-1}
Modelo de escena

\begin{itemize}
\item geometría
\item texturas
\item fuentes de luz
\item materiales
\end{itemize}

\subsection*{Ejercicios}
\label{sec-10-3}
\subsubsection*{Ejercicio 12}
\label{sec-10-3-1}
\begin{itemize}
\item apartado a
\label{sec-10-3-1-1}
4by

vértices = (n+1)*(m+1) = nm + n + m + 1
caras = 2nm

floats = 3*vertices = 3nm + 3n + 3m + 3
ints = 3*caras = 6nm

tamaño = 4*ints + 4*floats = 24nm + 12nm + 12n + 12m + 12 = 36nm + 12n + 12m + 12

\item apartado b
\label{sec-10-3-1-2}
592908

\item apartado c
\label{sec-10-3-1-3}
1/2
\end{itemize}

\subsubsection*{Ejercicio 13}
\label{sec-10-3-2}
verticestira = 2n+2
totalvert = verticestira * m
tamaño = 4*totalvert = 3*4m(2n+2) = 24nm + 24m

\section*{Desarrollo y sistemas de información}
\label{sec-11}
\subsection*{2. Diseño conceptual}
\label{sec-11-1}
\section*{Profunctor Optics - Bartosz Milewski}
\label{sec-12}
\begin{verbatim}
type Lens s t a b  = forall p. Strong p => p a b -> p s t
type Prism s t a b = forall p. Choice p => p a b -> p s t

class Profunctor p => Strong p where
  first' :: p a b -> p (a,c) (b,c)

class Profunctor p => Choice p where
  left' :: p a b -> p (Either a c) (Either b c)

class Profunctor p where
  dimap :: (a -> b) -> (c -> d) -> (p b c -> p a d)
\end{verbatim}

A profunctor is a bifunctor of the form ${\cal C}^{op} \times {\cal C} \to \mathsf{Set}$.
In principle, we are not constrained to a single category,
the important notion is that the functor must be contravariant
on the first argument and covariant on the second.

\begin{verbatim}
type f ~> g = forall x. f x -> g x
\end{verbatim}

Parametricity implies naturality (?).

We have defined natural transformations as polymorphic functions.
Is this equivalent to the usual definition of natural transformation using naturality squares?

But the usual definition of natural transformations talks about naturality squares
and naturlity conditions. Are these two definitions equivalent?

Is this equivalent to the usual definition of natural transformation?
That is, does every polymorphic function satisfy the naturality condition?


\subsection*{Yoneda Lemma}
\label{sec-12-1}
\begin{verbatim}
type Reader a x = a -> x
type Yo f a = Functor f => Reader a ~> f
-- Yo f a ~ f a
-- forall x. (a -> x) -> f x -> f a

toYo :: Functor f => f a -> Yo f a
toYo fa = \atox -> fmap atox fa

fromYo :: Functor f => Yo f a -> f a
fromYo alpha = alpha id
\end{verbatim}

The Yoneda embedding

\begin{verbatim}
forall x. (a -> x) -> (b -> x) ~ (b -> a)
\end{verbatim}

\section*{Monad transformers - Snoyman}
\label{sec-13}
Concurrency with IO a and IO b.

\section*{Mikrokosmos - Mario Román}
\label{sec-14}
\subsection*{Cálculo lambda sin tipos}
\label{sec-14-1}
Una expresión lambda es

\begin{itemize}
\item una variable,
\item una aplicación de dos expresiones $M\ N$,
\item una abstracción $(\lambda x.M)$, donde $M$ es un término que depende de $x$.
\end{itemize}

Una abstracción aplicada a otro término se puede reducir como

\[
(\lambda x.M)\ N \longrightarrow_{\beta} M[N/x]
\]

y las aplicaciones asocian a izquierda: $M\ N\ P$ se lee como $(M\ N)\ P$
en vez de $M\ (N\ P)$.

$\backslash$[

$\backslash$]

\subsection*{El intérprete}
\label{sec-14-2}
Podéis instalarlo desde github si tenéis Haskell y si no, podéis usarlo
directamente desde la página web

\begin{itemize}
\item \url{https://github.com/m42/mikrokosmos}
\item \url{https://m42.github.io/mikrokosmos/tutorial.html}
\end{itemize}

En Mikrokosmos, las lambdas se escriben como una \textbf{barra invertida}, y
el programa responde con la expresión lambda y una lista de posibles
nombres que tiene esa expresión.

\begin{verbatim}
mikro> (\x.x)
λa.a ⇒ I, ifelse, id
\end{verbatim}

Para ver cómo funciona, se pueden probar algunas expresiones aritméticas
simples

\begin{verbatim}
mult 2 3
plus 3 4
and true false
sum (take 5 naturals)
\end{verbatim}

Características:

\begin{itemize}
\item los argumentos van separados por espacios,
\item se entiende asociatividad a izquierda, y
\item se permite aplicación parcial.
\end{itemize}

Es un pequeño lenguaje de programación y está completamente basado en el 
cálculo lambda. Quiero explicaros cómo se puede obtener un lenguaje de
programación desde el cálculo lambda.

\subsection*{Primeras definiciones}
\label{sec-14-3}
Vamos a usar cálculo lambda. Las expresiones lambdas se leen
como 

\begin{verbatim}
(\x.\y.plus x y)
plus 3 4
(\e.plus e e) 3
\end{verbatim}

Diciendo: esta es una función que toma \texttt{x} e \texttt{y} y devuelve \texttt{x+y}.
Lo que vamos a aprender es cómo funcionan por dentro los números
o la función \texttt{plus}.

La función \textbf{identidad} y la función \textbf{constantemente}.

\begin{verbatim}
id = \x.x
const = \x.\y.x

id id
id const
id 3
id 5
const 4 2
const 4 3
const 4 (id (const id id))

devuelvecuatro = const 4
devuelvecuatro 5
\end{verbatim}

\subsection*{Técnica de Church}
\label{sec-14-4}
Queremos usar estructuras de datos. Tenemos primero que escribir
la estructura de datos como constructores y hacer depender de ellos
a los términos.

\begin{verbatim}
true = \t.\f.t
false = \t.\f.f

0 = \s.\z.z
1 = \s.\z.s z

cons = \h.\t.\c.\n.c h (t c n)
nil = \c.\n.n
\end{verbatim}

\subsection*{Librería}
\label{sec-14-5}
\subsubsection*{Básica}
\label{sec-14-5-1}
\begin{verbatim}
id = \x.x
const = \x.\y.x
compose = \g.\f.\x.g (f x)
\end{verbatim}

\subsubsection*{Booleanos}
\label{sec-14-5-2}
\begin{verbatim}
true = \x.\y.x
false = \x.\y.y

not = \p.p false true

and = \p.\q.p q false
and = \p.\q.p q p

or = \p.\q.p true q
or = \p.\q.p p q
\end{verbatim}

\subsubsection*{Aritmética básica}
\label{sec-14-5-3}
\begin{verbatim}
0 = \f.\x.x
succ = \n.\f.\x.f (n f x)

plus = \m.\n.n succ m
plus = \m.\n.(\f.\x.n f (m f x))

mult = \m.\n.compose m n
mult = \m.\n.\f.\x.m (n f) x

iseven = \n.n not true
iszero = \n.n (const false) true
\end{verbatim}

\subsubsection*{Tuplas}
\label{sec-14-5-4}
\begin{verbatim}
tuple = \x.\y.\z.z x y

first = \p.p true
second = \p.p false

pred = \n.first (n (\t.tuple (first t) (succ (first t))) (tuple 0 0))
minus = \m.\n.n pred m
leq = \m.\n.iszero (minus m n)
geq = \m.\n.iszero (minus n m)
\end{verbatim}

\subsubsection*{Listas}
\label{sec-14-5-5}
\begin{verbatim}
nil = \c.\n.n
cons = \h.\l.(\c.\n.c h (l c n))

fold = \o.\n.\l.l o n

sum = fold plus 0
prod = fold mult 1
all = fold and true
any = fold or false
length = fold (\h.\t.)

map = \f.fold (\h.\t.cons (f h) t) nil
filter = \p.fold (\h.t.(p h) (cons h t) t) nil

head = fold const nil
tail = \l.first (l (\a.\t.tuple (second t) (cons a (second t))) (tuple nil nil))
take = \n.\l.first (n (\t.tuple (cons (head (second t)) (first t)) (tail (second t))) (tuple nil l))
\end{verbatim}

\subsubsection*{Árboles}
\label{sec-14-5-6}
\begin{verbatim}
nil = \d.\n.n
node = \x.\l.\r.\d.\n.(d x (l d n) (r d n))
\end{verbatim}

\subsubsection*{Recursión}
\label{sec-14-5-7}
\begin{verbatim}
omega := (\x.x x)(\x.x x)
fix := (\f.(\x.f (x x)) (\x.\f (x x)))

fact := fix (\f.\n.iszero n 1 (mult n (f (pred n))))
fib :=  fix (\f.\n.iszero n 1 (plus (f (pred n)) (f (pred (pred n)))))

infinity := fix succ
naturals := fix (compose (cons 0) (map succ))
\end{verbatim}

\subsubsection*{Tipos}
\label{sec-14-5-8}

\section*{The formation of swarms as a consesus problem - Ulrich Krause}
\label{sec-15}
Estructuras complejas globales emergiendo de interacciones locales.
We will use topology instead of differential equations.

\subsection*{Model}
\label{sec-15-1}
Ensemble of birds in $\mathbb{R}^{d}$. Others $d$ different than 3 are allowed.
Position $x_i$ and velocity $v_i$ of each bird. The align by averaging.

\[
v_i = \sum_{j \in N} a_{ij}(t) v_i(t)
\]

The coefficients $a_{ij}$ model intensity of interactions; they depend
on the time. The set of seen birds is

S\[
S(i,t) = \left\{ j \in N \mid a_{ij}(t) > 0 \right\}
\]

\subsubsection*{Swarm formation}
\label{sec-15-1-1}
A swarm can be formed if

\[
\lim_t v_{i}(t) = v
\]

\subsection*{Swarm formation - theorem 1}
\label{sec-15-2}
\subsubsection*{Two assumptions}
\label{sec-15-2-1}
\begin{itemize}
\item Structure not too loose.
\item Interaction does not decay too fast.
\end{itemize}
\subsection*{Swarm formation 2}
\label{sec-15-3}
Interaction only at certain points of time.

\subsubsection*{Core of a stochastic matrix}
\label{sec-15-3-1}
If $A$ has positive diagonal, $\mathrm{cor}(A) \neq \varnothing \iff A^k$ is scrambling;
we call these matrices \textbf{coherent}.

New conditions

\begin{itemize}
\item structure of matrix not too loose;
\item intensity of interaction decays not too fast;
\item intensity of interaction decays slowly.
\end{itemize}

That can be interpreted as

\begin{itemize}
\item every bird sees itself,
\item there is a sight chain to a leader.
\end{itemize}

\subsection*{Flight formations}
\label{sec-15-4}
\subsubsection*{V-formation and echelon}
\label{sec-15-4-1}
A leader is the only one in the core
\subsubsection*{Other possible cores: sterling clouds}
\label{sec-15-4-2}
Loops in the sight chain; connected loops.

\begin{itemize}
\item Systematic account of flight formations?
\label{sec-15-4-2-1}
Graphs changing in time.

\item Computer simulations?
\label{sec-15-4-2-2}
\end{itemize}

\subsection*{Sight cones / cones of vision}
\label{sec-15-5}
Given by direction of flight. Non-convex cones would be also an
option.

\subsubsection*{Farkas lemma}
\label{sec-15-5-1}
\subsubsection*{Helly's theorem}
\label{sec-15-5-2}
\subsection*{Models of intensity of interaction}
\label{sec-15-6}
Cucker-Smale model of bird flocking.

\section*{Local variables}
\label{sec-16}
% Emacs 25.3.2 (Org mode 8.2.10)
\end{document}
